\begin{ex}
    \label{ex10.46}
    $S = <X_1 X_2, X_2 X_3>$とすると, $V_S$は$\ket{+++}$と$\ket{---}$で張られる.
\end{ex}

\begin{ex}
    \label{ex10.47}
    $i = 1, 2 ... , 8$に対して,
    $g_i \ket{0_L} = \ket{0_L}$, $g_i \ket{1_L} = \ket{1_L}$を確かめれば良い.
\end{ex}

\begin{ex}
    \label{ex10.48}
    計算すれば良い.
\end{ex}

\begin{ex}
    \label{ex10.49}
    例えば, $X_1Z_2 \notin N(S) - S$か調べる.
    \begin{align*}
        \{X_1Z_2, g2\}
        = \{ X_1 Z_2, X_2 Z_3 Z_4 X_5\}
        = \{ X_1 Z_2 , X_2\}Z_3 Z_4 X_5
        = (X_1 Z_2 X_2 + X_2 X_1 Z_2)Z_3 Z_4 X_5
        = 0
    \end{align*}
    であり, $l = 1, 2,3,4,5$に対しても同様に$\{X_1Z_2, g_l\} = 0$なので, $X_1Z_2 \in G_n - N(S)$. よって, $X_1Z_2 \notin N(S) - S$.
    \par
    同様のことを任意の1 qubit誤りの積$\sigma_j \sigma_k$について調べれば良い.
\end{ex}

\begin{ex}
    \label{ex10.50}
    \begin{align*}
        2(1+3n) \leq 2^n
    \end{align*}
    を満たす最大の自然数$n = 5$.
\end{ex}

\begin{ex}
    \label{ex10.51}
\end{ex}

\begin{ex}
    \label{ex10.52}
    計算すれば良い.
\end{ex}

\begin{ex}
    \label{ex10.53}
    $G_z =[000|A_2^\top 0 I]$より, 明らか.
\end{ex}

\begin{ex}
    \label{ex10.54}
    $G$の検査行列の標準形
    \begin{align*}
        \begin{pmatrix}
            I & A_1 & A_2 & B & 0 & C \\
            0 & 0   & 0   & D & I & E \\
        \end{pmatrix}
    \end{align*}
    とかく.
    \par
    $G$の生成元$g_i (i = 1, ..., r)$と$G_x = [0 E^\top I | C^\top 0 0 ]$の生成元$g_j^x$について.
    \begin{align*}
        r(g_i) \Lambda r(g_j^x) = ... = 2 C_{ij} = 0
    \end{align*}
    であることと, 演習\ref{ex10.33}より, $g_i (i = 1, ..., r)$と$g_j^x$は可換. 独立性は明らか.
    \par
    $G$の生成元$g_i (i = r+1, ..., n-k)$と$G_x = [0 E^\top I | C^\top 0 0 ]$の生成元$g_j^x$についても上と同様.
    \par
    $G_z = [0 0 0 | A_2^\top 0 I ]$の生成元$g_j^z$と$G_x = [0 E^\top I | C^\top 0 0 ]$の生成元$g_j^x$について.
    \begin{align*}
        r(g_i^z) \Lambda r(g_j^x) = ... = \delta_{ij}
    \end{align*}
    なので, $g_i^z$と$g_j^x$は, $i=j$のとき反可換, $i \neq j$のとき可換. 独立性は明らか.
\end{ex}

\begin{ex}
    \label{ex10.55}
    $G_x$の検査行列は,
    \begin{align*}
        \left(
        \begin{array}{c|c|c|c|c|c}
            000 & 110 & 1 & 000 & 000 & 0
        \end{array}
        \right)
    \end{align*}
    より, $\bar{X} = X_4 X_5 X_7$.
\end{ex}

\begin{ex}
    \label{ex10.56}
    $\ket{\psi}\in V_S, g \in S$として, $g$と$\bar{X}$は可換なので,
    \begin{align*}
        g \bar{X} \ket{\psi} =  \bar{X} g  \ket{\psi} = \bar{X} \ket{\psi}.
    \end{align*}
    $\bar{Z}$についても同様.
\end{ex}

\begin{ex}
    \label{ex10.57}
    \underline{5 qubit}
    \begin{align*}
        \left(
        \begin{array}{c|c}
            10001 & 11011 \\
            01001 & 00110 \\
            00101 & 11000 \\
            00011 & 10111
        \end{array}
        \right)
    \end{align*}
    \underline{9 qubit}
    \begin{align*}
        \left(
        \begin{array}{c|c}
            101111001 &           \\
            010101111 &           \\
            \hline
                      & 101000000 \\
                      & 100100000 \\
                      & 000010001 \\
                      & 000001001 \\
                      & 010000100 \\
                      & 000000010
        \end{array}
        \right)
    \end{align*}
\end{ex}

\begin{ex}
    \label{ex10.58}
    説明通りに操作することは, 演習\ref{ex4.34}に示した.
    \par
    図10.14の回路の等価性:
    \begin{align*}
        \Qcircuit @C=1em @R=1em {
        \lstick{\ket{0}}                        & \gate{H} & \ctrl{1} & \gate{H} & \meter & \raisebox{-3.0em}{=} &  &  &  &  &  &  &  & \lstick{\ket{0}}                        & \qw      & \targ     & \qw      & \meter \\
        \lstick{\alpha \ket{0} + \beta \ket{1}} & \qw      & \targ    & \qw      & \qw    &                      &  &  &  &  &  &  &  & \lstick{\alpha \ket{0} + \beta \ket{1}} & \gate{H} & \ctrl{-1} & \gate{H} & \qw
        }
    \end{align*}
    について示す.
    演習\ref{ex4.34}より, 左辺の回路の終状態は, $\ket{\psi_{in}} = \alpha \ket{0} + \beta \ket{1}$として,
    \begin{align*}
        \ket{0} \otimes \ket{+} \braket{+|\psi_{in}}
        +
        \ket{1}  \otimes \ket{-}\braket{-|\psi_{in}}
    \end{align*}
    一方, 右辺の回路の終状態は,
    \begin{align*}
        \frac{\alpha + \beta}{\sqrt{2}} \ket{0} \ket{+} + \frac{\alpha - \beta}{\sqrt{2}} \ket{1} \ket{-}
        =
        \ket{0} \otimes \ket{+} \braket{+|\psi_{in}}
        +
        \ket{1}  \otimes \ket{-}\braket{-|\psi_{in}}.
    \end{align*}
    よって, 示せた.
    図10.15の回路の等価性についても同様に示せる.
\end{ex}

\begin{ex}
    \label{ex10.59}
    明らか.
\end{ex}

\begin{ex}
    \label{ex10.60}
    演習\ref{ex10.57}で求めた, 検査行列の標準形をみて, シンドローム回路を作ると以下のようになる.
    \begin{align*}
        \Qcircuit @C=1.5em @R=1.5em {
        \lstick{\ket{0}} & \qw      & \targ     & \qw       & \qw       & \qw       & \targ     & \qw      & \targ     & \targ     & \qw       & \targ     & \targ     & \qw \\
        \lstick{\ket{0}} & \qw      & \qw       & \targ     & \qw       & \qw       & \targ     & \qw      & \qw       & \qw       & \targ     & \targ     & \qw       & \qw \\
        \lstick{\ket{0}} & \qw      & \qw       & \qw       & \targ     & \qw       & \targ     & \qw      & \targ     & \targ     & \qw       & \qw       & \qw       & \qw \\
        \lstick{\ket{0}} & \qw      & \qw       & \qw       & \qw       & \targ     & \targ     & \qw      & \targ     & \qw       & \targ     & \targ     & \targ     & \qw \\
        \lstick{}        & \gate{H} & \ctrl{-4} & \qw       & \qw       & \qw       & \qw       & \gate{H} & \ctrl{-4} & \qw       & \qw       & \qw       & \qw       & \qw \\
        \lstick{}        & \gate{H} & \qw       & \ctrl{-4} & \qw       & \qw       & \qw       & \gate{H} & \qw       & \ctrl{-5} & \qw       & \qw       & \qw       & \qw \\
        \lstick{}        & \gate{H} & \qw       & \qw       & \ctrl{-4} & \qw       & \qw       & \gate{H} & \qw       & \qw       & \ctrl{-5} & \qw       & \qw       & \qw \\
        \lstick{}        & \gate{H} & \qw       & \qw       & \qw       & \ctrl{-4} & \qw       & \gate{H} & \qw       & \qw       & \qw       & \ctrl{-7} & \qw       & \qw \\
        \lstick{}        & \gate{H} & \qw       & \qw       & \qw       & \qw       & \ctrl{-8} & \gate{H} & \qw       & \qw       & \qw       & \qw       & \ctrl{-8} & \qw
        }
    \end{align*}
\end{ex}

\begin{ex}
    \label{ex10.61}
    $E_j (j=0, ..., 21)$は単一qubit誤り.
\end{ex}

\begin{ex}
    \label{ex10.62}
\end{ex}

\begin{ex}
    \label{ex10.63}
    \begin{align*}
        \bar{U} \ket{0_L} & = \bar{U} \bar{Z} \ket{0_L} = \bar{X} \bar{U} \ket{0_L}      \\
        \bar{U} \ket{1_L} & = - \bar{U} \bar{Z} \ket{1_L} =  - \bar{X} \bar{U} \ket{1_L} \\
    \end{align*}
    より, $\bar{U} \ket{0_L}$は$\bar{X}$の固有値1の固有状態, $\bar{U} \ket{1_L}$は$\bar{X}$の固有値-1の固有状態. よって,
    \begin{align*}
        \bar{U} \ket{0_L} \propto \frac{\ket{0_L} + \ket{1_L}}{\sqrt{2}} \\
        \bar{U} \ket{1_L} \propto \frac{\ket{0_L} - \ket{1_L}}{\sqrt{2}}.
    \end{align*}
\end{ex}

\begin{ex}
    \label{ex10.64}
    \begin{align*}
        \Qcircuit @C=1em @R=1em {
        \lstick{} & \qw      & \ctrl{1} & \qw & \raisebox{-3.0em}{=} & \lstick{} & \ctrl{1} & \gate{Z} & \qw \\
        \lstick{} & \gate{Z} & \targ    & \qw &                      & \lstick{} & \targ    & \gate{Z} & \qw
        }
    \end{align*}
\end{ex}

\begin{ex}
    \label{ex10.65}
    $\ket{\psi} = \alpha \ket{0} + \beta \ket{1}$とかく.
    \begin{align*}
        \Qcircuit @C=1em @R=1em {
        \lstick{\ket{0}}    & \targ     & \qw      & \gate{Z}    & \qw & \rstick{\ket{\psi}} \\
        \lstick{\ket{\psi}} & \ctrl{-1} & \gate{H} & \meter \cwx &
        }
    \end{align*}
    上の回路の作用は,
    \begin{align*}
        \alpha \ket{0} \ket{0} + \beta \ket{0} \ket{1}
         & \longrightarrow
        \alpha \ket{0} \ket{0} + \beta \ket{1} \ket{1}                                  \\
         & \longrightarrow
        \frac{\alpha}{\sqrt{2}}\ket{0} \ket{0} + \frac{\alpha}{\sqrt{2}}\ket{0} \ket{1}
        + \frac{\beta}{\sqrt{2}}\ket{1} \ket{0} - \frac{\beta}{\sqrt{2}}\ket{1} \ket{1} \\
         & \longrightarrow
        Z (\alpha \ket{0} - \beta \ket{1}) = \alpha \ket{0} + \beta \ket{1} = \ket{\psi}
    \end{align*}
\end{ex}

\begin{ex}
    \label{ex10.66}
    明らか.
\end{ex}

\begin{ex}
    \label{ex10.67}
    明らか.
\end{ex}