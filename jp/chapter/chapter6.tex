\chapter{量子探索アルゴリズム}

\begin{ex}
    \label{ex6.1}
    \begin{align*}
        \left( 2 \ket{0} \bra{0} - I \right)
        \left( 2 \ket{0} \bra{0} - I \right)^\dagger
        =
        I.
    \end{align*}
\end{ex}


\begin{ex}
    \label{ex6.2}
    \begin{align*}
        \left( 2 \ket{\psi} \bra{\psi} - I \right) \sum_k \alpha_k \ket{k}
         & =
        \frac{2}{N} \sum_x \sum_y \sum_k \alpha_k \ket{x} \braket{y|k}- \sum_k \alpha_k \ket{k}
        \\
         & =
        \frac{2}{N} \sum_x \sum_y \sum_k \alpha_k \ket{x}
        \delta_{yk}- \sum_k \alpha_k \ket{k}
        \\
         & =
        \frac{2}{N} \sum_x \sum_k \alpha_k \ket{x}- \sum_k \alpha_k \ket{k}
        \\
         & =
        2 \braket{\alpha} \sum_x \ket{x} - \sum_k \alpha_k \ket{k}
        \\
         & =
        \sum_k  \left[- \alpha_k + 2 \braket{\alpha} \right] \ket{k}.
    \end{align*}
\end{ex}

\begin{ex}
    \label{ex6.3}

    $\ket{\alpha}, \ket{\beta}$基底で,
    \begin{align*}
        \begin{pmatrix}
            \cos \theta & - \sin \theta \\
            \sin \theta & \cos \theta
        \end{pmatrix}
        \begin{pmatrix}
            \cos (\theta/2) \\
            \sin (\theta/2)
        \end{pmatrix}
        =
        \begin{pmatrix}
            \cos (3\theta/2) \\
            \sin (3\theta2)
        \end{pmatrix}
    \end{align*}
    であることと
    \begin{align*}
        G \left( \cos \frac{\theta}{2}\ket{\alpha} + \sin \frac{\theta}{2} \ket{\beta}\right)
        =
        \cos \frac{3\theta}{2}\ket{\alpha} + \sin \frac{3\theta}{2} \ket{\beta}.
    \end{align*}
    を見比べれば良い.
\end{ex}

\begin{ex}
    \label{ex6.4}
    P114の手続きと全く同様.
\end{ex}
\begin{ex}
    \label{ex6.5}
    考えたい探索問題$f(x)$, 拡張した探索問題$f'(x,q)$とかくと,
    \begin{align*}
        f'(x,q) =
        \begin{cases}
            f(x) & (q=0) \\
            0 \  & (q=1)
        \end{cases}
    \end{align*}
    なので,
    拡張したオラクル$O'$;
    \begin{align*}
        \ket{q}\ket{x} \xrightarrow{O'} (-1)^{f'(x,q)}\ket{q}\ket{x}
        =
        \begin{cases}
            (-1)^{f(x)}\ket{q}\ket{x} & (q=0) \\
            \ket{q}\ket{x}            & (q=1)
        \end{cases}
    \end{align*}
    を実現する量子回路は
    \begin{align*}
        \Qcircuit @C=1em @R=1em {
        \lstick{\ket{q}} & \gate{X} & \ctrl{1} & \gate{X} & \qw \\
        \lstick{\ket{x}} & {/} \qw  & \gate{O} & {/} \qw  & \qw
        }
    \end{align*}
    となる.
\end{ex}

\begin{ex}
    \label{ex6.6}
    点線の箱へのInputを$\ket{a} \ket{b}$として, 点線の箱のInputへの作用は
    \begin{align*}
        \ket{a} \ket{b} \to \ket{a}XHX^{1-a}HX\ket{b}
         & =
        \begin{cases}
            \ket{a} XHXHX \ket{b} \  & (a = 0) \\
            \ket{A} XHHX \ket{b} \   & (a=1)
        \end{cases} \\
         & =
        \begin{cases}
            - \ket{a}Z \ket{b} \  & (a = 0) \\
            \ket{a} \ket{b} \     & (a=1)
        \end{cases} \\
         & =
        \begin{cases}
            - \ket{a}Z \ket{b} \  & (a = 0, b =0) \\
            \ket{a}Z \ket{b} \    & (a = 0, b =1) \\
            \ket{a} \ket{b} \     & (a=1)
        \end{cases} \\
         & =
        I - 2 \ket{00} \bra{00}
    \end{align*}

\end{ex}

\begin{ex}
    \label{ex6.7}
    図6.4の
    \begin{align*}
        \Qcircuit @C=1em @R=1em {
        \lstick{\ket{\phi}}         & {/} \qw   & \multigate{1}{O} & {/}\qw & \multigate{1}{O} & {/}\qw & \qw \\
        \lstick{\ket{0}}            & \qw       & \ghost{O}        & \gate{
        \begin{pmatrix}
                1 & 0                \\
                0 & e^{- i \Delta t}
            \end{pmatrix}} & \ghost{O} & \qw              & \qw
        }
    \end{align*}
    という回路について考える. ここで, オラクル$O$は,
    \begin{align*}
        \ket{\phi} \ket{0} \xrightarrow{O}
        \begin{cases}
            \ket{\phi} \ket{1} & (x = \phi)           \\
            \ket{\phi} \ket{0} & (\mathrm{otherwise}) \\
        \end{cases}
    \end{align*}
    と作用する.
    すると, 上記の回路の作用は,
    \begin{align*}
        \ket{\phi} \ket{0}
        \xrightarrow{O}
        \begin{cases}
            \ket{\phi} \ket{1} & (x = \phi)           \\
            \ket{\phi} \ket{0} & (\mathrm{otherwise}) \\
        \end{cases}
        \to
        \begin{cases}
            e^{-i \Delta t}\ket{\phi} \ket{1} & (x = \phi)           \\
            \ket{\phi} \ket{0}                & (\mathrm{otherwise}) \\
        \end{cases}
        \xrightarrow{O}
        \begin{cases}
            e^{-i \Delta t}\ket{\phi} \ket{0} & (x = \phi)           \\
            \ket{\phi} \ket{0}                & (\mathrm{otherwise}) \\
        \end{cases}
    \end{align*}
    第1qビットに注目すると, この回路の作用は$\exp{ \left[- i \ket{x}\bra{x} \Delta t \right]}$.
    図6.5についても同様.
\end{ex}

\begin{ex}
    \label{ex6.8}
    必要なステップ数は,
    \begin{align*}
        \frac{t}{\Delta t} = O \left( \frac{\sqrt{N}}{\Delta t}\right).
    \end{align*}
    全ステップ後の累積誤差は,
    \begin{align*}
        O \left( \frac{\sqrt{N}}{\Delta t} \Delta t ^r\right).
    \end{align*}
    累積誤差が$O(1)$になるには,
    \begin{align*}
        1 = O  \left( \frac{\sqrt{N}}{\Delta t} \Delta t^r\right)
        \to O \left( \Delta t\right) = O \left( N^{ - \frac{1}{2(r-1)}}\right).
    \end{align*}
    よって, 高い成功率を得るには,
    \begin{align*}
        \frac{t}{\Delta t}
        = O \left( \frac{\sqrt{N}}{N^{ - \frac{1}{2(r-1)}} }\right)
        = O \left( N^{\frac{r}{2(r-1)}}\right).
    \end{align*}
\end{ex}

\begin{ex}
    \label{ex6.9}
    演習\ref{ex4.15}そのまま.
\end{ex}

\begin{ex}
    \label{ex6.10}
    \begin{align*}
        \frac{t}{\Delta t } = \frac{\pi \sqrt{N}}{2 \Delta t}
    \end{align*}
    が$O(\sqrt{N})$かつ整数になるような$\Delta t$を選べば良い.
\end{ex}

\begin{ex}
    \label{ex6.11}
    \begin{align*}
        H = \sum_{x =1}^m \ket{x}\bra{x} + \ket{\psi}\bra{\psi}
    \end{align*}
\end{ex}

\begin{ex}
    \label{ex6.12}
    (1)\
    $\ket{\psi}, \ket{x}$で張られる空間の正規直交基底$\ket{x}, \ket{y}$として,
    $\ket{\psi} = \alpha \ket{x} + \beta \ket{y}$とかく. $\ket{x}, \ket{y}$基底で考える.
    Hamiltonianは,
    \begin{align*}
        H = \ket{x}\bra{\psi} + \ket{\psi} \bra{x}
        =
        \alpha I + (\beta X + \alpha Z)
    \end{align*}
    となるので, 時間発展op.は,
    \begin{align*}
        e^{- i H t}
         & =e^{- i \alpha t}e^{- i (\beta X + \alpha Z) t}                            \\
         & = e^{- i \alpha t} \left[ I \cos t - i (\beta X + \alpha Z) \sin t\right].
    \end{align*}
    よって,
    \begin{align*}
        e^{- i H t} \ket{ \psi}
        =
        e^{- i \alpha t} \left[ \cos t \ket{\psi} - i  \sin t \ket{x}\right]
    \end{align*}
    となることから, $t = 0$で$\ket{\psi}$だった系は$t = \pi$で$\ket{x}$に時間発展する.
    \par
    (2)\
\end{ex}

\begin{ex}
    \label{ex6.13}
    $M$の推定値
    \begin{align*}
        S = \frac{N}{k} \sum_{j=1}^k X_j
    \end{align*}
    に対して,
    \begin{align*}
        E[S]   & = \frac{N}{k}\sum_{j=1}^k E[X_j] = \frac{N}{k}\sum_{j=1}^k \frac{M}{N} = M \\
        E[S^2] & = \frac{N^2}{k^2} \sum_{j, l=1}^k E[ X_j X_l]
        =\frac{N^2}{k^2}\left( \sum_{j=1}^k E[ X_j^2] +  \sum_{j\neq l} E[X_j] E[X_k]\right)
        =\frac{N^2}{k^2} \left( k \frac{M}{N} + k(k-1) \frac{M}{N} \frac{M}{N}\right)
        = M^2 + \frac{M(N-M)}{k}
    \end{align*}
    なので, $S$の標準偏差は,
    \begin{align*}
        \Delta S = \sqrt{E[S^2] - E[S] } = \sqrt{\frac{M(N-M)}{k}}.
    \end{align*}
    \par
    精度$\sqrt{M}$以内で$M$を正確に推定する確率を$p$にするには, $p$に応じて$N$に依らない定数$\alpha$をとってきて,
    \begin{align*}
        \alpha \Delta S < \sqrt{M} \to k \geq \alpha^2(N-M)
    \end{align*}
    なる$k$を選べば良い. よって, $k = \Omega({N})$が示された.
\end{ex}

\begin{ex}
    \label{ex6.14}
\end{ex}

\begin{ex}
    \label{ex6.15}
    Cauchy-Schwarzの不等式より,
    \begin{align*}
        \left( \sum_x \left| \braket{\psi|x} \right| \right)^2
        \leq
        \left( \sum_x  \left|  \braket{\psi|x} \right|^2 \right)\left(\sum_x  1\right)
        = N
    \end{align*}
    であることを用いて,
    \begin{align*}
        \sum_x \left\| \psi - x \right\|^2
         & =
        \sum_x \left\| \psi \right\|^2 +\left\| x \right\|^2 - 2 \Re \braket{\psi|x} \\
         & =
        2N - 2\sum_x \Re \braket{\psi|x}
        \\
         & \geq
        2N - 2\sum_x \left| \braket{\psi|x} \right|
        \\
         & \geq
        2N - 2\sqrt{N}.
    \end{align*}
\end{ex}
\begin{ex}
    \label{ex6.16}
    教科書P133-P134の議論に手を加える.
    \par
    まず,
    \begin{align*}
        D_k
        = \sum_x \left\|  \psi_k^x - \psi_k \right\|
        \leq 4k^2
    \end{align*}
    であることは教科書で示した通り.
    \par
    $x$の可能な値に対して一様な平均をとったときの, 誤り確率が1/2以下になる;
    \begin{align*}
        \frac{1}{N} \sum_x \left| \braket{x|\psi_k^x} \right|^2 \leq \frac{1}{2}
    \end{align*}
    と仮定する. まず, $\ket{x}$の位相をうまく取り替えることで, $\braket{\psi_k^x|x} = \left| \braket{\psi_k^x|x} \right|$とできることを用いて,
    \begin{align*}
        E_k
         & = \sum_x \left\| \psi_k^x - x \right\|^2
        \\
         & = 2N - 2 \sum_x \left| \braket{\psi_k^x|x} \right|
        \\
         &
        \leq
        2N - 2 \sqrt{\sum_x \left| \braket{\psi_k^x|x} \right|^2}
        \leq
        2N -  \sqrt{2N}.
    \end{align*}
    最後の不等号で, 仮定を用いた.
    ここで, 演習\ref{ex6.15}の結果より,
    \begin{align*}
        F_k = \sum_x \left\| \psi - x \right\|^2
        \geq 2N - 2\sqrt{N}.
    \end{align*}
    であるから, (6.51)式は, $N \neq 2$であれば,
    \begin{align*}
        D_k \geq \left( \sqrt{E_k} - \sqrt{F_k} \right)^2
         & =
        \left( \sqrt{2N -  \sqrt{2N}} - \sqrt{2N - 2\sqrt{N}} \right)^2
        \\
         & =
        N \left( \sqrt{2 -  \sqrt{\frac{2}{N}}} - \sqrt{2 - \frac{2}{\sqrt{N}}} \right)^2
        \geq cN
    \end{align*}
    となる正数$c$が存在する.
    \par
    以上より,
    \begin{align*}
        cN \leq D_k \leq 4k^2 \to k \geq \sqrt{\frac{cN}{4}}.
    \end{align*}
    つまり, 探索サイズ$N(\neq2)$の問題にを解くにあたり, 探索空間の元$x$に対して一様な平均をとったときの誤り確率が1/2以下になることを保証するならば, Oracleの呼び出し回数$k$は$\Omega(N)$.
\end{ex}

\begin{ex}
    \label{ex6.17}
    https://arxiv.org/abs/quant-ph/9605034
\end{ex}
\begin{ex}
    \label{ex6.18}
    最小次数$l$のBoole関数$F({X_1, \dots, X_N})$の一般形は, $X_k^2 = X_k$なことに注意して,
    \begin{align*}
        F({X_1, \dots, X_N}) = \sum_{i = 0}^l \sum_{j=1}^{_N C _i}  \prod_{k=1}^i \alpha_{ij} X_{\beta_{ijk}}
    \end{align*}
    とかける. ここで, $\alpha_j$は$F({X_1, \dots, X_N})$を決めるための定数で, $\beta_{ijk}$は, 1から$n$までの数から$i$個選ぶことで得られる集合を, 全てかき集めた集合族の$j$番目の要素の小さい方から$k$番目の要素である. $F({X_1, \dots, X_N})$を一意に定めるために$2^N$本の条件式を用いて,
    $\sum_{i=0}^{l} {}_N C_i \ (\leq 2^N)$
    個の$\alpha_{ij}$を決める. 条件式の数の方が, 決めるべき定数$\alpha_{ij}$の総数より多いので, $\alpha_{ij}$は与えられた$F({X_1, \dots, X_N})$に対して一意に決まる.
\end{ex}

\begin{ex}
    \label{ex6.19}
    \begin{align*}
        P(X) = 1 - (1 - X_0) \dots (1-X_{N-1})
    \end{align*}
    は, $X_0, \dots X_{N-1}$の全てが0の時のみ0になる. つまり, $P(X)$はOR.
\end{ex}