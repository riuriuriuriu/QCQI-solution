\chapter{量子情報の距離測度}

\begin{ex}
    \label{ex9.1}
    1/2, 1/4
\end{ex}

\begin{ex}
    \label{ex9.2}
    \begin{align*}
        D((p, 1-p),(q, 1-q) ) = \frac{2\left| p - q\right|}{2} = \left| p - q\right|
    \end{align*}
\end{ex}

\begin{ex}
    \label{ex9.3}
    $\frac{1}{\sqrt{2}}, \frac{4 \sqrt{6} + \sqrt{3}}{12}$
\end{ex}

\begin{ex}
    \label{ex9.4}
    \begin{align*}
        D\left(p_x , q_x\right)
        =
        \frac{1}{2} \sum_x \left| p_x - q_x \right|
         & =
        \frac{1}{2} \sum_{\{x|p_x \leq q_x\}} \left| p_x - q_x \right|
        +
        \frac{1}{2} \sum_{\{x|p_x > q_x\}} \left| p_x - q_x \right|
        \\
         & =
        \frac{1}{2} \sum_{\{x|p_x \leq q_x\}}  q_x - p_x
        +
        \frac{1}{2} \sum_{\{x|p_x > q_x\}} p_x - q_x
        \\
         & =
        \frac{1}{2} \left( 1 -\sum_{\{x|p_x > q_x\}}  q_x  \right)
        -
        \frac{1}{2} \left( 1 -\sum_{\{x|p_x > q_x\}}  p_x  \right)
        +
        \frac{1}{2} \sum_{\{x|p_x > q_x\}} p_x - q_x
        \\
         & =
        \sum_{\{x|p_x > q_x\}} p_x - q_x
        =
        \max_S
        \left(\sum_{x \in S} p_x - \sum_{x \in S} q_x \right).
    \end{align*}
    $D\left(p_x , q_x\right) = D\left(q_x , p_x\right)$なので,
    \begin{align*}
        D\left(p_x , q_x\right)
        =
        \max_S \left(\sum_{x \in S} p_x - \sum_{x \in S} q_x \right)
        =
        \max_S
        \left(\sum_{x \in S} q_x - \sum_{x \in S} p_x \right)
        =
        \max_S
        \left|\sum_{x \in S} p_x - \sum_{x \in S} q_x \right|.
    \end{align*}
\end{ex}

\begin{ex}
    \label{ex9.5}
    演習\ref{ex9.4}参照.
\end{ex}

\begin{ex}
    \label{ex9.6}
    $\frac{1}{2}, \frac{\sqrt{13}}{12}$
\end{ex}

\begin{ex}
    \label{ex9.7}
    $\rho - \sigma$はHermiteなので, 対角化可能で, 固有状態は互いに直交する. $\rho - \sigma$の固有値を$\lambda_i$とすると,
    \begin{align*}
        \rho - \sigma = \sum_i \lambda_i \ket{i}\bra{i}=
        \sum_{\{i| \lambda_i > 0\}} \lambda_i \ket{i}\bra{i}
        +
        \sum_{\{i| \lambda_i < 0\}} \lambda_i \ket{i}\bra{i}
        =
        \sum_{\{i| \lambda_i > 0\}} \lambda_i \ket{i}\bra{i}
        -
        \sum_{\{i| -\lambda_i > 0\}} -\lambda_i \ket{i}\bra{i}
        =
        Q - S.
    \end{align*}
    ここで,
    \begin{align*}
        Q = \sum_{\{i| \lambda_i > 0\}} \lambda_i \ket{i}\bra{i}, \
        S = \sum_{\{i| - \lambda_i > 0\}} -\lambda_i \ket{i}\bra{i}
    \end{align*}
    で, $Q,S$の固有空間のそれぞれの任意の元は互いに直交する.
\end{ex}

\begin{ex}
    \label{ex9.8}
    式(9.22)より, ある直交射影オペレータ$P$が存在して,
    \begin{align*}
        D \left(\sum_i p_i \rho_i, \sigma\right)
         & =
        \tr \left( P \left( \sum_i p_i \rho_i - \sigma \right)\right)
        \\
         & =
        \sum_i p_i \tr \left( P \rho_i \right)
        -
        \sum_i p_i \tr \left( P \sigma\right)
        \\
         & =
        \sum_i p_i
        \left(
        \tr \left( P \left( \rho_i - \sigma \right) \right)
        \right)
        \\
         & \leq
        \sum_i p_i D\left(\rho_i, \sigma\right)
    \end{align*}
\end{ex}

\begin{ex}
    \label{ex9.9}
    $\hilbertsp$を有限次元の複素内積空間$\mathbb{C}^n \ \left(n < \infty\right)$とする. $\mathbb{C}^{n \times n}$の位相は, ヒルベルトシュミット内積により自然に定まるノルムを備えた距離空間の位相で定める. このとき, $\mathbb{C}^{n \times n}$の部分空間$\densityopset{\mathbb{C}^{n}} = \left\{ \rho \in \mathbb{C}^{n \times n} | \tr\rho = 1,\  \rho \ge 0\right\}$が凸かつ有界閉集合であることを示す.

    $\densityopset{\mathbb{C}^{n}}$が凸であることを示す. $p \in \left[0,1\right]$, $\rho, \sigma \in \densityopset{\mathbb{C}^{n}}$とすると, 
    \begin{align*}
        \tr\left[ p \rho + \left(1-p\right)\sigma\right] = 1
    \end{align*}
    が成り立ち, 任意の$\ket{\psi} \in \mathbb{C}^n$に対して, 
    \begin{align*}
        p \braket{\psi|\rho|\psi} + (1-p) \braket{\psi|\sigma|\psi} \ge 0
    \end{align*}
    が成り立つので, $\densityopset{\mathbb{C}^{n}}$は凸.

    $\densityopset{\mathbb{C}^{n}}$が閉集合であることを示す. まず, $\tr \colon \mathbb{C}^{n \times n} \to \mathbb{C}$は連続. また, $x \in \mathbb{C}^{n}$に対して, $f_x \colon \mathbb{C}^{n \times n} \ni A \to x^\top A x \in \mathbb{C}$は連続. さらに, $\mathbb{C}$の部分集合$X := \{1\}$, $ Y := \{x + iy | x \ge 0\}$は閉集合なので, 
    \begin{align}
        \densityopset{\mathbb{C}^{n}}
        =
        \left\{ \rho \in \mathbb{C}^{n \times n} | \tr\rho = 1, \ \rho \ge 0\right\}
        =
        \tr^{-1}\left(X\right)
        \cap
        \bigcap_{x \in \mathbb{C}^n} f_x^{-1}\left(Y\right)
    \end{align}
    は閉集合. 

    $\densityopset{\mathbb{C}^{n}}$が有界であることを示す. 距離空間の有界性の定義より, $\rho, \sigma \in \densityopset{\mathbb{C}^n}$に対して, 距離$d\left(\rho, \sigma\right)$が有界であることを示せばよい. $\rho, \sigma$のスペクトル分解を
    \begin{align}
        \rho = \sum_{i=1}^n p_i \ket{i}\bra{i}, \ \sigma = \sum_{i=1}^n q_i \ket{\tilde{i}}\bra{\tilde{i}}
    \end{align}
    とすれば, 
    \begin{align}
        d \left(\rho, \sigma\right) 
        =
        \tr\left[ \left(\rho-\sigma\right)\left(\rho-\sigma\right)\right] 
        =
        \sum_{i=1}^n p_i^2 + \sum_{i=1}^n q_i^2 - 2 \sum_{i,j = 1}^n p_i q_i \left| \braket{i|\tilde{j}}\right|^2 < \infty
    \end{align}
    であるから$\densityopset{\mathbb{C}^{n}}$は有界.

    以上より, $\densityopset{\mathbb{C}^{n}}$が$\mathbb{C}^{n \times n}$の有界閉集合, つまりコンパクトであることが言えた. したがって, Schauderの不動点定理より, 連続なTPCP写像$\qo \colon \mathbb{C}^{n \times n} \to \mathbb{C}^{n \times n}$の$\densityopset{\mathbb{C}^n}$への制限$\qo|_{\densityopset{\mathbb{C}^n}} \colon \densityopset{\mathbb{C}^n} \to \densityopset{\mathbb{C}^n}$は不動点を持つので, $\qo \left(\rho\right) = \rho$を満たす$\rho \in \densityopset{\mathbb{C}^n}$が存在する.
\end{ex}

\begin{ex}
    \label{ex9.10}
    厳密に収縮的な$\qo$が2つの不動点を持つとすると,
    \begin{align*}
        \qo \left( \rho\right) = \rho, \ \qo \left( \sigma\right) = \sigma
    \end{align*}
    となるが,
    \begin{align*}
        D \left( \rho , \sigma\right)
        =
        D \left( \qo(\rho) , \qo(\sigma)\right)
        <
        D \left( \rho , \sigma\right)
    \end{align*}
    と矛盾.
\end{ex}

\begin{ex}
    \label{ex9.11}
    同時凸性より,
    \begin{align*}
        D \left( \qo(\rho), \qo(\sigma)\right)
         & =
        D\left( p \rho_0 + (1-p)\qo'(\rho), p \rho_0 + (1-p)\qo'(\sigma) \right)
        \\
         & \leq
        p D \left( \rho_0, \rho_0\right)
        +
        (1-p) D \left( \qo'(\rho), \qo'(\sigma)\right)
        \\
         & =
        (1-p) D \left( \qo'(\rho), \qo'(\sigma)\right)
        \\
         & \geq(1-p)D\left( \rho, \sigma\right)
        \\
         & < D \left( \rho, \sigma\right).
    \end{align*}
\end{ex}

\begin{ex}
    \label{ex9.12}
    \begin{align*}
        \rho = \frac{I + \bm{r} \cdot \bm{\sigma}}{2},
        \sigma = \frac{I + \bm{s} \cdot \bm{\sigma}}{2}
    \end{align*}
    として,
    \begin{align*}
        \qo(\rho)
        =
        \frac{I + (1-p)\bm{r}\cdot\bm{\sigma}}{2},
        \qo(\sigma)
        =
        \frac{I + (1-p)\bm{s}\cdot\bm{\sigma}}{2}
    \end{align*}
    なので,
    \begin{align*}
        D \left( \qo(\rho), \qo(\sigma)\right)
        = (1-p)\frac{\left| \bm{r}- \bm{s} \right|}{2}
        <\frac{\left| \bm{r}- \bm{s} \right|}{2}
        = D \left( \rho, \sigma\right)
    \end{align*}
\end{ex}

\begin{ex}
    \label{ex9.13}
    ビット反転チャンネルの量子演算は,
    \begin{align*}
        \rho = \frac{I + \bm{r} \cdot \bm{\sigma}}{2},
        \sigma = \frac{I + \bm{s} \cdot \bm{\sigma}}{2}
    \end{align*}
    として,
    \begin{align*}
        \qo \left( \rho\right)
         & =
        \frac{I + (1-p) X \rho X}{2}
        =
        \frac{I + r_1 X + (2p-1)r_2Y + (2p-1)r_3Z}{2} \\
        \qo \left( \sigma\right)
         & =
        \frac{I + s_1 X + (2p-1)s_2Y + (2p-1)s_3Z}{2}
    \end{align*}
    なので,
    \begin{align*}
        D \left( \qo(\rho), \qo(\sigma)\right)=
        \sqrt{(r_1 - s_1)^2 + (2p-1)^2(r_2 - s_2)^2 + (2p-1)^2(r_3 - s_3)^2}.
    \end{align*}
    よって,
    \begin{align*}
        D \left( \qo(\rho), \qo(\sigma)\right)
        \leq
        D \left(\rho, \sigma\right).
    \end{align*}
    等号は, 実数$t$として,
    \begin{align*}
        \bm{r} = \bm{s} = t
        \begin{pmatrix}
            1 \\ 0\\ 0
        \end{pmatrix}
    \end{align*}
    の時に成立する. 不動点の集合は, 実数$t$として,
    \begin{align*}
        \bm{r} = t
        \begin{pmatrix}
            1 \\ 0\\ 0
        \end{pmatrix}.
    \end{align*}
\end{ex}

\begin{ex}
    \label{ex9.14}
    正値オペレータ$A$はスペクトル分解可能で,
    \begin{align*}
        A = \sum_i \lambda_i \ket{i}\bra{i}
    \end{align*}
    とかけ, ユニタリ$U$を用いて
    \begin{align*}
        \sqrt{UAU^\dagger}
        =
        \sqrt{\sum_i \lambda_i U\ket{i}\bra{i}U^\dagger}
        =
        \sum_i\sqrt{\lambda_i} U\ket{i}\bra{i}U^\dagger
        =
        U \sqrt{A} U^\dagger
    \end{align*}
    が成立する. このことを用いて,
    \begin{align*}
        F \left( U \rho U^\dagger, U \sigma U^\dagger\right)
         & =
        \tr \sqrt{ \sqrt{U \rho U^\dagger} U \sigma U^\dagger \sqrt{U \rho U^\dagger}}
        \\
         & =
        \tr \sqrt{ U \sqrt{\rho} U^\dagger U \sigma U^\dagger U \sqrt{\rho} U^\dagger}
        \\
         & =
        \tr \sqrt{ U \sqrt{\rho} \sigma \sqrt{\rho} U^\dagger}
        \\
         & =
        \tr U  \sqrt{\sqrt{\rho} \sigma \sqrt{\rho}} U^\dagger
        \\
         & =
        \tr \sqrt{\sqrt{\rho} \sigma \sqrt{\rho}}
        \\
         & =
        F \left(\rho, \sigma\right)
    \end{align*}
\end{ex}

\begin{ex}
    \label{ex9.15}
    定理9.4と同様.
\end{ex}

\begin{ex}
    \label{ex9.16}
    \begin{align*}
        \braket{m|A \otimes B|m}
         & =
        \sum_{i,j}
        \bra{i_R} \bra{i_Q}
        A \otimes B
        \ket{j_R} \ket{j_Q}
        \\
         & =
        \sum_{i,j}
        \bra{i_R}A\ket{j_R}
        \bra{i_Q}B\ket{j_Q}
        \\
         & =
        \sum_{i,j}
        \bra{j_R}A^\dagger\ket{i_R}^{*}
        \bra{i_Q}B\ket{j_Q}
        \\
         & =
        \tr \left[ A^\top B\right]
    \end{align*}
\end{ex}


\begin{ex}
    \label{ex9.17}
    $0 \leq F\left( \rho, \sigma\right) \leq 1$なので,
    \begin{align*}
        0 \leq A(\rho, \sigma) =  \arccos F\left( \rho, \sigma\right) \leq \frac{\pi}{2}
    \end{align*}
    で, 左の等号の成立は,
    \begin{align*}
        F\left( \rho, \sigma\right) = 1
        \leftrightarrow
        \rho = \sigma
    \end{align*}
\end{ex}

\begin{ex}
    \label{ex9.18}
    定理9.6と$arccos$の単調性より示せる.
\end{ex}

\begin{ex}
    \label{ex9.19}
    定理9.7より,
    \begin{align*}
        F \left( \sum_i p_i \rho_i , \sum_i p_i \sigma_i \right)
        \geq
        \sum_i p_i F(\rho_i, \sigma_i)
    \end{align*}
\end{ex}

\begin{ex}
    \label{ex9.20}
    演習\ref{ex9.19}で, 全ての$i$に対して,
    \begin{align*}
        \sigma_i = \sigma
    \end{align*}
    とすれば,
    \begin{align*}
        F \left( \sum_i p_i \rho_i , \sigma \right)
        \geq
        \sum_i p_i F(\rho_i, \sigma).
    \end{align*}
\end{ex}

\begin{ex}
    \label{ex9.21}
    POVMとして,
    \begin{align*}
        \{E_1 = \ket{\psi}\bra{\psi}, E_2 = I - \ket{\psi}\bra{\psi}\}
    \end{align*}
    を考える. すると, 定理9.1より,
    \begin{align*}
        D \left( \ket{\psi}, \sigma\right)
         & \geq
        \frac{1}{2}
        \left|
        \tr\left( \ket{\psi}\bra{\psi}E_1\right)
        -
        \tr\left( \sigma E_1\right)
        \right|
        +
        \frac{1}{2}
        \left|
        \tr\left( \ket{\psi}\bra{\psi}E_2\right)
        -
        \tr\left( \sigma E_2\right)
        \right|
        \\
         & =
        \frac{1}{2}
        \left(
        1
        -
        \tr\left( \sigma E_1\right)
        +
        \tr\left( \sigma E_2\right)
        \right)
        =
        1 - \tr\left( \sigma E_1\right)
        =
        1 - \braket{\psi|\sigma|\psi}
        =
        1 - F\left( \ket{\psi}, \sigma\right)^2.
    \end{align*}
\end{ex}

\begin{ex}
    \label{ex9.22}
    誤差$E$の定義より, 
    \begin{align*}
        E\left( VU, \qof \circ \qo\right) 
        = d \left( VU \rho U U^\dagger V^\dagger, \qof \circ \qo \left( \rho \right)\right)
    \end{align*}
    を満たす$\rho$が存在する. したがって, 
    \begin{align*}
        E\left( VU, \qof \circ \qo\right) 
        &\le
        d \left( VU \rho U^\dagger V^\dagger,\  \qof \circ \qo \left( \rho \right)\right)
        \\
        &\le 
        d \left( VU \rho U^\dagger V^\dagger,\  V \qo\left(\rho\right) V^\dagger \right)
        +
        d \left(V \qo\left(\rho\right) V^\dagger,\  \qof \circ \qo \left(\rho\right) \right)
        \\
        &=
        d \left( U \rho U^\dagger,\ \qo\left(\rho\right) \right)
        +
        d \left(V \qo\left(\rho\right) V^\dagger,\  \qof \left(\qo \left(\rho\right)\right) \right)
        \\
        &\le
        E\left(U, \qo\right) + E\left(V, \qof\right)
    \end{align*}
    を得る.
\end{ex}

\begin{ex}
    $0 \leq F \leq 1 $に注意して,
    \begin{align*}
        \bar{F} = 1 
        &\longleftrightarrow
        \sum_j p_j \left( 1 - F \left( \rho_i, \qo\left( \rho_j\right)\right)^2\right) = 0\\
        &\longleftrightarrow
        1 - F \left( \rho_i, \qo\left( \rho_j\right)\right)^2 = 0 \\
        &\longleftrightarrow
        F \left( \rho_j, \qo \left( \rho_j \right)\right) = 1 \\
        &\longleftrightarrow
        \rho_j =  \qo \left( \rho_j \right).
    \end{align*} 
\end{ex}
