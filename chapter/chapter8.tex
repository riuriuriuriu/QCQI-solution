\chapter{量子雑音と量子情報}

\begin{ex}
    \label{ex8.1}
    \begin{align*}
        \qo(\rho)
        =
        U \ket{\psi} \bra{\psi} U^\dagger
        =
        U \rho U^\dagger
    \end{align*}
\end{ex}

\begin{ex}
    \label{ex8.2}
    密度行列は, スペクトル分解可能;
    \begin{align*}
        \rho = \sum_i p_i \ket{i} \bra{i}.
    \end{align*}
    測定オペレータの集合$\{ M_m\}$に対し, $\rho$の測定結果が$m$である確率は,
    \begin{align*}
        p(m) =
        \sum_i p(m|i) p_i
        =
        \sum_i \braket{i | M_m^\dagger M_m | i} p_i
        =
        \tr \left[ \sum_i \braket{i | M_m^\dagger M_m | i} p_i \right]
        =
        \tr \left[ M_m^\dagger M_m \rho\right]
        =
        \tr \left[M_m \rho  M_m^\dagger \right].
    \end{align*}
    測定後の系の状態$\rho_m$は,
    \begin{align*}
        \left\{ \ket{i^m} =
        \frac{M_m \ket{i}}{\sqrt{\braket{i|M_m^\dagger M_m | i}}},
        p(i|m)
        \right\}
    \end{align*}
    というアンサンブルで,
    \begin{align*}
        \rho_m
        =
        \sum_i p(i|m) \ket{i^m} \bra{i^m}
        =
        \sum_i
        \frac{p(m|i) p_i}{p(m)}
        \frac{M_m \ket{i} \bra{i} M_m^\dagger}
        {\braket{i|M_m^\dagger M_m | i}}
        =
        \sum_i p(i|m) \ket{i^m} \bra{i^m}
        =
        \sum_i
        \frac{p_i M_m \ket{i} \bra{i} M_m^\dagger}{p(m)}
        =
        \frac{M_m \rho M_M^\dagger}
        {\tr\left[ M_m \rho M_m^\dagger\right]}
    \end{align*}
\end{ex}

\begin{ex}
    \label{ex8.3}
    ABの初期状態$\rho_A \otimes \rho_B = \rho =
        \sum_{a,b} p_{ab} \ket{a}\bra{a} \otimes \ket{b}\bra{b}$,
    CDの初期状態$\ket{0_C}\bra{0_C} \otimes \ket{0_D}\bra{0_D}$
    として,
    \begin{align*}
        \qo\left(\rho\right)
         & =
        \tr_{A \otimes D}
        \left[
            U \rho \otimes \ket{0_C}\bra{0_C} \otimes \ket{0_D}\bra{0_D} U^\dagger
            \right]
        \\
         & =
        \sum_{a,a', b, d'}
        p_{ab}
        \bra{a'} \otimes \bra{d'}
        \left(
        U \ket{a} \bra{a} \otimes \ket{b}\bra{b} \otimes \ket{0}\bra{0} \otimes \ket{0}\bra{0} U^\dagger
        \right)
        \ket{a'} \otimes \ket{d'}
        \\
         & =
        \sum_{a,a', b, d'}
        p_{ab}
        \bra{a'} \otimes \bra{d'}
        U
        \ket{0_C} \otimes \ket{0_D}
        \ket{a}\bra{a} \otimes \ket{b}\bra{b}
        \bra{0_C} \otimes \bra{0_D}
        U^\dagger
        \ket{a'} \otimes \ket{d'}
        \\
         & =
        \sum_{a',d'}
        \bra{a'} \otimes \bra{d'}
        U
        \ket{0_C} \otimes \ket{0_D}
        \rho
        \bra{0_C} \otimes \bra{0_D}
        U^\dagger
        \ket{a'} \otimes \ket{d'} =
        \sum_{a,d}
        E_{ad}
        \rho
        E_{ad}^\dagger.
    \end{align*}
    ただし,
    \begin{align*}
        E_{ad} = \bra{a'} \otimes \bra{d'}
        U
        \ket{0_C} \otimes \ket{0_D}
    \end{align*}
    で, ADでの完全性条件より,
    \begin{align*}
        \sum_{a,d}
        E_{ad}
        E_{ad}^\dagger
         & =
        \sum_{a,d}
        \bra{a'} \otimes \bra{d'}
        U
        \ket{0_C} \otimes \ket{0_D}
        \bra{0_C} \otimes \bra{0_D}
        U^\dagger
        \ket{a'} \otimes \ket{d'}
        \\
         & =
        \sum_{a,d}
        \bra{0_C} \otimes \bra{0_D}
        U^\dagger
        \ket{a'} \otimes \ket{d'}
        \bra{a'} \otimes \bra{d'}
        U
        \ket{0_C} \otimes \ket{0_D}
        \\
         & =
        \bra{0_C} \otimes \bra{0_D}
        U^\dagger
        U
        \ket{0_C} \otimes \ket{0_D} = I_{AB}.
    \end{align*}
\end{ex}

\begin{ex}
    \label{ex8.4}
    環境の初期状態$\ket{0}$として,
    \begin{align*}
        E_k
        = \left(I \otimes \bra{k}\right)   U  \left(I \otimes \ket{0}\right)
        = 
        \left(I \otimes \bra{k}\right)  
        \left( P_0 \otimes I + P_1 \otimes X  \right)
        \left(I \otimes \ket{0}\right)
        = P_0 \braket{k|0} + P_1 \braket{k|1}.
    \end{align*}
    なので, オペレータ和表現は, 
    \begin{align*}
        \qo (\rho)
        = \sum_{k = 0, 1} E_k \rho E_k^\dagger
        = P_0 \rho P_0 + P_1 \rho P_1.
    \end{align*}
\end{ex}

\begin{ex}
    \label{ex8.5}
    環境の初期状態$\ket{0}$として,
    \begin{align*}
        E_k
        = \braket{k | U | 0}
        = \frac{X}{\sqrt{2}} \braket{k|0} + \frac{Y}{\sqrt{2}} \braket{k|1}.
    \end{align*}
    なので, 量子演算は,
    \begin{align*}
        \qo (\rho)
        = \sum_{k = 0, 1} E_k \rho E_k^\dagger
        = \frac{X \rho X + Y \rho Y}{2}.
    \end{align*}
\end{ex}

\begin{ex}
    \label{ex8.6}
    \par
    CP写像$\qo \colon \linearopset{\mathbb{C}^m} \to \linearopset{\mathbb{C}^n}$の演算要素$\left\{E_k \colon \mathbb{C}^m \to \mathbb{C}^n \right\}_{k=1}^{l_E \le nm}$, CP写像$\qof \colon \linearopset{\mathbb{C}^n} \to \linearopset{\mathbb{C}^m}$演算要素$\left\{F_k \colon \mathbb{C}^n \to \mathbb{C}^m \right\}_{k=1}^{l_F \le nm}$とする. このとき, 合成写像$\qo \circ \qof \colon \linearopset{\mathbb{C}^n} \to \linearopset{\mathbb{C}^n}$は, $A \in \linearopset{\mathbb{C}^n}$に対して,
    \begin{align*}
        \qo \circ \qof \left( A\right)
        =
        \sum_{i = 1}^{l_E}\sum_{j=1}^{l_F}
        E_i F_j A F_j^\dagger E_i^\dagger
    \end{align*}
    でかける. 演習\ref{ex8.10}より, $\qo \circ \qof \colon \linearopset{\mathbb{C}^n} \to \linearopset{\mathbb{C}^n}$の演算要素$\left\{G_i \colon \mathbb{C}^n \to \mathbb{C}^n \right\}_{i=1}^{l_G \le nm}$を作ることができる. したがって, $\qo \circ \qof \colon \linearopset{\mathbb{C}^n} \to \linearopset{\mathbb{C}^n}$もCP写像.
\end{ex}

\begin{ex}
    \label{ex8.7}
    \begin{align*}
        \qo_m \left( \rho \right)
        =
        \tr_E \left( M_m U \rho \otimes \sigma U^\dagger M_m^\dagger \right)
    \end{align*}
    として, Qの終状態は,
    \begin{align*}
        \frac{\qo_m \left( \rho \right)}{\tr \left[ \qo_m(\rho)\right]}
    \end{align*}
    で,
    \begin{align*}
        \sigma = \sum_j q_j \ket{j}\bra{j}
    \end{align*}
    ならば,
    \begin{align*}
        \qo_m \left( \rho \right)
        =
        \sum_{jk} q_j \bra{e_k} M_m U \rho \otimes \ket{j}\bra{j} U^\dagger M_m^\dagger \ket{e_k}
        =
        \sum_{j,k}E_{jk} \rho E_{jk}^\dagger.
    \end{align*}
    ここで,
    \begin{align*}
        E_{jk} = \sqrt{q_j} \braket{e_k | M_m U | j}.
    \end{align*}
\end{ex}

\begin{ex}
    \label{ex8.8}
\end{ex}

\begin{ex}
    \label{ex8.9}
    \begin{align*}
        \qo_m \left( \rho \right)
         & = \tr_{env} \left[ P_m U \left(\rho \otimes \ket{e_0} \bra{e_0} \right) U^\dagger P_m\right] \\
         & =
        \sum_{m',k'}
        \braket{
            m', k'
            |P_m U \left(\rho \otimes \ket{e_0} \bra{e_0} \right) U^\dagger P_m|
            m', k'
        }
        \\
         & =
        \sum_{m',k',l,l'}
        \braket{m', k'|m,l}
        \braket{
            m,l
            |P_m U \left(\rho \otimes \ket{e_0} \bra{e_0} \right) U^\dagger P_m|
            m', l'
        }
        \braket{m', l'|m',k'}
        \\
         & =
        \sum_{k}
        \braket{
            m,k
            |U \left(\rho \otimes \ket{e_0} \bra{e_0} \right) U^\dagger |
            m, k
        }
        \\
         & =
        \sum_{i,k}
        p_i
        \braket{
            m,k
            |U \ket{i}\bra{i} \otimes \ket{e_0}\bra{e_0} U^\dagger |
            m, k
        }
        \\
         & =
        \sum_{i,k, m', k', m'', k''}
        p_i
        \braket{
            m,k
            |
            \Big( E_{m'k'} \ket{i} \otimes \ket{m',k'} \Big)
            \Big( \bra{i} E_{m''k''}^\dagger \otimes \bra{m'',k''} \Big)
            |
            m, k
        }
        \\
         & =
        \sum_{i,k}
        p_i
        E_{mi} \ket{i}
        \bra{i} E_{mi}^\dagger
        =
        \sum_{k}
        E_{mk} \rho E_{mk}^\dagger
    \end{align*}
    なので, 測定後の状態は,
    \begin{align*}
        \frac{\qo_m \left(\rho\right)}
        {\tr\left[ \qo_m\left(\rho\right)\right]}.
    \end{align*}
\end{ex}

\begin{ex}
    \label{ex8.10}
    $d$次元ヒルベルト空間$\hilbertsp$とし, CP写像$\qo \colon \linearopset{\hilbertsp} \to \linearopset{\hilbertsp}$が有限の$M$個の演算要素$\left\{ E_i \colon \hilbertsp \to \hilbertsp \right\}_{i = 1}^{M}$を用いて,
    \begin{align*}
        \qo \left( A\right) = \sum_{i=1}^M E_i A E_i^\dagger
    \end{align*}
    で定義されてるとする.
    \par
    $d \times d$行列の集合$\{E_i\}_{i=1}^{M}$から, できるだけ多くの線形独立な$r \left(\le d^2\right)$個の演算要素を取り出して, 適当に添字を付け替えることで, 線形独立な演算要素の集合$\{E_i\}_{i=1}^{r}$を作ることができる.
    よって, 任意の$i= 1, \dots , M$に対して, ある複素数列$\left\{ c_{ij} \right\}_{j=1}^{r}$を用いて,
    \begin{align*}
        E_i = \sum_{j=1}^{r} c_{ij} E_j
    \end{align*}
    と表せる.
    したがって, 任意の$i, j = r+1 , \dots, M$に対して,
    \begin{align*}
        w_{ij} = \tr\left[E_i^\dagger E_j\right] = \sum_{k=1}^{r} c_{jk} w_{ik}
    \end{align*}
    が成立するので, $\rank W \leq r \leq d^2$.
    また, 任意の$i, j = 1 , \dots, M$に対して,
    \begin{align*}
        w_{jk} = \tr\left[ E_j^\dagger E_k\right]= \tr\left[ E_k^\dagger E_j\right]^*
        = w_{kj}^*
    \end{align*}
    が成立するので, 行列$W = \left(w_{ij}\right)_{i,j = 1, \dots , M}$はHermite.  したがって, $W$はユニタリ行列$U = \left(u_{ij}\right)_{i,j = 1, \dots , M}$で対角化可能で, 対角行列$u^\dagger W u$の対角成分の非ゼロ要素は$\mathrm{rank}W$個である.
    \par
    ここで, $\{F_i \colon \hilbertsp \to \hilbertsp \}_{i=1}^{M}$を
    \begin{align*}
        F_i = \sum_{i=1}^{M} u_{il}^{*} E_l
    \end{align*}
    で定義する. この定義は
    $E_{i}$の行列表示$\left( e^{i}_{\alpha \beta} \right)_{\alpha, \beta=1,\dots,d}$と
    $F_{i}$の行列表示$\left( f^{i}_{\alpha \beta} \right)_{\alpha, \beta=1,\dots,d}$に対して,
    \begin{align*}
        f_{\alpha \beta}^{i} = \sum_{l = 1}^M u_{il}^* e^l_{\alpha \beta}
    \end{align*}
    とも表せる.
    すると, $W$の非ゼロ固有値$\lambda_1, \dots , \lambda_{\rank W}$として,
    \begin{align*}
        \sum_{\alpha,\beta=1}^d
        \left| f_{\alpha \beta}^{i} \right|^2
        =
        \sum_{i,j,k=1}^M
        \sum_{\alpha, \beta = 1}^d
        u_{i j} e_{\alpha \beta}^{j*} e_{\alpha \beta}^k u_{ik}^*
        =
        \sum_{i,j,k=1}^M
        u_{ij} \tr\left[ E_j^\dagger E_k\right] u^\dagger_{ki}
        =
        \sum_{i,j,k=1}^M u_{ij} w_{jk} u^\dagger_{ki}
        =
        \begin{cases}
            \lambda_i & ( i = 1, \dots, \mathrm{rank} W)      \\
            0         & ( i = \mathrm{rank} W + 1, \dots , M)
        \end{cases}
    \end{align*}
    であるから, 任意の$i = \mathrm{rank} W + 1, \dots , M $に対して, $\qof_i = 0$.
    よって,
    \begin{align*}
        \qo \left( A \right)
        =
        \sum_{i=1}^M 
        E_i A E_i^{\dagger}
        =
        \sum_{i,j,k=1}^M
        u_{ki}F_k A F_j^{\dagger} u_{ji}^*
        =
        \sum_{k=1}^M
        F_k A F_j^{\dagger}
        =
        \sum_{k=1}^{\rank W}
        F_k A F_k^\dagger
    \end{align*}
    を得る.
    ゆえに, CP写像$\qo \colon \linearopset{\hilbertsp} \to \linearopset{\hilbertsp}$の演算要素としては$\{F_i \colon \hilbertsp \to \hilbertsp \}_{i=1}^{\rank W}$をとることができる.
\end{ex}

\begin{ex}
    \label{ex8.11}
    \cite{Ishizaka2012}の定理4.6.1.
\end{ex}

\begin{ex}
    \label{ex8.12}
    実行列$M$の極分解
    \begin{align*}
        M = U |M|
    \end{align*}
    に対して, $U$は直交行列なので行列式が$\pm1$に限られる. そこで,
    \begin{align*}
        \det U = 1  & \rightarrow O = U, \ S = |M|   \\
        \det U = -1 & \rightarrow O = -U, \ S = -|M|
    \end{align*}
    とすることで, 行列式が1であるように$O$を取ることができて,
    \begin{align*}
        M = OS
    \end{align*}
    と書ける.
\end{ex}

\begin{ex}
    \label{ex8.13}
    演習\ref{ex4.8}より明らか.
\end{ex}

\begin{ex}
    \label{ex8.14}
    演習\ref{ex8.12}で見たように, $\det S$は正とは限らない.
\end{ex}

\begin{ex}
    \label{ex8.15}
    \begin{align*}
        X \ket{\pm} & = \pm \ket{\pm}  \\
        Y \ket{\pm} & = \mp i\ket{\mp} \\
        Z \ket{\pm} & = \ket{\mp}
    \end{align*}
    なので,
    \begin{align*}
        \rho = \frac{I + \bm{r} \cdot \bm{\sigma}}{2}
    \end{align*}
    に対して,
    \begin{align*}
        \braket{\pm | \rho | \pm}
        =
        \frac{I + \braket{ \pm| \bm{r} \cdot \bm{\sigma} |\pm}}{2}
        =
        \frac{I \pm r_x}{2}.
    \end{align*}
    したがって,
    \begin{align*}
        \qo\left( \rho \right)
        =
        \sum_{i = \pm} \ket{i}\bra{i} \rho \ket{i}\bra{i}
        =
        \frac{1 + r_x}{2} \ket{+}\bra{+} + \frac{1 - r_x}{2} \ket{-}\bra{-}
        =
        \frac{I + r_x X}{2}
    \end{align*}
    なので, この量子演算の作用をBloch球面上で表すと,
    \begin{align*}
        \bm{r} =
        \begin{pmatrix}
            r_x \\
            r_y \\
            r_z
        \end{pmatrix}
        \longrightarrow
        \bm{r}' =
        \begin{pmatrix}
            r_x \\
            0   \\
            0
        \end{pmatrix}.
    \end{align*}
\end{ex}

\begin{ex}
    \label{ex8.16}
    $p\neq 0$として,
    \begin{align*}
        E_0 =
        \sqrt{1-p}
        \begin{pmatrix}
            0 & 1 \\
            1 & 0
        \end{pmatrix}
    \end{align*}
    とする. 演算集合$\{ E_0 \}$で定義される量子演算はトレース非保存で,
    \begin{align*}
        \qo \left( \rho\right)
        =
        E_0 \rho E_0^\dagger
        =
        (1-p)X \rho X
        =
        \frac{1-p}{2} I
        +
        \frac{1-p}{2}
        \left( r_x X - r_y Y - r_z Z\right)
    \end{align*}
    となる. これはBloch球表示できない.
\end{ex}

\begin{ex}
    \label{ex8.17}
    \begin{align*}
        \qo \left( A \right)
        =
        \frac{A + XAX + YAY + ZAZ}{4}
    \end{align*}
    として,
    \begin{align*}
        \qo(I) = I,\ \qo(X) = \qo(Y) = \qo(Z)
    \end{align*}
    なので,
    \begin{align*}
        \rho = \frac{I + \bm{r} \cdot \bm{\sigma}}{2}
    \end{align*}
    に対して,
    \begin{align*}
        \qo ( \rho) = \frac{1}{2}\qo ( I ) = \frac{I}{2}.
    \end{align*}
\end{ex}

\begin{ex}
    \label{ex8.18}
    $\rho \in \densityopset{\complex{d}}$に対して, $\tr\left[ \rho^i\right] \ge 0 \left( i=1,2,\cdots\right)$が成立. また, 分極解消チャンネルのパラメータは$0 \le p \le \frac{d^2}{d^2-1}$であることに注意して, 
    \begin{align*}
        \tr\left[ \qo \left( \rho^k\right)\right]
        =
        \tr\left[ 
            \left(
                \frac{p}{d} \mathbf{1} + \left(1-p\right)\rho 
            \right)^k
            \right]
        =
        \sum_{i=0}^k
        \comb{k}{i}
        \left( \frac{p}{d}\right)^{k-i}
        \left(1-p\right)^i
        \tr\left[\rho^i\right]
        \le
        \left(1-p\right)^k \tr\left[\rho^k\right]
        \le
        \tr\left[ \rho^k\right]
    \end{align*}
    を得る. 
\end{ex}
\begin{ex}
    \label{ex8.19}
    $d = 2^n$とする. $\linearopset{\mathbb{C}^d}$の基底として$\left\{M_{\bm{\alpha}}\right\}_{\bm{\alpha}\in\left\{0,1,2,3\right\}^n}$をとる. ここで, $M_{\bm{\alpha}}$は,
    \begin{align*}
        M_{\bm{\alpha} = \left( \alpha_1, \cdots, \alpha_n \right)} = \bigotimes_{i=1}^n \sigma_{\alpha_i},\ 
        \sigma_0 = I, \ \sigma_1 = X, \ \sigma_2 = Y,\ \sigma_3 = Z  
    \end{align*}
    とした. 
    \par
    このとき, 
    \begin{align*}
        \sum_{\bm{\alpha} \in \left\{0,1,2,3\right\}^n}
        M_{\bm{\alpha}} M_{\bm{\beta}} M_{\bm{\alpha}}^\dagger
        =
        \begin{cases}
            d^2 \mathbf{1} & \left(\bm{\beta} = \bm{0}\right) \\
            0 & \left( \text{otherwise}\right)
        \end{cases}
        \tag{$\star$}
    \end{align*}
    が成り立つ. これを示す. $\bm{\beta} = \bm{0}$のときは明らか. $\bm{\beta} \neq \bm{0}$とすると, $\beta_i \neq 0 $なる$i$が存在し, 
    \begin{align*}
        \sum_{\alpha_i = 0,1,2,3}
        \sigma_{\alpha_i} \sigma_{\beta_i}  \sigma_{\alpha_i}^\dagger
        =
        0
    \end{align*}
    であるから,
    \begin{align*}
        \sum_{\bm{\alpha} \in \left\{0,1,2,3\right\}^n}
        M_{\bm{\alpha}} M_{\bm{\beta}} M_{\bm{\alpha}}^\dagger
        =
        \left(
            \sum_{\alpha_1 = 0,1,2,3}
            \sigma_{\alpha_1} \sigma_{\beta_1}  \sigma_{\alpha_1}^\dagger
        \right)
        \otimes
        \dots
        \otimes
        \left(
            \sum_{\alpha_i = 0,1,2,3}
            \sigma_{\alpha_i} \sigma_{\beta_i}  \sigma_{\alpha_i}^\dagger
        \right)
        \otimes
        \dots
        \otimes
        \left(
            \sum_{\alpha_n = 0,1,2,3}
            \sigma_{\alpha_n} \sigma_{\beta_n}  \sigma_{\alpha_n}^\dagger
        \right)
        =
        0
    \end{align*}
    が成り立つ.
    \par
    $A \in \linearopset{\mathbb{C}^d}$は, $\left\{c_{\bm{\alpha}}\right\}_{\bm{\alpha} \in \left\{0,1,2,3\right\}^n}$を用いて,
    \begin{align*}
        A = \sum_{\bm{\alpha} \in \left\{0,1,2,3\right\}^n} c_{\bm{\alpha}} M_{\bm{\alpha}}
    \end{align*}
    と展開できる. すると, 
    \begin{align*}
        \tr\left[A\right] = \sum_{\bm{\alpha} \in \left\{0,1,2,3\right\}^n} c_{\bm{\alpha}} \tr\left[M_{\bm{\alpha}}\right]
        =
        c_{\bm{0}}d
    \end{align*}
    が成り立つ. 一方, $(\star)$より, 
    \begin{align*}
        \sum_{\bm{\alpha} \in \left\{0,1,2,3\right\}^n} 
        M_{\bm{\alpha}} A M_{\bm{\alpha}}^\dagger
        =
        \sum_{\bm{\alpha} \in \left\{0,1,2,3\right\}^n} 
        \sum_{\bm{\beta} \in \left\{0,1,2,3\right\}^n}
        c_{\bm{\beta}} 
        M_{\bm{\alpha}} M_{\bm{\beta}} M_{\bm{\alpha}}^\dagger
        =
        c_{\bm{0}}d^2 \mathbf{1}
    \end{align*}
    が成り立つ. ゆえに, 
    \begin{align*}
        \sum_{\bm{\alpha} \in \left\{0,1,2,3\right\}^n} 
        M_{\bm{\alpha}} A M_{\bm{\alpha}}^\dagger
        =
        d \tr\left[A\right]\mathbf{1}
    \end{align*}
    が成り立つ. したがって, CPTP写像
    \begin{align*}
        \qo \colon \linearopset{\mathbb{C}^{d}} \ni A \mapsto \frac{p \tr\left[ A\right]}{d}\mathbf{1} + \left(1-p\right)A \in \linearopset{\mathbb{C}^{d}}
    \end{align*}
    に対して, 
    \begin{align*}
        \qo \left(A\right)
        &=
        \frac{p \tr\left[ A\right]}{d}\mathbf{1} + \left(1-p\right)A
        \\
        &=
        \frac{p}{d^2}
        \sum_{\bm{\alpha} \in \left\{0,1,2,3\right\}^n} 
        M_{\bm{\alpha}} A M_{\bm{\alpha}}^\dagger
        +
        (1-p)
        M_{\bm{0}} A M_{\bm{0}}^\dagger
        \\
        &=
        \left( 1 - p + \frac{p}{d^2}\right)
        M_{\bm{0}} A M_{\bm{0}}^\dagger
        +
        \frac{p}{d^2}
        \sum_{\bm{\alpha} \notin \bm{0}}
        M_{\bm{\alpha}} A M_{\bm{\alpha}}^\dagger
    \end{align*}
    が成り立つ. したがって, 分極解消チャンネルの演算要素は, 
    \begin{align*}
        \left\{ 
            \sqrt{1-p+\frac{p}{d^2}} M_{\bm{0}}
        \right\}
        \cup
        \left\{
            \frac{\sqrt{p}}{d} M_{\bm{\alpha}}
        \right\}_{\bm{\alpha}\neq\bm{0}}
    \end{align*}
    となる.
\end{ex}

\begin{ex}
    \label{ex8.20}
    問題の図の回路の作用$U$として,
    \begin{align*}
        U \equiv C_1 \left[ X \right] C_2\left[ R_y(\theta)\right]
        =
        \begin{pmatrix}
            1 & 0 & 0                     & 0                       \\
            0 & 0 & \sin \frac{\theta}{2} & \cos \frac{\theta}{2}   \\
            0 & 0 & \cos \frac{\theta}{2} & - \sin \frac{\theta}{2} \\
            0 & 1 & 0                     & 0
        \end{pmatrix}.
    \end{align*}
    第1qubitの初期状態$\rho_\mathrm{in}$を
    \begin{align*}
        \rho_\mathrm{in}
        =
        \begin{pmatrix}
            a   & b \\
            b^* & c
        \end{pmatrix}
    \end{align*}
    として, 初期状態は,
    \begin{align*}
        \rho_\mathrm{in} \otimes \ket{0}\bra{0}
        =
        \begin{pmatrix}
            a   & 0 & b & 0 \\
            0   & 0 & 0 & 0 \\
            b^* & 0 & c & 0 \\
            0   & 0 & 0 & 0
        \end{pmatrix}.
    \end{align*}
    ゆえに, 終状態は,
    \begin{align*}
        U  \rho_\mathrm{in} \otimes \ket{0}\bra{0} U^\dagger
        =
        \begin{pmatrix}
            a                         & b \sin\frac{\theta}{2}                        & b \cos \frac{\theta}{2}                       & 0 \\
            b^* \sin \frac{\theta}{2} & c \sin^2 \frac{\theta}{2}                     & c \sin \frac{\theta}{2} \cos \frac{\theta}{2} & 0 \\
            b^* \cos \frac{\theta}{2} & c \sin \frac{\theta}{2} \cos \frac{\theta}{2} & c \cos^2 \frac{\theta}{2}                     & 0 \\
            0                         & 0                                             & 0                                             & 0
        \end{pmatrix}.
    \end{align*}
    したがって, 第1qubitの終状態$\rho_\mathrm{out}$は,
    \begin{align*}
        \rho_\mathrm{out}
        =
        \tr_\mathrm{2nd\ qubit} \left[U  \rho_\mathrm{in} \otimes \ket{0}\bra{0} U^\dagger\right]
        =
        \begin{pmatrix}
            a + c \sin^2 \frac{\theta}{2} & b \cos\frac{\theta}{2}   \\
            b^* \cos \frac{\theta}{2}     & c \cos^2\frac{\theta}{2}
        \end{pmatrix}.
    \end{align*}
    演習\ref{ex8.22}より,
    \begin{align*}
        \rho_\mathrm{out} = \qo_\mathrm{AD}\left( \rho_\mathrm{in} \otimes \ket{0} \bra{0}\right).
    \end{align*}
\end{ex}

\begin{ex}
    \label{ex8.21}
    (1)\ \par
    (2) \
    \begin{align*}
        \sum_{k=0}^{\infty} E_k^\dagger E_k
         & =
        \sum_{k=0}^{\infty}
        \sum_{l=k}^{\infty}
        \sum_{m=k}^{\infty}
        \sqrt{{}_l C_k}\sqrt{{}_m C_k}
        \sqrt{\left(1- \gamma\right)^{l-k}\gamma^k}
        \sqrt{\left(1- \gamma\right)^{m-k}\gamma^k}
        \ket{l}\braket{l-k|m-k}\bra{m}
        \\
         & =
        \sum_{k=0}^{\infty}
        \sum_{m=k}^{\infty}
        {}_m C_k
        \left(1- \gamma\right)^{l-k}\gamma^k
        \ket{m}\bra{m}
    \end{align*}
    なので, 全ての$\ket{n}$に対して,
    \begin{align*}
        \sum_{k=0}^{\infty} E_k^\dagger E_k \ket{n} =
        \sum_{k=0}^{\infty}
        \sum_{m=k}^{\infty}
        {}_m C_k
        \left(1- \gamma\right)^{l-k}\gamma^k
        \ket{m}\braket{m|n}
        =
        \sum_{k=0}^{\infty}
        {}_n C_k
        \left(1- \gamma\right)^{l-k}\gamma^k
        \ket{n}
        =
        \ket{n}
    \end{align*}
    だから,
    \begin{align*}
        \sum_{k=0}^{\infty} E_k^\dagger E_k = I.
    \end{align*}
\end{ex}

\begin{ex}
    \label{ex8.22}
    振幅ダンピングの演算要素;
    \begin{align*}
        E_0 =
        \begin{pmatrix}
            1 & 0                \\
            0 & \sqrt{1- \gamma}
        \end{pmatrix}
        ,
        E_1 =
        \begin{pmatrix}
            1 & \sqrt{\gamma} \\
            0 & 0
        \end{pmatrix}
    \end{align*}
    として,
    \begin{align*}
        \qo_\mathrm{AD} \left(\rho\right)
        =
        E_0 \rho E_0^\dagger + E_1 \rho E_1^\dagger
        =
        \begin{pmatrix}
            a + \left(1-a\right) \gamma & b \sqrt{1-\gamma}      \\
            b^*\sqrt{1- \gamma}         & c\left(1-\gamma\right)
        \end{pmatrix}
    \end{align*}
\end{ex}

\begin{ex}
    \label{ex8.23}
    振幅ダンピングの演算要素;
    \begin{align*}
        E_0 =
        \begin{pmatrix}
            1 & 0                \\
            0 & \sqrt{1- \gamma}
        \end{pmatrix}
        ,
        E_1 =
        \begin{pmatrix}
            1 & \sqrt{\gamma} \\
            0 & 0
        \end{pmatrix}
    \end{align*}
    とする.
    単一qubitの初期状態が2重軌道qubit;
    \begin{align*}
        \rho
        = \ket{\psi} \bra{\psi}
        =
        \left| a \right|^2 \ket{01} \bra{01}
        + a b^* \ket{01} \bra{10}
        + a^* b \ket{10} \bra{01}
        + \left| b \right|^2 \ket{10} \bra{10}
    \end{align*}
    でかけるとする. すると,
    \begin{align*}
        \left( \qo_\mathrm{AD} \otimes I \right)
        \left( \rho\right)
        =
        \left| a \right|^2 & \ket{0}\bra{0} \otimes \ket{1}\bra{1}
        + a^* b \sqrt{1 - \gamma} \ket{1}\bra{0} \otimes \ket{0}\bra{1}
        + ab^* \sqrt{1- \gamma} \ket{0}\bra{0} \otimes \ket{1}\bra{0}
        \\
                           & +  \left| b \right|^2 \left( 1 - \gamma \right)  \ket{1}\bra{1} \otimes \ket{0}\bra{0}
        + \gamma \left| b \right|^2 \ket{0}\bra{0} \otimes \ket{0} \bra{0}
    \end{align*}
    となり,
    \begin{align*}
        \left( \qo_\mathrm{AD} \otimes \qo_\mathrm{AD} \right)
        \left( \rho\right)
        =
        \left| a \right|^2 \left( 1- \gamma \right) & \ket{0}\bra{0} \otimes \ket{1}\bra{1}
        + a^* b\left( 1- \gamma\right) \ket{1}\bra{0} \otimes \ket{0}\bra{1}
        + ab^* \left( 1- \gamma\right) \ket{0}\bra{0} \otimes \ket{1}\bra{0}
        \\
                                                    & +  \left| b \right|^2 \left( 1 - \gamma \right)  \ket{1}\bra{1} \otimes \ket{0}\bra{0}
        + \gamma \left| b \right|^2 \ket{0}\bra{0} \otimes \ket{0} \bra{0}
        + \gamma \left| a \right|^2 \ket{0}\bra{0} \otimes \ket{0} \bra{0}.
    \end{align*}
    つまり,
    \begin{align*}
        \left( \qo_\mathrm{AD} \otimes \qo_\mathrm{AD} \right)
        \left( \rho\right)
        =
        \left( 1 - \gamma \right) \rho + \gamma \ket{0}\bra{0} \otimes \ket{0}\bra{0}
        =
        E_0^{\mathrm{dr}} \rho E_0^{\mathrm{dr}} + E_1^{\mathrm{dr}} \rho E_1^{\mathrm{dr}}.
    \end{align*}
\end{ex}

\begin{ex}
    \label{ex8.24}
    振幅ダンピングの演算要素
    \begin{align*}
        E_0 =
        \begin{pmatrix}
            1 & 0                \\
            0 & \sqrt{1- \gamma}
        \end{pmatrix}
        ,
        E_1 =
        \begin{pmatrix}
            1 & \sqrt{\gamma} \\
            0 & 0
        \end{pmatrix}
    \end{align*}
    とする.
    $\delta = 0$のJaynes-Cummings相互作用の時間発展$U$
    \begin{align*}
        U =
        \ket{0}\bra{0} \otimes \ket{0} \bra{0}
        +
        \cos gt \left(
        \ket{0}\bra{0} \otimes \ket{1} \bra{1} + \ket{1}\bra{1} \otimes \ket{0} \bra{0}
        \right)
        +
        -i \sin gt \left(
        \ket{0}\bra{1} \otimes \ket{1} \bra{0} + \ket{1}\bra{0} \otimes \ket{0} \bra{1}
        \right)
    \end{align*}
    で, 初期状態は $ \rho = \ket{ \mathrm{field}} \otimes \ket{\mathrm{atom}} = \ket{0} \otimes \ket{1}$
    とすると,
    \begin{align*}
        \qo \left( \rho\right)
         & =
        \tr_\mathrm{field} \left[ U \rho U^\dagger \right]   \\
         & =
        \tr_\mathrm{field} \left[
            \cos^2 gt  \ket{0}\bra{0} \otimes \ket{1} \bra{1}
            +
            \sin^2 gt  \ket{1}\bra{1} \otimes \ket{0} \bra{0}
        \right]                                              \\
         & =
        \cos^2 gt \ket{1}\bra{1} + \sin^2 gt \ket{0} \bra{0} \\
         & =E_0 \rho E_0^\dagger + E_1 \rho E_1^\dagger.
    \end{align*}
    最後の等式では, $\gamma = \sin^2 gt$とした.
\end{ex}

\begin{ex}
    \label{ex8.25}
    \begin{align*}
        p = \frac{e^{- \beta E_0}}{e^{- \beta E_0} + e^{- \beta E_1}}
    \end{align*}
    を$\beta$について解いて,
    \begin{align*}
        \beta = \frac{\log\frac{p}{1-p}}{E_1 - E_0}
    \end{align*}
\end{ex}

\begin{ex}
    \label{ex8.26}
    問題の図の回路の作用$U$として,
    \begin{align*}
        U \equiv C_1\left[ R_y(\theta)\right]
        =
        \begin{pmatrix}
            1 & 0 & 0                     & 0                       \\
            0 & 1 & 0                     & 0                       \\
            0 & 0 & \cos \frac{\theta}{2} & - \sin \frac{\theta}{2} \\
            0 & 0 & \sin \frac{\theta}{2} & \cos \frac{\theta}{2}
        \end{pmatrix}.
    \end{align*}
    第1qubitの初期状態$\rho_\mathrm{in}$を
    \begin{align*}
        \rho_\mathrm{in}
        =
        \begin{pmatrix}
            a   & b \\
            b^* & c
        \end{pmatrix}
    \end{align*}
    として, 初期状態は,
    \begin{align*}
        \rho_\mathrm{in} \otimes \ket{0}\bra{0}
        =
        \begin{pmatrix}
            a   & 0 & b & 0 \\
            0   & 0 & 0 & 0 \\
            b^* & 0 & c & 0 \\
            0   & 0 & 0 & 0
        \end{pmatrix}.
    \end{align*}
    ゆえに, 終状態は,
    \begin{align*}
        U  \rho_\mathrm{in} \otimes \ket{0}\bra{0} U^\dagger
        =
        \begin{pmatrix}
            a                         & 0 & b \cos\frac{\theta}{2}                      & b \sin \frac{\theta}{2}                     \\
            0                         & 0 & 0                                           & 0                                           \\
            b^* \cos \frac{\theta}{2} & 0 & c \cos^2 \frac{\theta}{2}                   & c \sin\frac{\theta}{2}\cos \frac{\theta}{2} \\
            b^* \sin \frac{\theta}{2} & 0 & c \sin\frac{\theta}{2}\cos \frac{\theta}{2} & c \sin^2 \frac{\theta}{2}
        \end{pmatrix}.
    \end{align*}
    したがって, 第1qubitの終状態$\rho_\mathrm{out}$は,
    \begin{align*}
        \rho_\mathrm{out}
        =
        \tr_\mathrm{2nd\ qubit} \left[U  \rho_\mathrm{in} \otimes \ket{0}\bra{0} U^\dagger\right]
        =
        \begin{pmatrix}
            a                         & b \cos\frac{\theta}{2} \\
            b^* \cos \frac{\theta}{2} & c
        \end{pmatrix}.
    \end{align*}
    \par
    一方,
    位相ダンピングの演算要素
    \begin{align*}
        E_0 =
        \begin{pmatrix}
            1 & 0                 \\
            0 & \sqrt{1- \lambda}
        \end{pmatrix}
        ,
        E_1 =
        \begin{pmatrix}
            0 & 0              \\
            0 & \sqrt{\lambda}
        \end{pmatrix}
    \end{align*}
    として,
    \begin{align*}
        E_0 \rho_\mathrm{in} E_0^\dagger + E_1 \rho_\mathrm{in} E_1^\dagger =
        \begin{pmatrix}
            a                      & b \sqrt{1 - \lambda} \\
            b^* \sqrt{1 - \lambda} & c
        \end{pmatrix}.
    \end{align*}
    \par
    ゆえに, $\lambda = \sin^2 \frac{\theta}{2}$として,
    \begin{align*}
        \rho_\mathrm{out} = E_0 \rho_\mathrm{in} E_0^\dagger + E_1 \rho_\mathrm{in} E_1^\dagger.
    \end{align*}
\end{ex}


\begin{ex}
    \label{ex8.27}
    \begin{align*}
        u =
        \begin{pmatrix}
            \sqrt{\alpha}     & \frac{\sqrt{\alpha} \left( 1 - \sqrt{1 - 1- \lambda}\right)}{\sqrt{\lambda}}       \\
            \sqrt{1 - \alpha} & - \frac{\sqrt{1 - \alpha} \left( 1 + \sqrt{1 - 1- \lambda}\right)}{\sqrt{\lambda}}
        \end{pmatrix}.
    \end{align*}
\end{ex}

\begin{ex}
    \label{ex8.28}

\end{ex}

\begin{ex}
    \label{ex8.29}
    \underline{振幅ダンピング}
    \begin{align*}
        \qo \left( I \right)
        =
        E_0 E_0^\dagger + E_1 E_1^\dagger
        =
        \begin{pmatrix}
            1 + \gamma & 0          \\
            0          & 1 - \gamma
        \end{pmatrix}
    \end{align*}
    \underline{位相ダンピング, 分極解消}
    \begin{align*}
        \qo \left( I \right)
        = I
    \end{align*}
\end{ex}

\begin{ex}
    \label{ex8.30}
    初期状態$\rho_\mathrm{init}$は,
    \begin{align*}
        \rho_\mathrm{init}
        =
        \begin{pmatrix}
            a   & b     \\
            b^* & 1 - a
        \end{pmatrix}.
    \end{align*}
    コヒーレンス劣化を考えると, 時刻$t$での状態は,
    \begin{align*}
        \rho\left(t\right)
        =
        \begin{pmatrix}
            \left(a - a_0 \right) e^{ - \frac{t}{T_1}} + a_0 & b e^{-\frac{t}{T_2}}                                   \\
            b^* e^{-\frac{t}{T_2}}                           & - \left(a - a_0 \right) e^{ - \frac{t}{T_1}} + 1 - a_0
        \end{pmatrix} \tag{$\star$} .
    \end{align*}
    \par
    \underline{振幅ダンピング(AD)によるコヒーレンス劣化} \ \par
    式(8.112)より,
    \begin{align*}
        \qo_\mathrm{AD}\left( \rho_\mathrm{init} \right)
        =
        \begin{pmatrix}
            1 - \left(1-a\right)\left(1-\gamma\right) & b \sqrt{1-\gamma}                     \\
            b^* \sqrt{1-\gamma}                       & \left(1-a\right)\left(1-\gamma\right)
        \end{pmatrix}
    \end{align*}
    で, \ $\sqrt{1-\gamma} = e^{- \frac{t}{T_2}}$とすると,
    \begin{align*}
        \qo_\mathrm{AD}\left( \rho_\mathrm{init} \right)
        =
        \begin{pmatrix}
            1 - \left(1-a\right)e^{- \frac{2t}{T_2}} & b e^{- \frac{t}{T_2}}                \\
            b^* e^{- \frac{t}{T_2} }                 & \left(1-a\right)e^{- \frac{2t}{T_2}}
        \end{pmatrix}
    \end{align*}
    この式と$(\star)$で$a_0 = 1$とした式を比べて,
    \begin{align*}
        T_1 = \frac{T_2}{2}.
    \end{align*}

    \underline{振幅ダンピング(AD)と位相ダンピング(PD)両方によるコヒーレンス劣化} \ \par
    \begin{align*}
        \qo_\mathrm{PD}
        \left(
        \qo_\mathrm{AD}\left( \rho_\mathrm{init} \right)
        \right)
        =
        \begin{pmatrix}
            1 - \left( 1 - \gamma_\mathrm{AD}\right)\left(1-a\right)        & b \sqrt{1 - \lambda_\mathrm{PD}} \sqrt{1 - \gamma_\mathrm{AD}} \\
            b^* \sqrt{1 - \lambda_\mathrm{PD}}\sqrt{1 - \gamma_\mathrm{AD}} & \left( 1 - \gamma_\mathrm{AD}\right)\left(1-a\right)           \\
        \end{pmatrix}
    \end{align*}
    で,
    \begin{align*}
        1-\gamma_\mathrm{AD} = e^{- \frac{t}{T_1}}
        , \ \sqrt{1- \lambda_\mathrm{PD}} \sqrt{1-\gamma_\mathrm{AD}} = e^{- \frac{t}{T_2}}
    \end{align*}
    とすると$(\star)$で$a_0 = 1$とした式に一致する. このとき,
    \begin{align*}
        e^{- \frac{t}{T_2}}
        = \sqrt{1- \lambda_\mathrm{PD}} \sqrt{1-\gamma_\mathrm{AD}}
        =  \sqrt{1- \lambda_\mathrm{PD}} e^{- \frac{t}{2 T_1}} \leq e^{-\frac{t}{2T_1}}
        \to
        T_1 \geq \frac{T_2}{2}.
    \end{align*}
\end{ex}

\begin{ex}
    \label{ex8.31}
    考える系のHamiltonian;
    \begin{align*}
        H = \chi a^\dagger a \left( b + b^\dagger\right)
        =  \chi N_a \left( b + b^\dagger\right)
    \end{align*}
    の時間発展演算子$U$は,
    \begin{align*}
        U = e^{ - i H t} = e^{ - i \chi t N_a \left( b + b^\dagger\right) }.
    \end{align*}
    ここで,
    \begin{align*}
        \left[ - i \chi t N_A b^\dagger,  - i \chi t N_A b\right] = \left( \chi t\right)^2 N_a
    \end{align*}
    であることと, Baker-Campbell-Hausdorfの公式;
    \begin{align*}
        \left[ A, B\right] = \mathrm{c \ number} \to e^{A+B} = e^{- \frac{c}{2}} e^A e^B
    \end{align*}
    を用いて,
    \begin{align*}
        U
        = e^{ - i \chi t N_a \left( b + b^\dagger\right) }
        = e^{ - \frac{\left( \chi t\right)^2}{2}N_a^2} e^{- i \chi t N_a b^\dagger}e^{ i \chi t N_a b}.
    \end{align*}
    系の初期状態
    $\ket{ \psi_0} = \ket{\mathrm{system}} \otimes \ket{\mathrm{environment}}$
    を
    \begin{align*}
        \ket{\psi_0} = \sum_n c_n \ket{n} \otimes \ket{0}
    \end{align*}
    とすると, 時刻$t$での系の状態は,
    \begin{align*}
        U \ket{\psi_0}
         & =
        \sum_n c_n
        e^{- \frac{\left( \chi t\right)^2}{2}N_a^2}
        e^{- i \chi t N_a b^\dagger}
        e^{ i \chi t N_a b}
        \ket{n} \otimes \ket{0} \\
         & =
        \sum_n c_n
        e^{ - \frac{\left( \chi t\right)^2}{2}N_a^2}
        e^{- i \chi t N_a b^\dagger}
        \ket{n} \otimes \ket{0} \\
         & =
        \sum_n \sum_k c_n
        e^{ - \frac{\left( \chi t\right)^2}{2}N_a^2}
        \frac{\left( - i \chi t N_a b^\dagger \right)^k}{k!}
        \ket{n} \otimes \ket{0} \\
         & =
        \sum_n \sum_k c_n
        e^{- \frac{\left( \chi t\right)^2}{2}N_a^2}
        \frac{\left( - i \chi t n \right)^k}{ \sqrt{k!}}
        \ket{n} \otimes \ket{k} \\
         & =
        \sum_n \sum_k c_n
        e^{- \frac{\left( \chi t\right)^2}{2}n^2}
        \frac{\left( - i \chi t n \right)^k}{ \sqrt{k!}}
        \ket{n} \otimes \ket{k} \\
    \end{align*}
    となる. 時刻$t$での系の状態を密度行列$\rho$でかくと,
    \begin{align*}
        \rho
        =
        U \ket{\psi_0} \bra{\psi_0} U ^\dagger
         & =
        \sum_{n,m} \sum_{k,l}
        c_n c_m^*
        e^{\frac{\left( \chi t\right)^2}{2} \left( n^2 + m^2\right)}
        \frac{\left( - i \chi t n \right)^k}{ \sqrt{k!}}
        \frac{\left(  i \chi t n \right)^l}{ \sqrt{l!}}
        \ket{n}\bra{m} \otimes \ket{k}\bra{l}.
    \end{align*}
    よって, 時刻$t$での主システムの状態は,
    \begin{align*}
        \tr_\mathrm{env} \rho
         & =
        \sum_{n,m}\sum_{k}
        c_n c_m^*
        e^{ - \frac{\left( \chi t\right)^2}{2} \left( n^2 + m^2\right)}
        \frac{\left( - i \chi t n \right)^k}{ \sqrt{k!}}
        \frac{\left(  i \chi t m \right)^k}{ \sqrt{k!}}
        \ket{n}\bra{m} \\
         & =
        \sum_{n,m}
        c_n c_m^*
        e^{ - \frac{\left( \chi t\right)^2}{2} \left( n^2 + m^2\right)}
        e^{\chi t m n}
        \ket{n}\bra{m} \\
         & =
        \sum_{n,m}
        c_n c_m^*
        e^{ - \frac{\left( \chi t\right)^2}{2} \left( n^2 + m^2\right)}
        e^{ \left( \chi t \right)^2 m n}
        \ket{n}\bra{m} \\
         & =
        \sum_{n,m}
        c_n c_m^*
        e^{ - \frac{\left( \chi t\right)^2}{2} \left( m - n \right)^2}
        \ket{n}\bra{m}.
    \end{align*}
\end{ex}