\chapter{量子力学入門}


\begin{ex}
    \label{ex2.1}
    \begin{align*}
        \begin{pmatrix}
            1 \\ -1
        \end{pmatrix}
        +
        \begin{pmatrix}
            1 \\ 2
        \end{pmatrix}
        -
        \begin{pmatrix}
            2 \\ 1
        \end{pmatrix}
        =
        0
    \end{align*}
\end{ex}

\begin{ex}
    \label{ex2.2}
    入出力基底が共に,
    \begin{align*}
        \ket{0},\ket{1}
    \end{align*}
    のとき,
    \begin{align*}
        A
        =
        \begin{pmatrix}
            0 & 1 \\
            1 & 0
        \end{pmatrix}
    \end{align*}
    上記の入出力基底を, 規定の変換行列$U$
    \begin{align*}
        U
        =
        \frac{1}{\sqrt{2}}
        \begin{pmatrix}
            1 & 1  \\
            1 & -1
        \end{pmatrix}
    \end{align*}
    を用いて,
    \begin{align*}
        \frac{\ket{0}+\ket{1}}{\sqrt{2}},\frac{\ket{0}-\ket{1}}{\sqrt{2}}
    \end{align*}
    に取り替えると, 表現行列は,
    \begin{align*}
        U^{-1}AU
        =
        \begin{pmatrix}
            1 & 0  \\
            0 & -1
        \end{pmatrix}.
    \end{align*}
\end{ex}

\begin{ex}
    \label{ex2.1}
    問題文の$A,B$というオペレーターを$T_A, T_B$と書き, 問題文で与えられた基底に対するその表現行列をそれぞれ$A,B$と書くとすると,
    \begin{align*}
        T_B T_A \ket{v_i}
        = T_B A_{ji}\ket{w_j}
        = A_{ji} B_{kj} \ket{x_k}
        = B_{kj} A_{ji} \ket{x_k}
    \end{align*}
    なので, $T_BT_A$の基底$\{\ket{v_i}\}$から$\{\ket{x_i}\}$への表現行列は$BA$.
\end{ex}

\begin{ex}
    \label{ex2.4}
    任意の状態$\ket{\psi}$は, $V$の基底$\{\ket{v_i}\}$を用いて,
    \begin{align*}
        \ket{\psi} = \sum_i c_i \ket{v_i}
    \end{align*}
    とかけ,
    $V \to V$の単位オペレーター$I$は, $\ket{\psi}$に対して,
    \begin{align*}
        I \ket{\psi} = \ket{\psi}
    \end{align*}
    つまり
    \begin{align*}
        \sum_{i,j} c_i I_{ji} \ket{v_j} = \sum_{j} c_j \ket{v_j}
    \end{align*}
    のように作用するので,
    \begin{align*}
        0 =
        \sum_{i,j} \left( c_i I_{ji} - \frac{c_j}{\dim V} \right) \ket{v_j}
    \end{align*}
    が成立. よって, 基底の1次独立性から全ての$j$に対して,
    \begin{align*}
        c_j = \sum_i c_i I_{ji}
    \end{align*}
    つまり
    \begin{align*}
        I_{ij} = \delta_{ij}.
    \end{align*}
    ゆえに, $I$の表現行列は単位行列.
\end{ex}

\begin{ex}
    \label{ex2.5}
    式(2.13)の内積の定義を満たすか確かめれば良い.
\end{ex}

\begin{ex}
    \label{ex2.6}
    \begin{align*}
        \left(
        \lambda_i \ket{w_i} , \ket{v}
        \right)
        =
        \left(
        \ket{v},
        \lambda_i \ket{w_i}
        \right)^*
        =
        \lambda_i^*
        \left(
        \ket{v},\ket{w_i}
        \right)^*
        =
        \lambda_i^*
        \left(
        \ket{w_i},\ket{v}
        \right)
    \end{align*}
\end{ex}

\begin{ex}
    \label{ex2.7}
    式(2.14)で定義された$\bm{C}^2$の標準内積を用いることにすると,
    \begin{align*}
        \braket{w|v}=0.
    \end{align*}
    $\ket{w},\ket{v}$を規格化すると, それぞれ
    \begin{align*}
        \frac{1}{\sqrt{2}}
        \begin{pmatrix}
            1 \\ 1
        \end{pmatrix}
        ,\
        \frac{1}{\sqrt{2}}
        \begin{pmatrix}
            1 \\ -1
        \end{pmatrix}.
    \end{align*}
\end{ex}

\begin{ex}
    \label{ex2.8}
    $d$次元の計量線型空間$V$の基底$\{\ket{w_k}\}_{k=1}^d$に対して,
    \begin{align*}
        \ket{v_{k}}
        =
        \begin{dcases}
            \frac{w_{k}}{||w_{k}||} & (k=1)                \\
            \frac{
            \ket{w_{k}}-\sum_{i=1}^{k-1}\braket{v_i|w_{k}} \ket{v_i}
            }{
            ||\ket{w_{k}}-\sum_{i=1}^{k-1}\braket{v_i|w_{k}} \ket{v_i}||
            }                       & (\mathrm{otherwise})
        \end{dcases}
    \end{align*}
    で定義された$\{\ket{v_k}\}_{k=1}^d$が$V$の正規直交基底になっていること示す.
    \par
    正規性は, 明らかである.
    \par
    直交性について, 帰納法で示す.
    $k=1$のとき,
    \begin{align*}
        \braket{v_1|v_2}
        \propto
        \braket{v_1|w_2}- \braket{v_1|w_2} \braket{v_1|v_1} = 0.
    \end{align*}
    また, $i \neq j\ (i,j=1,2 \dots, k)$なる任意のの$i,j$で,
    \begin{align*}
        \braket{v_j|v_i} = 0
    \end{align*}
    だと仮定すると,
    \begin{align*}
        \braket{v_j|v_{k+1}}
        \propto
        \braket{v_j|w_{k+1}}-\sum_{i=1}^k\braket{v_i|w_{k+1}} \braket{v_j|v_i}
        =
        \braket{v_j|w_{k+1}}-\braket{v_j|w_{k+1}}\braket{v_j|v_j}
        =
        0.
    \end{align*}
    以上より, $i,j=1,2,\dots,d$に対して,
    \begin{align*}
        \braket{v_i|v_j} = \delta_{ij}
    \end{align*}
    が言えた. したがって, $\{\ket{v_k}\}_{k=1}^d$は線形独立で, $V$を張る. つまり, $\{\ket{v_k}\}_{k=1}^d$は$V$の正規直交基底.
\end{ex}

\begin{ex}
    \label{ex2.9}
    \begin{align*}
        \sigma_0 & = \ket{0}\bra{0} + \ket{1}\bra{1}      \\
        \sigma_1 & = \ket{0}\bra{1} + \ket{1}\bra{0}      \\
        \sigma_2 & = -i \ket{0}\bra{1} + i \ket{1}\bra{0} \\
        \sigma_3 & = \ket{0}\bra{0} - \ket{1}\bra{1}
    \end{align*}
\end{ex}

\begin{ex}
    \label{ex2.10}
    式(2.25)より,
    \begin{align*}
        \ket{v_j}\bra{v_k}
        = \sum_{i,l} \ket{v_i}\braket{v_i|v_j}\braket{v_k|v_l}\bra{v_l}
        = \sum_{i,l} \ket{v_i}\delta_{ij}\delta_{kl}\bra{v_l}
    \end{align*}
    なので, 正規直交基底の下での$\ket{v_j}\bra{v_k}$表現行列$A$の成分$A_{il}$は,
    \begin{align*}
        A_{il} = \delta_{ij}\delta_{kl}.
    \end{align*}
\end{ex}

\begin{ex}
    \label{ex2.11}
    $X$の固有値は$1,-1$で, 対応する規格化された固有ベクトルはそれぞれ,
    \begin{align*}
        \frac{1}{\sqrt{2}}
        \begin{pmatrix}
            1 \\ 1
        \end{pmatrix}
        ,\
        \frac{1}{\sqrt{2}}
        \begin{pmatrix}
            1 \\ -1
        \end{pmatrix}.
    \end{align*}

    $Y$の固有値は$1,-1$で, 対応する規格化された固有ベクトルはそれぞれ,
    \begin{align*}
        \frac{1}{\sqrt{2}}
        \begin{pmatrix}
            i \\ 1
        \end{pmatrix}
        ,\
        \frac{1}{\sqrt{2}}
        \begin{pmatrix}
            i \\ -1
        \end{pmatrix}.
    \end{align*}

    $Z$の固有値は$1,-1$で, 対応する規格化された固有ベクトルはそれぞれ,
    \begin{align*}
        \frac{1}{\sqrt{2}}
        \begin{pmatrix}
            1 \\ 0
        \end{pmatrix}
        ,\
        \frac{1}{\sqrt{2}}
        \begin{pmatrix}
            0 \\ 1
        \end{pmatrix}.
    \end{align*}
\end{ex}

\begin{ex}
    \label{ex2.12}
    \begin{align*}
        A =
        \begin{pmatrix}
            1 & 0 \\
            1 & 1
        \end{pmatrix}
    \end{align*}
    の固有方程式は,
    \begin{align*}
        (\lambda-1)^2=0
    \end{align*}
    となり, $A$の固有空間の直和$W$として,
    \begin{align*}
        W =
        \left\{
        t
        \begin{pmatrix}
            0 \\1
        \end{pmatrix}
        \middle| \ t \in \bm{C}
        \right\}
        \neq \bm{C}^2
    \end{align*}
    より, $A$は対角化不可能.
\end{ex}

\begin{ex}
    \label{ex2.13}
    \begin{align*}
        (\ket{w} \bra{v})^\dagger = \bra{v}^\dagger \ket{w}^\dagger = \ket{v} \bra{w}
    \end{align*}
\end{ex}

\begin{ex}
    \label{ex2.14}
    任意の$\ket{v},\ket{w} \in V$に対して,
    \begin{align*}
        \left(\left(\sum_i a_i A_i \right)^\dagger
        \ket{v}, \ket{w} \right)
        =
        \left(\ket{v}, \sum_i a_i A_i \ket{w} \right)
        =
        \sum_i a_i\left(\ket{v},  A_i \ket{w} \right)
        =
        \sum_i \left( a_i^* \ket{v},  A_i \ket{w} \right)
        =
        \left(\sum_i a_i^* A_i^\dagger \ket{v},  \ket{w} \right)
    \end{align*}
    であり, $\ket{w}$は任意なので,
    \begin{align*}
        \left(\sum_i a_i A_i \right)^\dagger = \sum_i a_i^* A_i^\dagger
    \end{align*}
\end{ex}

\begin{ex}
    \label{ex2.15}
    任意の$\ket{v},\ket{w} \in V$に対して,
    \begin{align*}
        \left(\ket{v}, A \ket{w} \right)
        =
        \left(A^\dagger \ket{v}, \ket{w} \right)
        =
        \left(\ket{v}, (A^\dagger)^\dagger \ket{w} \right)
    \end{align*}
\end{ex}

\begin{ex}
    \label{ex2.16}
    \begin{align*}
        P^2
        = \sum_i \sum_j \ket{i}\bra{i} \ket{j}\bra{j}
        = \sum_i \sum_j \ket{i}\delta_{ij}\bra{j}
        = \sum_i \ket{i}\bra{i}
        = P
    \end{align*}
\end{ex}

\begin{ex}
    \label{ex2.17}
    「正規行列$A$の固有値が実数$\Longleftrightarrow$正規行列$A$はHermite」を示す.
    \par
    $\Longrightarrow)$
    スペクトル分解をすると, $a\in \mathrm{R}$として,
    \begin{align*}
        A = \sum_a a \ket{a} \bra{a}
    \end{align*}
    とかけるので,
    \begin{align*}
        A^\dagger =  \sum_a a^* \ket{a} \bra{a} = \sum_a a\ket{a} \bra{a} = A.
    \end{align*}
    \par
    $\Longleftarrow)$
    $A$がHermiteとする. $A$の固有値$\lambda$と対応する固有ベクトル$\ket{v}$として,
    \begin{align*}
        \lambda = \left(\ket{v}, A \ket{v} \right) = \left( A \ket{v}, \ket{v} \right) = \lambda^*
    \end{align*}
    より, $\lambda$は実数であることが言えた.
\end{ex}

\begin{ex}
    \label{ex2.18}
    ユニタリ行列$U$の固有値$\lambda$と対応する固有ベクトル$\ket{v}$として,
    \begin{align*}
        \braket{v|v} = \left( U \ket{v}, U \ket{v} \right) = |\lambda|^2 \braket{v|v}
    \end{align*}
    で, $\braket{v|v}\neq 1$なので,
    \begin{align*}
        |\lambda| = 1.
    \end{align*}
\end{ex}

\begin{ex}
    \label{ex2.19}
    Pauli行列の定義より明らか.
\end{ex}

\begin{ex}
    \label{ex2.20}
    完全性条件を挟んで,
    \begin{align*}
        A_{ij}^{''}
        = \braket{w_i|A|w_j}
        = \sum_k \sum_l \braket{w_i|v_k}\braket{v_k|A|v_l}\braket{v_l|w_j}
        = \sum_k \sum_l \braket{w_i|v_k}A_{kl}^{'}\braket{v_l|w_j}
    \end{align*}
\end{ex}

\begin{ex}
    \label{ex2.21}
    $M$がHermiteならば, $\mathrm{式}(2.37)=\mathrm{式}(2.41)$が明らか.
\end{ex}


\begin{ex}
    \label{ex2.22}
    Hermite オペレータ $A$の異なる固有値$\lambda_i, \lambda_j \ (\lambda_i \neq \lambda_j)$
    と対応する固有ベクトル$\ket{i}, \ket{j}$とすると,
    \begin{align*}
        0 = \left(\ket{i}, A \ket{j} \right) - \left( A \ket{i}, \ket{j} \right) = (\lambda_j - \lambda_i^*) \braket{i|j}
        \to
        \braket{i|j} = 0
    \end{align*}
\end{ex}

\begin{ex}
    \label{ex2.23}
    射影オペレータ $P$の固有値$\lambda$と対応する固有ベクトル$\ket{v}$とすると, \ $P^2=P$より,
    \begin{align*}
        0 = (P^2 - P)\ket{v} = \lambda(\lambda-1)\ket{v}.
    \end{align*}
    左から$\bra{v}$をかけて,
    \begin{align*}
        0 = \lambda(\lambda-1) \to \lambda = 0,1.
    \end{align*}
\end{ex}

\begin{ex}
    \label{ex2.24}
    任意のオペレータ$A$は,
    \begin{align*}
        A = \frac{A + A^\dagger}{2} + i \frac{A-A^\dagger}{2i}
    \end{align*}
    の形でかける.
    また, 任意の$\ket{v}$に対して, Hermite オペレータ $H$の期待値は,
    \begin{align*}
        \left( \ket{v}, H \ket{v} \right)
        = \left( H^\dagger \ket{v}, \ket{v} \right)
        = \left( H \ket{v}, \ket{v} \right)
        = \left( \ket{v}, H \ket{v} \right)^*
    \end{align*}
    と実となるので,
    \begin{align*}
        \braket{v|\frac{A + A^\dagger}{2}|v}
        ,\ \braket{v|\frac{A-A^\dagger}{2i}|v}
        \in \mathbb{R}.
    \end{align*}
    よって, $A$を正のオペレーターとすると, 任意の$\ket{v}$に対して,
    \begin{align*}
        0 \leq \braket{v|A|v}
        = \braket{v|\frac{A + A^\dagger}{2}|v}
        + i \braket{v|\frac{A-A^\dagger}{2i}|v}
        \to
        \braket{v|\frac{A-A^\dagger}{2i}|v} = 0.
    \end{align*}
    $\ket{v}$は, 任意なので, $A = A^\dagger$が成り立つ.
\end{ex}

\begin{ex}
    \label{ex2.25}
    \begin{align*}
        \braket{v|A^\dagger A | v}
        = \left(\ket{v}, A^\dagger A\ket{v} \right)
        = \left(A \ket{v},A\ket{v} \right)
        = || A \ket{v}|| ^2 \geq 0.
    \end{align*}
\end{ex}

\begin{ex}
    \label{ex2.26}
    \begin{align*}
        \ket{0} =
        \begin{pmatrix}
            1 \\ 0
        \end{pmatrix}
        , \
        \ket{1} =
        \begin{pmatrix}
            0 \\ 1
        \end{pmatrix}
    \end{align*}
    とする. テンソル積の形で書くと,
    \begin{align*}
        \ket{\psi}^{\otimes 2} & = \frac{\ket{00} + \ket{01} + \ket{10} + \ket{11}}{2} \\
        \ket{\psi}^{\otimes 3} & = \frac{
            \ket{000} + \ket{001} + \ket{010} + \ket{011}
            +\ket{100} + \ket{101} + \ket{110} + \ket{111} }{2\sqrt{2}}.
    \end{align*}
    Kronecker積の形で書くと,
    \begin{align*}
        \ket{\psi}^{\otimes 2}
         & = \frac{1}{2}
        \begin{pmatrix}
            1 \\ 1 \\ 1 \\ 1
        \end{pmatrix} \\
        \ket{\psi}^{\otimes 3}
         & = \frac{1}{2\sqrt{2}}
        \begin{pmatrix}
            1 \\ 1 \\ 1 \\ 1 \\ 1 \\ 1 \\ 1 \\ 1
        \end{pmatrix}.
    \end{align*}
\end{ex}

\begin{ex}
    \label{ex2.27}
    \begin{align*}
        X \otimes Z
        =
        \begin{pmatrix}
            0 & 0  & 1 & 0  \\
            0 & 0  & 0 & -1 \\
            1 & 0  & 0 & 0  \\
            0 & -1 & 0 & 0  \\
        \end{pmatrix}
        , \
        I \otimes X
        =
        \begin{pmatrix}
            0 & 1 & 0 & 0 \\
            1 & 0 & 0 & 0 \\
            0 & 0 & 0 & 1 \\
            0 & 0 & 1 & 0 \\
        \end{pmatrix}
        , \
        X \otimes I
        =
        \begin{pmatrix}
            0 & 0 & 1 & 0 \\
            0 & 0 & 0 & 1 \\
            1 & 0 & 0 & 0 \\
            0 & 1 & 0 & 0 \\
        \end{pmatrix}.
    \end{align*}
    $I \otimes X \neq X \otimes I$にあるようにテンソル積は非可換.
\end{ex}

\begin{ex}
    \label{ex2.28}
    テンソル積の転置共役$\left( A \otimes B \right)^\dagger$を,
    \begin{align*}
        \left(
        \left(A \otimes B \right)^\dagger
        \left( \ket{v_1} \otimes \ket{w_1}\right),
        \ket{v_2} \otimes \ket{w_2}
        \right)
        =
        \left(
        \ket{v_1} \otimes \ket{w_1},
        \left( A \otimes B \right) \left( \ket{v_2} \otimes \ket{w_2} \right)
        \right)
    \end{align*}
    で定義すると,
    \begin{align*}
        \left(
        \left(A \otimes B \right)^\dagger
        \left( \ket{v_1} \otimes \ket{w_1}\right),
        \ket{v_2} \otimes \ket{w_2}
        \right)
         & =
        \left(
        \ket{v_1} \otimes \ket{w_1},
        A \ket{v_2} \otimes B \ket{w_2}
        \right) \\
         & =
        \left(
        \ket{v_1},A \ket{v_2}
        \right)
        \left(
        \ket{w_2},B \ket{w_2}
        \right) \\
         & =
        \left(
        A^\dagger \ket{v_1}, \ket{v_2}
        \right)
        \left(
        B^\dagger \ket{w_1}, \ket{w_2}
        \right) \\
         & =
        \left(
        \left(A^\dagger \otimes B^\dagger \right)
        \left( \ket{v_1} \otimes \ket{w_1}\right),
        \ket{v_2} \otimes \ket{w_2} .
        \right)
    \end{align*}
    $\ket{v_1} \otimes \ket{w_1}$
    ,
    $\ket{v_2} \otimes \ket{w_2}$は任意なので, $ \left(A \otimes B \right)^\dagger=A^\dagger \otimes B^\dagger$を得る.
    \par
    テンソル積の複素共役$\left(A \otimes B \right)^*$を, オペレータ形式でどう定義すればわからないので,
    Kronecker積の形で考える.
    \begin{align*}
        \left(A \otimes B \right)^*
        =
        \begin{pmatrix}
            A_{11}B & A_{12}B & \dots  & A_{1n}B \\
            A_{21}B & A_{22}B & \dots  & A_{2n}B \\
            \vdots  & \vdots  & \vdots & \vdots  \\
            A_{m1}B & A_{m2}B & \dots  & A_{mn}B \\
        \end{pmatrix}^*
        =
        \begin{pmatrix}
            A_{11}^* B^* & A_{12}^*B^* & \dots  & A_{1n}^*B^* \\
            A_{21}^*B^*  & A_{22}^*B^* & \dots  & A_{2n}^*B^* \\
            \vdots       & \vdots      & \vdots & \vdots      \\
            A_{m1}^*B^*  & A_{m2}^*B^* & \dots  & A_{mn}^*B^* \\
        \end{pmatrix}
        =
        A^* \otimes B^*
    \end{align*}
    \par
    テンソル積の転置$\left(A \otimes B \right)^T$を,
    \begin{align*}
        \left(A \otimes B \right)^T
        =
        \left(A \otimes B \right)^{*\dagger}
    \end{align*}
    で定義すると, 上で示したテンソル積の転置共役, 複素共役に対する分配性から,
    \begin{align*}
        \left(A \otimes B \right)^T
        =
        \left(A \otimes B \right)^{*\dagger}
        =
        \left(A^* \otimes B^* \right)^{\dagger}
        =
        A^{*\dagger} \otimes B^{*\dagger}
        =
        A^T \otimes B^T.
    \end{align*}
\end{ex}

\begin{ex}
    \label{ex2.29}
    $U_1, U_2$がユニタリのとき,
    \begin{align*}
        (U_1 \otimes U_2)(U_1 \otimes U_2)^\dagger
        =
        (U_1 \otimes U_2)(U_1^\dagger\otimes U_2^\dagger)
        =
        (U_1U_1^\dagger) \otimes (U_2U_2^\dagger)
        =
        I_1 \otimes I_2
    \end{align*}
\end{ex}

\begin{ex}
    \label{ex2.30}
    $A_1, A_2$がHetmiteのとき,
    \begin{align*}
        (A_1 \otimes A_2)^\dagger
        =
        A_1^\dagger \otimes A_2^\dagger
        =
        A_1 \otimes A_2
    \end{align*}
\end{ex}

\begin{ex}
    \label{ex2.31}
    $A, B$が正のオペレータのとき, 任意のベクトル$\ket{v}\otimes \ket{w}$に対して,
    \begin{align*}
        \left( \bra{v}\otimes \bra{w} \right)
        (A \otimes B )
        \left( \ket{v}\otimes \ket{w} \right)
        =
        \braket{v|A|v}
        \braket{w|B|w}
        \geq 0
    \end{align*}
    より, $A \otimes B$も正のオペレータ.
\end{ex}

\begin{ex}
    \label{ex2.32}
    ここでは, 射影オペレータ$P$を,
    \begin{align*}
        P^\dagger = P , P^2 = P
    \end{align*}
    を満たすオペレータと定義する. この定義は, 式(2.35)と矛盾しない.
    \par
    $P_1, P_2$が射影オペレータのとき,
    \begin{align*}
        \left( P_1 \otimes P_2 \right)^\dagger
         & =
        P_1 \otimes P_2
        \\
        \left( P_1 \otimes P_2 \right) \left( P_1 \otimes P_2 \right)
         & =
        I_1 \otimes I_2
    \end{align*}
    より, $P_1 \otimes P_2$は射影オペレータ.
\end{ex}

\begin{ex}
    \label{ex2.33}
    Hadamard変換$H$は,
    \begin{align*}
        H =
        \frac{1}{\sqrt{2}}
        \left(
        \ket{0} \bra{0} + \ket{1} \bra{0} + \ket{0} \bra{1} - \ket{1} \bra{1}
        \right).
    \end{align*}
    上式の最後の項の$-$符号に注意する.
    \par
    例えば, $n=2$のとき,
    \begin{align*}
        H^{\otimes 2}
         & =
        \frac{1}{2}
        \big(
        \ket{0} \bra{0} + \ket{1} \bra{0} + \ket{0} \bra{1} - \ket{1} \bra{1}
        \big)
        \otimes
        \big(
        \ket{0} \bra{0} + \ket{1} \bra{0} + \ket{0} \bra{1} - \ket{1} \bra{1}
        \big)
        \\
         & =
        \frac{1}{2}
        \big(
        \ket{00} \bra{00} + \ket{01} \bra{00} + \ket{00} \bra{01} - \ket{01} \bra{01}
        +
        \ket{10} \bra{00} + \ket{11} \bra{00} + \ket{10} \bra{01} - \ket{11} \bra{01}
        \\
         & \ \ \ \ \ \ \ + \ket{00} \bra{10} + \ket{01} \bra{10} + \ket{00} \bra{11} - \ket{01} \bra{11}
        -
        \ket{10} \bra{10} - \ket{11} \bra{10} - \ket{10} \bra{11} + \ket{11} \bra{11}
        \big)                                                                                            \\
         & =
        \frac{1}{2} \sum_{x,y} (-1)^{x \cdot y}\ket{x}\bra{y} .
    \end{align*}
    ここで, $x,y$は各成分が0または1の2次元のベクトルで, $x \cdot y $は標準内積.
    \par
    一般の$n$に対しても,
    $x,y$を各成分が0または1の$n$次元のベクトル, $x \cdot y $を標準内積として,
    \begin{align*}
        H^{\otimes n}
        = \frac{1}{\sqrt{2^n}} \sum_{x,y} (-1)^{x \cdot y}\ket{x}\bra{y}
    \end{align*}
    を得ることは少し考えればわかる.
    \par
    特に, Kronecker積の形で書くと,
    \begin{align*}
        H
        =
        \frac{1}{\sqrt{2}}
        \begin{pmatrix}
            1 & 1  \\
            1 & -1 \\
        \end{pmatrix}
        ,\
        H^{\otimes 2}
        =
        \frac{1}{2}
        \begin{pmatrix}
            1 & 1  & 1  & 1  \\
            1 & -1 & 1  & -1 \\
            1 & 1  & -1 & -1 \\
            1 & -1 & -1 & 1
        \end{pmatrix}.
    \end{align*}
\end{ex}

\begin{ex}
    \label{ex2.34}
    基底
    \begin{align*}
        \ket{0}, \ket{1}
    \end{align*}
    の下での表現行列が,
    \begin{align*}
        \begin{pmatrix}
            4 & 3 \\
            3 & 4
        \end{pmatrix}
    \end{align*}
    なるオペレータ$A$を考える.
    固有値問題を解くと, 基底を
    \begin{align*}
        \ket{+} = \frac{\ket{0}+\ket{1}}{\sqrt{2}},\ \ket{-} = \frac{\ket{0}-\ket{1}}{\sqrt{2}}
    \end{align*}
    取り替えることで, 表現行列$A$が
    \begin{align*}
        \begin{pmatrix}
            7 & 0 \\
            0 & 1
        \end{pmatrix}
    \end{align*}
    と対角化できることがわかる. つまり,
    \begin{align*}
        A = 7 \ket{+}\bra{+} + 1 \ket{-}\bra{-} .
    \end{align*}
    ゆえに, オペレータ$A$の平方根$f(A)$を
    \begin{align*}
        \sqrt{A} = \sqrt{7} \ket{+}\bra{+} +  \ket{-}\bra{-}
    \end{align*}
    で定義できる. 同様に, オペレータ$A$の対数$\log{A}$を
    \begin{align*}
        \sqrt{A} = \log{7} \ket{+}\bra{+}
    \end{align*}
    で定義できる.
\end{ex}

\begin{ex}
    \label{ex2.35}
    \begin{align*}
        \bm{v} \cdot \bm{\sigma}
        =
        \begin{pmatrix}
            v_3         & v_1 - i v_2 \\
            v_1 + i v_2 & - v_3
        \end{pmatrix}
    \end{align*}
    の固有値$\lambda = \pm 1$の対応する固有ベクトルをそれぞれ$\ket{-1}, \ket{1}$とかくと,
    \begin{align*}
        \bm{v} \cdot \bm{\sigma} = \ket{1}\bra{1} - \ket{-1}\bra{-1} .
    \end{align*}
    よって,
    \begin{align*}
        \exp{\left( i \theta \bm{v} \cdot \bm{\sigma} \right)}
         & =
        e^{i \theta} \ket{1}\bra{1} - e^{- i \theta} \ket{-1}\bra{-1} \\
         & =
        \cos{\theta} \big(\ket{1}\bra{1} + \ket{-1}\bra{-1} \big)
        + i \sin{\theta} \big(\ket{1}\bra{1} - \ket{-1}\bra{-1} \big)
        \\
         & =
        (\cos{\theta}) I + i (\sin{\theta}) \bm{v} \cdot \bm{\sigma}
    \end{align*}
\end{ex}

\begin{ex}
    \label{ex2.36}
    Pauli行列の定義より明らか.
\end{ex}

\begin{ex}
    \label{ex2.37}
    \begin{align*}
        \mathrm{tr}{AB} = A_{ij}B_{ji} =  A_{ij}B_{ji} = \mathrm{tr}{BA}
    \end{align*}
\end{ex}

\begin{ex}
    \label{ex2.38}
    トレースの定義と$\sum$の線型性より明らか.
\end{ex}

\begin{ex}
    \label{ex2.39}
    (1)\
    $L_V \times L_V$上で定義された内積
    \begin{align*}
        \left( A, B\right) = \tr(A^\dagger B)
    \end{align*}
    が, 式(2.13)を満たすか調べれば良い;
    \begin{align*}
         & \left( A, \sum_i \lambda_i B_i \right)
        =
        \tr\left(\sum_i \lambda_i A^\dagger B_i \right)
        =
        \sum_i \lambda_i  \tr \left(A^\dagger B_i \right)
        =
        \sum_i \lambda_i  \left( A, B_i \right)
        \\
         & (A, B) = \tr(A^\dagger B) = \tr( B^T A^*) = \tr( B^\dagger A)^* = (B, A)^*
        \\
         &
        (A,A) = \tr(A^\dagger A)
        =
        \sum_{i,j} A^\dagger_{ij} A_{ji}
        =
        \sum_{i,j} A^*_{ji} A_{ji}
        =
        \sum_{i,j} | A_{ji}|^2
        \geq 0
        \\
         &
        0 = (A,A) \Leftrightarrow A = O .
    \end{align*}
    \par
    (2)\ $V$が$d$次元のとき, $A: V \to V$なるオペレータ$A$は$d^2$の自由度を持つので,
    $L_V$は$d^2$次元.
    \par
    (3)\ $L_V$の正規直交基底$\{A_i\}_{i=1}^{d^2}$のうち, 全ての$i$に対して$A_i$がHermiteとなる正規直交基底$\{A_i\}_{i=1}^{d^2}$を求める. $V$の正規直交基底を$\{\ket{i}\}_{i=1}^{d}$とすると, $L_V$の正規直交基底は$\{e_{ij}=\ket{i} \bra{j}\}_{i,j=1}^{d}$となる;
    \begin{align*}
        \left( e_{ij}, e_{kl} \right) = \tr\left(\ket{j} \braket{i|k} \bra{l} \right) = \delta_{ij} \delta_{kl}.
    \end{align*}
    $\{e_{ij}\}_{i,j=1}^{d}$は, $i=j$のときHermiteだが, $i \neq j$のときはHermiteではない.
    そこで, $\{e_{ii}\}_{i=1}^{d}$で張られる$L_V$の部分空間$L_P$とその正規直交補空間$L_Q$を考える. $L_P$のHermiteな正規直交基底は, $\{e_{ii}\}_{i=1}^{d}$である. 一方, $L_Q$のHermiteな正規直交基底は,
    \begin{align*}
        e'_{ij} = \frac{e_{ij} + e_{ji}}{\sqrt{2}},
        e''_{ij} = \frac{e_{ij} - e_{ji}}{\sqrt{2}i} \ (i<j)
    \end{align*}
    である. 以上より, $L_V = L_P \oplus L_Q$のHermiteな正規直交基底は,
    \begin{align*}
        \left\{
        \ket{i} \bra{i}, \frac{\ket{i} \bra{j}+ \ket{j} \bra{i}}{\sqrt{2}}, \frac{\ket{i} \bra{j} - \ket{j} \bra{i}}{\sqrt{2}i}
        \right\}_{i,j = 1,2,\dots d , i < j}
    \end{align*}
\end{ex}

\begin{ex}
    \label{ex2.40}
    Pauli行列の定義より明らかに,
    \begin{align*}
        [\sigma_j , \sigma_k] = 2i \epsilon_{jkl} \sigma_l.
    \end{align*}
\end{ex}

\begin{ex}
    \label{ex2.41}
    Pauli行列の定義より明らかに,
    \begin{align*}
        \{ \sigma_i, \sigma_j \} = 2 \delta_{ij}I.
    \end{align*}
\end{ex}

\begin{ex}
    \label{ex2.42}
    \begin{align*}
        [A,B] + \{ A,B \} = AB - BA + AB + BA = 2AB.
    \end{align*}
\end{ex}

\begin{ex}
    \label{ex2.43}
    \begin{align*}
        \sigma_j \sigma_k
        =
        \frac{[\sigma_j,\sigma_k] + \{\sigma_j,\sigma_k \}}{2}
        =
        \delta_{jk} I + i \epsilon_{jkl} \sigma_l.
    \end{align*}
\end{ex}

\begin{ex}
    \label{ex2.44}
    \begin{align*}
        B = A^{-1} A B = A^{-1} \frac{ [A,B] + \{ A,B \} }{2} = 0.
    \end{align*}
\end{ex}

\begin{ex}
    \label{ex2.45}
    \begin{align*}
        [A,B]^\dagger = (AB-BA)^\dagger = B^\dagger A^\dagger - A^\dagger B^\dagger = [B^\dagger, A^\dagger].
    \end{align*}
\end{ex}

\begin{ex}
    \label{ex2.46}
    \begin{align*}
        [A,B] = - (BA - AB) = -[B,A].
    \end{align*}
\end{ex}

\begin{ex}
    \label{ex2.47}
    \begin{align*}
        \left( i [A,B]\right)^\dagger
        =
        -i [B^\dagger, A^\dagger]
        =
        -i [B,A]
        =
        i [A,B].
    \end{align*}
\end{ex}

\begin{ex}
    \label{ex2.48}
    ベクトル空間$V$上で定義されたHermiteの行列$H$のスペクトル分解は, $H$の固有値$\lambda \in \mathrm{R}$として,
    \begin{align*}
        H = \sum_\lambda \lambda \ket{\lambda} \bra{\lambda}
    \end{align*}
    なので,
    \begin{align*}
        J & = \sqrt{H^\dagger H} = \sqrt{H H}
        = \sqrt{\sum_\lambda \sum_{\lambda'} \lambda' \lambda \ket{\lambda}\braket{\lambda|\lambda'}  \bra{\lambda'}}
        = \sqrt{\sum_\lambda \lambda^2 \ket{\lambda} \bra{\lambda}}
        = \sum_\lambda |\lambda| \ket{\lambda} \bra{\lambda} \\
        K & = \sqrt{HH^\dagger} = \sqrt{HH} = J
    \end{align*}
    ここで, $\{ \ket{\lambda} \}$は$V$の正規直交基底になっている. したがって,
    Hermite行列$H$の極分解は,
    \begin{align*}
        H = U \sqrt{H^2} = \sqrt{H^2} U .
    \end{align*}
    特に, $\lambda \geq 0$なら$H$は正の行列$P$となり,
    \begin{align*}
        J = K = P
    \end{align*}
    が成立するので, $P$の極分解は,
    \begin{align*}
        P = I P = P I .
    \end{align*}
    \par
    ユニタリ行列$U$の極分解は,
    \begin{align*}
        U = I U = U I .
    \end{align*}
    \par
\end{ex}

\begin{ex}
    \label{ex2.49}
    ベクトル空間$V$上で定義された正規行列$A$のスペクトル分解は,
    $A$の固有値$a$, 対応する固有ベクトル$\ket{a}$として,
    \begin{align*}
        A = \sum_a a \ket{a} \bra{a} .
    \end{align*}
    ここで, $\{ \ket{a} \}$は$V$の正規直交基底になっている.
    すると,
    \begin{align*}
        J
        =
        \sqrt{A^\dagger A}
        =
        \sqrt{\sum_a \sum_{a'} a' a^* \ket{a}\braket{a|a'}  \bra{a'}}
        =
        \sqrt{\sum_a |a|^2 \ket{a} \bra{a}}
        =
        \sum_a \sqrt{|a|} \ket{a} \bra{a} .
    \end{align*}
    定理2.3の証明より,
    \begin{align*}
        U = \sum_a \ket{e_a} \bra{a}
    \end{align*}
    とすれば, $A$の左極分解は,
    \begin{align*}
        A = UJ .
    \end{align*}
    \par
    右極分解についても同様.
\end{ex}

\begin{ex}
    \label{ex2.50}
    %
    %
    %
    %
    %
    %
    %
    \begin{align*}
        A
        =
        \begin{pmatrix}
            1 & 0 \\
            1 & 1 \\
        \end{pmatrix}
    \end{align*}
    として, $A^\dagger A$
    \begin{align*}
        A^\dagger A
        =
        \begin{pmatrix}
            2 & 1 \\
            1 & 1 \\
        \end{pmatrix}
    \end{align*}
    の固有値は
    \begin{align*}
        \lambda_{\pm} = \frac{3 \pm \sqrt{5}}{2}
    \end{align*}
    で, 対応する固有ベクトル$\ket{\lambda_{\pm}}$は,
    \begin{align*}
        \ket{\lambda_\pm}
        =
        \frac{1}{10\pm2\sqrt{5}}
        \begin{pmatrix}
            1 \pm \sqrt{5} \\ 2
        \end{pmatrix}.
    \end{align*}
    よって, $J = \sqrt{A^\dagger A}$は,
    \begin{align*}
        J = A^\dagger A
        = \sqrt{\lambda_+} \ket{\lambda_+} \bra{\lambda_+} + \sqrt{\lambda_-}\ket{\lambda_-} \bra{\lambda_-}
    \end{align*}
    %
    %
    %
    %
    %
    %
    %
    あとで計算する
\end{ex}

\begin{ex}
    \label{ex2.51}
    \begin{align*}
        H H^\dagger = I.
    \end{align*}
\end{ex}

\begin{ex}
    \label{ex2.52}
    \begin{align*}
        H H = I.
    \end{align*}
\end{ex}

\begin{ex}
    \label{ex2.53}
    \begin{align*}
        \ket{\lambda = \pm 1}
        =
        \begin{pmatrix}
            1 \\ \pm \sqrt{2} - 1
        \end{pmatrix}
    \end{align*}
\end{ex}

\begin{ex}
    \label{ex2.54}
    $A,B$は可換なHermiteなので, 同じ正規直交基底$\{ \ket{i}\}$で同時対角化可能で,
    \begin{align*}
        A = \sum_i a_i \ket{i} \bra{i},
        B = \sum_i b_i \ket{i} \bra{i}
    \end{align*}
    とかけるので,
    \begin{align*}
        \exp(A) \exp(B)
        =
        \sum_i \sum_j e^{a_i} \ket{i} \bra{i}
        e^{b_j} \ket{j} \bra{j}
        =
        \sum_i e^{a_i+b_i} \ket{i} \bra{i}
        =
        \exp(A+B).
    \end{align*}
\end{ex}

\begin{ex}
    \label{ex2.55}
    $H$はHermiteなので, $H,H^\dagger$が可換だから,
    演習\ref{ex2.54}より,
    \begin{align*}
        U(t_1,t_2)U^\dagger(t_1,t_2)
        =
        \exp{\left[ - iH(t_1 - t_2)\right]}
        \exp{\left[ iH^\dagger(t_1 - t_2)\right]}
        =
        \exp{\left[ i(H^\dagger - H)(t_1 - t_2)\right]}
        =
        I.
    \end{align*}
\end{ex}

\begin{ex}
    \label{ex2.56}
    ユニタリオペレータ$U$は正規なので, $\lambda_i=e^{i\theta_i} (\theta_i \in R)$として,
    \begin{align*}
        U = \sum_i \lambda_i \ket{\lambda_i} \bra{\lambda_i}
    \end{align*}
    とスペクトル分解できる. よって,
    \begin{align*}
        K = - i \log{U} = \sum_i \theta_i \ket{i} \bra{i}
    \end{align*}
    となり, これは明らかにHermite. したがって,
    \begin{align*}
        \exp(iK) = \sum_i e^{i\theta_i} \ket{i} \bra{i} = U.
    \end{align*}
\end{ex}

\begin{ex}
    \label{ex2.57}
    状態$\ket{\psi}$に対して, $L_l$を測定した後の状態$\ket{\phi}$は,
    \begin{align*}
        \ket{\phi} = \frac{L_l \ket{\psi}}{\braket{\psi|L_l^\dagger L_l| \psi}}.
    \end{align*}
    さらに, この状態に対して, $M_m$を測定した後の状態は,
    \begin{align*}
        \frac{M_m \ket{\phi}}{\braket{\phi|M_m^\dagger M_m| \phi}}
        =
        \frac{M_m L_l \ket{\psi}}{\braket{\psi| L_l^\dagger M_m^\dagger M_m L_l| \psi}}
        =
        \frac{N_{lm} \ket{\psi}}{\braket{\psi| L_l^\dagger M_m^\dagger N_{lm}| \psi}}.
    \end{align*}
\end{ex}

\begin{ex}
    \label{ex2.58}
    平均測定値$E(M)$は,
    \begin{align*}
        E(M) = \braket{\psi| M |\psi} = m \braket{\psi | \psi} = m.
    \end{align*}
    標準偏差$\Delta(M)$は,
    \begin{align*}
        \Delta(M) = \sqrt{\braket{\psi| M^2 |\psi} - \braket{\psi| M |\psi}^2} = \sqrt{m^2 - m^2} = 0.
    \end{align*}
\end{ex}

\begin{ex}
    \label{ex2.59}
    \begin{align*}
        X = \ket{0}\bra{1} + \ket{1}\bra{0}
    \end{align*}
    なので,
    平均値は,
    \begin{align*}
        \braket{0|X|0} = 0.
    \end{align*}
    標準偏差$\Delta(X)$は,
    \begin{align*}
        \Delta(X)
        = \sqrt{\braket{0| X^2 |0} - \braket{0| X |0}^2}
        = \sqrt{1 - 0}
        = 1.
    \end{align*}
\end{ex}

\begin{ex}
    \label{ex2.60}
    まず, $v_3 \neq 1$のときを考える.
    \begin{align*}
        \bm{v} \cdot \bm{\sigma}
        =
        \begin{pmatrix}
            v_3         & v_1 - i v_2 \\
            v_1 + i v_2 & - v_3
        \end{pmatrix}
    \end{align*}
    より, 固有値は, $\bm{v}$が単位ベクトルなことに注意して,
    \begin{align*}
        0 = \lambda^2 - |\bm{v}|^2 = \lambda^2 - 1
        \to \lambda = \pm1.
    \end{align*}
    対応する固有ベクトルは,
    \begin{align*}
        \ket{\lambda=\pm1} =
        \frac{1}{\sqrt{2(1\mp v_3)}}
        \begin{pmatrix}
            -v_1 + i v_2 \\ v_3 \mp 1.
        \end{pmatrix}
    \end{align*}
    よって, 射影オペレータは,
    \begin{align*}
        P_{\pm}
         & = \ket{\lambda = \pm1} \bra{\lambda = \pm1} \\
         & =
        \frac{1}{2(1\mp v_3)}
        \begin{pmatrix}
            -v_1 + i v_2 \\ v_3 \mp 1
        \end{pmatrix}
        \begin{pmatrix}
            v_1 - i v_2 & v_3 \mp 1
        \end{pmatrix}                    \\
         & =
        \frac{1}{2(1\mp v_3)}
        \begin{pmatrix}
            v_1^2 + v_2^2            & (-v_1 + i v_2)(v_3 \mp1) \\
            (-v_1 + i v_2)(v_3 \mp1) & (v_3 \mp 1)^2
        \end{pmatrix}                    \\
         & =
        \frac{1}{2(1\mp v_3)}
        \begin{pmatrix}
            1 - v_3^2                & (-v_1 + i v_2)(v_3 \mp1) \\
            (-v_1 + i v_2)(v_3 \mp1) & (v_3 \mp 1)^2
        \end{pmatrix}                    \\
         & =
        \frac{1}{2}
        \begin{pmatrix}
            1\pm v_3         & \pm(v_1 - i v_2) \\
            \pm(v_1 - i v_2) & 1\mp v_3
        \end{pmatrix}                    \\
         & =
        \frac{I \pm \bm{v} \cdot \bm{\sigma}}{2}.
    \end{align*}
    \par
    一方, $v_3 = 1$のとき, $v_1 = v_2 = 0$となり,
    \begin{align*}
        \bm{v} \cdot \bm{\sigma}
        =
        \begin{pmatrix}
            1 & 0   \\
            0 & - 1
        \end{pmatrix}
        =
        Z.
    \end{align*}
    固有ベクトルは,
    \begin{align*}
        \ket{\lambda = 1} = \ket{0},
        \ket{\lambda= -1} = \ket{1}.
    \end{align*}
    よって, 射影オペレータは,
    \begin{align*}
        P_{+} = \ket{0}\bra{0},
        P_{-} = \ket{1}\bra{1}
        \to
        P_{\pm} = \frac{I \pm \bm{v} \cdot \bm{\sigma}}{2}.
    \end{align*}
\end{ex}

\begin{ex}
    \label{ex2.61}
    $v_3\neq 1$のとき,
    $\ket{0}$の状態を測定して$+1$を得る確率は,
    \begin{align*}
        \left|
        \braket{\lambda = +1 | \bm{v} \cdot \bm{\sigma} | 0}
        \right|^2
        =
        \left|
        \braket{\lambda = +1 | 0}
        \right|^2
        =
        \left|
        \frac{-v_1 + iv_2}{\sqrt{2(1-v_3)}}
        \right|^2
        =
        \frac{1+v_3}{2}.
    \end{align*}
    これは, $v_3 = 1$でも成立する.
    $+1$を測定した直後の状態は,
    \begin{align*}
        \frac{P_+ \ket{0}}{\sqrt{\braket{0 | P_+^\dagger P_+ | 0}}}
        =
        \frac{\ket{\lambda = +1}\braket{\lambda = +1|0}}{\sqrt{\braket{\lambda = +1 | 0}\braket{0 | \lambda = +1}}}
        =
        e^{i\theta} \ket{\lambda=+1}.
    \end{align*}
    ここで, $\theta$は$\braket{\lambda = +1 | 0}$の偏角.
\end{ex}

\begin{ex}
    \label{ex2.62}
    測定オペレータがPOVMと一致するとすると,
    \begin{align*}
        M_m = M_m^\dagger M_m .
    \end{align*}
    $M_m^\dagger M_m$は正のオペレータなので, $M_m$も正のオペレータ.
    よって, 演習\ref{ex2.24}より$M_m$もHermite. したがって,
    \begin{align*}
        M_m = M_m^\dagger M_m = M_m^2
    \end{align*}
    となり, $M_m$は射影オペレータ.
    \\
    また, 完全性条件
    \begin{align*}
        \sum_m M^\dagger_m M_m =  \sum_m M_m = I
    \end{align*}
    から,
    \begin{align*}
        M_m
         & = \sum_{m'} M_{m'} M_m
        = M_m^2 + \sum_{m'\neq m} M_{m'} M_m
        = M_m + \sum_{m'\neq m} M_{m'} M_m
        \\
         & \to  \sum_{m'\neq m} M_{m'} M_m = O
    \end{align*}
    となり, $M_{m'} M_m (m'\neq m)$は正のオペレータゆえ,
    \begin{align*}
        M_{m'} M_m  = O \ (m'\neq m).
    \end{align*}
    先に示した$M_m^2 = M_m$と合わせて,
    \begin{align*}
        M_{m'} M_m  = \delta_{m m'} M_m.
    \end{align*}
    こうして, 測定オペレータがPOVMと一致するとすると, $M_m$が直交射影オペレータになることが言えた.
\end{ex}

\begin{ex}
    \label{ex2.63}
    定理2.3より, 明らか.
\end{ex}

\begin{ex}
    \label{ex2.64}
    \begin{align*}
        \braket{\psi_i|\phi_j} \propto \delta_{ij}
    \end{align*}
    を満たすような$\ket{\psi_i}(\forall_{i \neq j})$と直交する$\ket{\phi_j}(\neq0)$を作りたい. $\{ \ket{\psi_i}\}$で張られる$m$次元ベクトル空間$V$の部分空間$W_j$は$\{ \ket{\psi_i} \}_{i\neq j}$で張られるとして, その直交補空間$W_j^\perp$とする. $W_j^\perp$への射影オペレータ$P_j^\perp$としてやれば, 作りたかった$\ket{\phi_j}$を
    \begin{align*}
        \ket{\phi_j} = P_j^\perp \ket{\psi_j}
    \end{align*}
    で作れる. $\{ \ket{\psi_i}\}$の線型独立性から, $\ket{\phi_j} \neq 0$が保証される.
    \par
    こうして作った$\{ \ket{\phi_j}\}$を使って,
    \begin{align*}
        E_j
        =
        \begin{cases}
            c \ket{\phi_i} \bra{\phi_i} & (j=1,2, ...,m) \\
            I - \sum_{k=1}^{k=m} E_k    & (j=m+1)
        \end{cases}
    \end{align*}
    なる$\{ E_i\}$を作る. ここで, $c$は任意の状態$\ket{\psi}$に対して,
    \begin{align*}
        c < \frac{1}{\sum_{k=1}^{k=m} \braket{\psi| E_k |\psi}}
    \end{align*}
    を満たす定数. このように定義された$\{ E_i\}$が求めたかったPOVM(正かつ$\sum_{k=1}^{m+1} E_k = I$を満たす)である.
\end{ex}

\begin{ex}
    基底を
    \begin{align*}
        \left\{ \frac{\ket{0}+\ket{1}}{\sqrt{2}},  \frac{\ket{0}-\ket{1}}{\sqrt{2}}\right\}
    \end{align*}
    ととれば良い.
\end{ex}

\begin{ex}
    \begin{align*}
        \frac{\bra{0}\otimes\bra{0} + \bra{1}\otimes \bra{1}}{\sqrt{2}}
        X_1 \otimes Z_2
        \frac{\ket{0} \otimes \ket{0} + \ket{1} \otimes \ket{1}}{\sqrt{2}}
        =
        \frac{\bra{0}\otimes\bra{0} + \bra{1}\otimes \bra{1}}{\sqrt{2}}
        \frac{\ket{1} \otimes \ket{0} - \ket{0} \otimes \ket{1}}{\sqrt{2}}
        =0.
    \end{align*}
\end{ex}

\begin{ex}
    $W$の直交補空間$W^{\perp}$とかき, それぞれの正規直交基底を$\{ \ket{w_i}\}, \{ \ket{w^{\perp}_j}\}$とし,
    \begin{align*}
        U'
        =
        \sum_{i} U\ket{w_i} \bra{w_i}
        +
        \sum_{j} \ket{w^\perp_j} \bra{w^\perp_j}
    \end{align*}
    とすればよい.
\end{ex}

\begin{ex}
    \label{ex2.68}
    あらゆる単一qビット$\ket{a},\ket{b}$は,
    \begin{align*}
        \ket{a} = a_0 \ket{0} + a_1 \ket{1}
        \\
        \ket{b} = b_0 \ket{0} + b_1 \ket{1}
    \end{align*}
    でかける.
    \begin{align*}
        \ket{\psi} = \frac{\ket{00} + \ket{11}}{\sqrt{2}}
    \end{align*}
    に対して,
    \begin{align*}
        \ket{\psi}
        =
        \ket{a} \ket{b}
    \end{align*}
    とかけると仮定すると,
    \begin{align*}
        a_0 b_1 = a_1 b_0 = 0, a_0 b_0 \neq 0, a_1 b_1 \neq 0
    \end{align*}
    となるが, これを満たす$a_0, a_1, b_0, b_1$は存在せず矛盾.
\end{ex}

\begin{ex}
    \label{ex2.69}
    Bell基底は,
    \begin{align*}
        \ket{\beta_{xy}} = \frac{\ket{0,y}+ (-1)^x\ket{1,\bar{y}}}{\sqrt{2}}
    \end{align*}
    でかけ,
    \begin{align*}
        \braket{\beta_{x'y'}|\beta_{xy}} = \frac{\delta_{yy'}+ (-1)^{x+x'}\delta_{\bar{y}\bar{y'}}}{2} = \delta_{xx'}\delta_{yy'}.
    \end{align*}
    となるので, 正規直交基底.
\end{ex}

\begin{ex}
    \label{ex2.70}
    Bell状態
    \begin{align*}
        \ket{\beta_{xy}} = \frac{\ket{0,y}+ (-1)^x\ket{1,\bar{y}}}{\sqrt{2}}
    \end{align*}
    に対して, $\braket{\beta_{xy}|E \otimes I| \beta{xy}}$は,
    \begin{align*}
        \braket{\beta_{xy}|E \otimes I| \beta{xy}}
        =
        \frac{\braket{0|E|0} + \braket{1|E|1}}{2}
    \end{align*}
    と$x,y$に依らない. 上に示したことから, AliceがBobに送ったqビットを, Eveが観測しても得られる状態の期待値はBell状態に依らないので, EveはAliceの送りたかった情報を推論できない.
\end{ex}

\begin{ex}
    \label{ex2.71}
    密度オペレータ$\rho$は, 正のオペレータなので, スペクトル定理から, $\rho$の固有空間の正規直交基底$\{ \ket{\psi_i}\}$を用いて, 
    \begin{align*}
        \rho = \sum_i p_i \ket{\psi_i}\bra{\psi_i}.
    \end{align*}
    とかけるので,
    \begin{align*}
        \tr{\left(\rho^2 \right)}
         & =
        \tr
        \left(
        \sum_{i,j} p_i p_j \ket{\psi_i}\braket{\psi_i|\psi_j}\bra{\psi_j}
        \right)                                                  \\
         & =
        \sum_{i,j} p_i p_j
        \tr
        \left(
        \ket{\psi_i}\braket{\psi_i|\psi_j}\bra{\psi_j}
        \right)                                                  \\
         & =
        \sum_{i,j} p_i p_j
        \delta_{ij}
        \tr
        \left(
        \ket{\psi_i}\bra{\psi_j}
        \right)                                                  \\
         & =
        \sum_{i} p_i ^2
        \tr
        \left(
        \braket{\psi_i|{\psi_i}}
        \right)                                                  \\
         & =\sum_{i} p_i ^2 \le \left(\sum_{i} p_i\right)^2 = 1.
    \end{align*}
    等号は,
    \begin{align*}
        p_i =
        \begin{cases}
            1 & (i=i_0)              \\
            0 & (\mathrm{otherwise})
        \end{cases}
    \end{align*}
    つまり, $\rho$が純粋状態の時にのみ成り立つ.
\end{ex}

\begin{ex}
    \label{ex2.71}
    (1)\
    \begin{align*}
        \rho^2
         & =
        \frac{I + \bm{r}\cdot\bm{\sigma}}{2}
        \frac{I + \bm{r}\cdot\bm{\sigma}}{2} \\
         & =
        \frac{1}{4}
        \left(
        I + 2 r_i \sigma_i + r_i r_j \sigma^i \sigma^j
        \right)                              \\
         & =
        \frac{1}{4}
        \left(
        I + 2 r_i \sigma_i + r_i r_j \frac{\{\sigma^i, \sigma^j\}}{2}
        \right)                              \\
         & =
        \frac{1}{4}
        \left(
        (1+\bm{r}^2)I + 2 \bm{r} \cdot \bm{\sigma}
        \right)                              \\
         & =
        \frac{\bm{r}^2 - 1}{4}I + \rho
    \end{align*}
    で$\rho$は,
    \begin{align*}
        \rho = \rho^\dagger
    \end{align*}
    を満たすので, 任意の状態$\ket{\psi}$に対して, $\bm{r}^2 \le 1$なら,
    \begin{align*}
        \braket{\psi | \rho | \psi}
        =
        \braket{\psi | \rho^2 + \frac{1-\bm{r}^2}{4} | \psi}
        =
        \braket{\psi | \rho^\dagger \rho | \psi}
        +
        \braket{\psi | \frac{1-\bm{r}^2}{4} | \psi}
        \ge 0.
    \end{align*}
    また, $\tr{(\sigma_i)} = 0$なので,
    \begin{align*}
        \tr(\rho) = 1.
    \end{align*}
    以上より, $\rho$は, 正かつトレースが1なので, 定理2.5より$\rho$は密度オペレータ.
    \par
    (2)\ $\bm{r} = \bm{0}$.
    \par
    (3)\
    \begin{align*}
        \rho \mathrm{が純粋状態}
        \Leftrightarrow
        \tr(\rho^2) = 1
        \Leftrightarrow
        \tr\left(
        \frac{1-\bm{r}^2}{4}+\rho
        \right)
        \Leftrightarrow
        |\bm{r}| =1.
    \end{align*}
    \par
    (4)
    $\bm{r} = (\sin\theta \cos\phi, \sin\theta \sin\phi, \cos\theta)$として,
    \begin{align*}
        \rho
        =
        \frac{I + \bm{r}\cdot\bm{\sigma}}{2}
        =
        \frac{1}{2}
        \begin{pmatrix}
            1 + \cos\theta        & \sin\theta e^{-i\phi} \\
            \sin\theta e^{i \phi} & 1 - \cos\theta
        \end{pmatrix}
        =
        \ket{\psi} \bra{\psi}.
    \end{align*}
    ただし, ここで
    \begin{align*}
        \ket{\psi} = {\cos\frac{\theta}{2}}\ket{0}+e^{i\phi}{\sin\frac{\theta}{2}}\ket{1}.
    \end{align*}
\end{ex}

\begin{ex}
    \begin{align*}
        \rho = \sum_{i} p_i \ket{\psi_i} \bra{\psi_i}
    \end{align*}
    の左から$\rho^{-1}$をかけて,
    \begin{align*}
        1 = \sum_{i} p_i \rho^{-1} \ket{\psi_i} \bra{\psi_i}.
    \end{align*}
    左から$\bra{\psi_j}$, 右から$\ket{\psi_j}$をかけて,
    \begin{align*}
        1 = p_j \braket{\psi_j|\rho^{-1}|\psi_j} \to p_j = \frac{1}{\braket{\psi_j|\rho^{-1}|\psi_j}}.
    \end{align*}
\end{ex}

\begin{ex}
    \label{ex2.74}
    複合システムの密度オペレータ$\rho_{AB}$は,
    \begin{align*}
        \rho_{AB} = \ket{a} \bra{a} + \ket{b} \bra{b}.
    \end{align*}
    よって, システム$A$で縮約した密度オペレータは,
    \begin{align*}
        \rho^A = \ket{a} \bra{a} \tr\left(\ket{b} \bra{b}\right) = \ket{a} \bra{a}
    \end{align*}
    なので,
    \begin{align*}
        tr\left({\rho^{A}}^2\right) = 1
    \end{align*}
    と$\rho^A$が純粋状態であることがわかる.
\end{ex}

\begin{ex}
    \label{ex2.75}
    各Bell状態
    \begin{align*}
        \ket{\beta_{xy}} = \frac{\ket{0,y}+ (-1)^x\ket{1,\bar{y}}}{\sqrt{2}}
    \end{align*}
    に対して, 密度オペレータは,
    \begin{align*}
        \rho
         & =
        \frac{\ket{0,y}+ (-1)^x\ket{1,\bar{y}}}{\sqrt{2}}
        \frac{\bra{0,y}+ (-1)^x\bra{1,\bar{y}}}{\sqrt{2}} \\
         & =
        \frac{
            \ket{0,y}\bra{0,y} + (-1)^x\ket{1,\bar{y}}\bra{0,y}
            +
            (-1)^x\ket{0,y}\bra{1,\bar{y}} + \ket{1,\bar{y}}\bra{1,\bar{y}}
        }{2}.
    \end{align*}
    よって, 各qビットを縮約した密度オペレータは,
    \begin{align*}
        \rho^1 & = \frac{\ket{y}\bra{y}  + \ket{\bar{y}}\bra{\bar{y}} }{2} = \frac{I}{2} \\
        \rho^2 & = \frac{ \ket{0}\bra{0} +  \ket{1}\bra{1}}{2} = \frac{I}{2}.
    \end{align*}
\end{ex}

\begin{ex}
    \label{ex2.76}
    $m\times n$の行列$C$に対して, $m \times m$のユニタリ行列$U$, $n\times n$のユニタリ行列$V$, $m \times n$の行列$D$;
    D =
    \begin{align*}
        \begin{pmatrix}
            \lambda_1 &           &        &                          &                                        \\
                      & \lambda_2 &        &                          &                                        \\
                      &           & \ddots &                          &                                        \\
                      &           &        & \lambda_{\mathrm{rank}A} &                                        \\
                      &           &        &                          & O_{m-\mathrm{rank}A, n-\mathrm{rank}A} \\
        \end{pmatrix}
    \end{align*}
    を用いて,
    \begin{align*}
        C = UDV
    \end{align*}
    とかける(特異値分解). ここで, $\lambda_i$は$C$の特異値で非負である.
    \par
    $A \otimes B$の任意の純粋状態$\ket{\psi}$に対して,
    $A,B$の正規直交基底$\ket{j}, \ket{k}$を用いて,
    \begin{align*}
        \ket{ \psi} = \sum_{jk} C_{jk} \ket{j} \ket{k}.
    \end{align*}
    $C$の特異値分解を考えると, ユニタリ行列$U,V$が存在し,
    \begin{align*}
        \ket{ \psi} = \sum_{ijk} U_{ji} D_{ii} V_{ik}\ket{j} \ket{k}.
    \end{align*}
    ここで,
    \begin{align*}
        \ket{i_A} = \sum_j U_{ji} \ket{j}, \ket{i_B} = \sum_k V_{ik} \ket{k}
    \end{align*}
    とすれば, $\ket{i_A}, \ket{i_B}$は$A,B$の正規直交基底なので, $\ket{i_A}\ket{i_B}$
    は$A \otimes B$の正規直交基底で,
    \begin{align*}
        \ket{ \psi} = \sum_{i} D_{ii} \ket{i_A}\ket{i_B}.
    \end{align*}
    ただし, $\tr\left( \ket{\psi} \bra{\psi}\right) = 1$より,
    \begin{align*}
        \sum_{i} D_i^2 = 1
    \end{align*}
    を満たす.
\end{ex}

\begin{ex}
    \label{ex2.77}
    例えば, $\mathbb{C}^2 \otimes \mathbb{C}^2 \otimes \mathbb{C}^2$の状態$\ket{\psi}$;
    \begin{align*}
        \ket{\psi}= \ket{0} \left( {\ket{00} + \ket{11}}\right)
    \end{align*}
    は,
    \begin{align*}
        \ket{\psi}
        =
        \lambda_0 \ket{0_{A}{0_{B}}{0_{C}}}
        +
        \lambda_1 \ket{1_{A}{1_{B}}{1_{C}} } \\
        \braket{i_{A} | j_{A}} = \braket{i_{B} | j_{B}} = \braket{i_{C} | j_{C}} = \delta_{ij} \
    \end{align*}
    の形でかけない. なぜなら, $\ket{\psi}$でかっこでくくり出された$\ket{0}$のせいである.
\end{ex}

\begin{ex}
    \label{ex2.78}
    $\ket{\psi}$のSchmidt数が1なら, $A,B$の正規直交基底$\ket{i_A}, \ket{i_B}$として, $\ket{\psi}$は$\ket{ \psi } \propto \ket{i_A} \ket{i_B}$と積状態.
    逆に, $\ket{ \psi } = \ket{a} \ket{b}$なら, $\ket{a}, \ket{b}$のそれぞれが$A,B$の正規直交基底になるような基底の変換$U_A,U_B$が存在する(Schmidtの直交化法).
    \par
    $\rho^A, \rho^B$が純粋状態とすると, $\rho^{AB} = \ket{a}\ket{b} \bra{a}\bra{b}$. これは, $\ket{\psi}=\ket{a}\ket{b}$が積状態であることを意味している. 逆に, $\ket{\psi}$が積状態であるとすると, $\ket{\psi}$とかけ, $\rho^A = \ket{a} \bra{a}, \rho^B = \ket{b} \bra{b}$と$\rho^A, \rho^B$が純粋状態であることを意味している.
\end{ex}

\begin{ex}
    \label{ex2.79}
    (1)\
    \begin{align*}
        \frac{\ket{00} + \ket{11}}{\sqrt{2}}
    \end{align*}
    はそれ自身でSchmidt分解.
    \par
    (2)\
    \begin{align*}
        \frac{\ket{00}+\ket{01}+\ket{10}+\ket{11}}{2} = \frac{\ket{0} + \ket{1}}{\sqrt{2}} \frac{\ket{0} + \ket{1}}{\sqrt{2}} + 0 \frac{\ket{0} - \ket{1}}{\sqrt{2}} \frac{\ket{0} - \ket{1}}{\sqrt{2}} .
    \end{align*}
    (3)\
    \begin{align*}
        \frac{1}{\sqrt{3}}
        \begin{pmatrix}
            1 & 1 \\
            1 & 0
        \end{pmatrix}
    \end{align*}
    %
    %
    %
    %
    %
    %
    %
    の特異値分解を演習\ref{ex2.50}を参考にしてやって, 定理2.7の証明に従って頑張る.
\end{ex}

\begin{ex}
    \label{ex2.80}
    \begin{align*}
        \ket{\psi} = \sum_i \lambda_i \ket{i_A}\ket{i_B}                  \\
        \ket{\phi} = \sum_i \lambda_i \ket{\tilde{i}_A} \ket{\tilde{i}_B} \\
    \end{align*}
    に対して,
    \begin{align*}
        U = \sum_j \ket{j_A} \bra{\tilde{j}_A} \\
        V= \sum_j \ket{j_B} \bra{\tilde{j}_B}  \\
    \end{align*}
    とすれば, $U,V$はユニタリで,
    \begin{align*}
        \ket{\psi} = (U \otimes V)\ket{\phi}
    \end{align*}
    を満たす.
\end{ex}

\begin{ex}
    \label{ex2.81}
    \begin{align*}
        \rho^A = \sum_i p_i \ket{i^A} \bra{i^A}
    \end{align*}
    の純粋化
    \begin{align*}
        \ket{AR_1} = \sum_i \sqrt{p_i} \ket{i^A} \ket{i^{R1}} \\
        \ket{AR_2} = \sum_i \sqrt{p_i} \ket{i^A} \ket{i^{R2}}
    \end{align*}
    に対して,
    \begin{align*}
        U_R = \sum_i \ket{i^{R1}} \bra{i^{R2}}
    \end{align*}
    とすれば, $U_R$はユニタリで,
    \begin{align*}
        \ket{A R_1} = (I_A \otimes U_R) \ket{A R_2}
    \end{align*}
    を満たす.
\end{ex}

\begin{ex}
    \label{ex2.82}
    \begin{align*}
        \rho^A = \sum_i p_i \ket{\psi_i} \bra{\psi_i}
    \end{align*}
    とする. \par
    (1)\
    (2.211)式と同様.
    \par
    (2)\
    $R$を測定して, 結果$i$を得る確率は,
    \begin{align*}
        R_i = I \otimes \ket{i} \bra{i}
    \end{align*}
    として,
    \begin{align*}
        \tr(R_i^\dagger R_i \rho^{AR}) = \tr(R_i \rho^{AR}).
    \end{align*}
    結果$i$を得た後の状態は,
    \begin{align*}
        \frac{R_i \rho^{AR} R_i}{\tr (R_i \rho^{AR})}.
    \end{align*}
    (3)\
    システム$AR$に対する$\rho^A$の任意の純粋化$\ket{AR}$とは, システム$R$上の正規直交基底を任意にとって行った純粋化の意味.
    システム$A$の正規直交基底は,
    システム$AR$に対する$\rho^A$の任意の純粋化$\ket{AR}$は, $R$上のユニタリオペレータ$U_R$として,
    \begin{align*}
        \ket{AR} = \sum_i \sqrt{p_i} \ket{\psi_i} \otimes U_R \ket{i}.
    \end{align*}
    純粋状態$\ket{AR} \bra{AR}$に対して, $R$を測定して$i$を得る確率は,
    \begin{align*}
        R_i = I \otimes U_R \ket{i} \bra{i} U_R^\dagger
    \end{align*}
    として,
    \begin{align*}
        \tr\left[R_i^\dagger R_i ( \ket{AR} \bra{AR})\right]
         & =
        \sum_{j,k} \sqrt{p_j} \sqrt{p_k}
        \tr
        \left[
            \big(
            I \otimes U_R \ket{i}\bra{i} U_R^\dagger
            \big)
            \big(
            \ket{\psi_j} \otimes U_R\ket{j}
            \big)
            \big(
            \bra{\psi_k} \otimes \bra{k} U_R^\dagger
            \big)
            \right]
        \\
         & =
        \sum_{k} \sqrt{p_i} \sqrt{p_k}
        \tr
        \left[
            \big(
            \ket{\psi_i} \otimes U_R\ket{i}
            \big)
            \big(
            \bra{\psi_k} \otimes \bra{k} U_R^\dagger
            \big)
            \right]
        \\
         & =
        \sum_{k} \sqrt{p_i} \sqrt{p_k} \braket{\psi_k | \psi_i} \braket{i|k} \\
         & =
        p_i.
    \end{align*}
    $i$を得た後の状態は,
    \begin{align*}
        \frac{R_i \rho^{AR} R_i}{p_i}=
        \left(
        \ket{\psi_i} \otimes \ket{i}
        \right)
        \left(
        \bra{\psi_i} \otimes \bra{i}
        \right).
    \end{align*}
    以上より, 任意の$\rho^A$の任意の純粋化に対して,
    $R$を測定すると確率$p_i$で, システム$A$の測定後の状態が$\ket{\psi_i}$となる
    $R$の正規直交基底$\{ U_R \ket{i} \}$が存在することが言えた.
\end{ex}

\begin{problem}
\label{problem2.1}
演習\ref{ex2.35}で見たように, 単位ベクトル$\bm{n}$に対して,
\begin{align*}
    \left(\bm{n} \cdot \bm{\sigma} \right)^n =
    \begin{cases}
        \bm{n} \cdot \bm{\sigma} & (n: \mathrm{even} ) \\
        I                        & (n: \mathrm{odd} )
    \end{cases}.
\end{align*}
よって, Taylorの定理より,
\begin{align*}
    f(\theta \bm{n} \cdot \bm{\sigma} )
     & =
    I \sum_{n: \mathrm{even}} \frac{f^{(n)}(0)}{n!} \theta^n
    +
    \bm{n} \cdot \bm{\sigma} \sum_{n: \mathrm{odd}} \frac{f^{(n)}(0)}{n!} \theta^n
    \\
     & =
    I \sum_{n: \mathrm{even}} \frac{f^{(n)}(0)}{n!} \frac{\theta^n + (-\theta)^n}{2}
    +
    \bm{n} \cdot \bm{\sigma} \sum_{n: \mathrm{odd}} \frac{f^{(n)}(0)}{n!} \frac{\theta^n - (-\theta)^n}{2}
    \\
     & =
    I \sum_{n} \frac{f^{(n)}(0)}{n!} \frac{\theta^n + (-\theta)^n}{2}
    +
    \bm{n} \cdot \bm{\sigma} \sum_{n} \frac{f^{(n)}(0)}{n!} \frac{\theta^n - (-\theta)^n}{2}
    \\
     & =
    \frac{f(\theta) + f(-\theta)}{2} I + \frac{f(\theta) - f(-\theta)}{2}\bm{n} \cdot \bm{\sigma}.
\end{align*}
\end{problem}

\begin{problem}
\label{problem2.2}
(1)\
$A,B$のある正規直交基底$\{\ket{i^A}\},\{ \ket{i^B} \}$を用いて, $A \otimes B$の純粋状態$\ket{\psi}$は, $\lambda_i > 0$として,
\begin{align*}
    \ket{\psi} = \sum_{i=1}^{\mathrm{Sch}(\psi)} \lambda_i \ket{i^A} \ket{i^B}
\end{align*}
とSchmidt分解可能. すると,
\begin{align*}
    \rho^A = \tr_B(\ket{\psi} \bra{\psi}) = \sum_{i=1}^{\mathrm{Sch}(\psi)} \lambda_i^2 \ket{i^A} \bra{i^A}
\end{align*}
となり, これは$\rho^A$のスペクトル分解にもなっていて, $\rho^A$の台の次元の数($\mathrm{rank}\rho^A$)が$\mathrm{Sch}(\psi)$に等しいことを意味している.
\par
(2)\
\begin{align*}
    \ket{\psi} = \sum_{i = 1}^{n} \ket{\alpha_i}\ket{\beta_i}
\end{align*}
とかけているとき$\{ \ket{\alpha_i} \}$は互いに一次独立になっているとする. すると, $\{ \ket{\alpha_i} \}$で張られる次元$n$のベクトル空間$V$が定義できる. ここで, (1)の結果を用いると,
\begin{align*}
    \mathrm{Sch}(\psi)
    =  \mathrm{rank} \left( \tr_B(\ket{\psi} \bra{\psi}) \right)
    =  \mathrm{rank} \left( \sum_{j,k = 1}^{n} \ket{\alpha_j}\braket{\beta_k|\beta_j} \bra{\alpha_k}\right)
    \leq n.
\end{align*}
最後の不等式では, $\left( \sum_{j,k = 1}^{n} \ket{\alpha_j}\braket{\beta_k|\beta_j} \bra{\alpha_k}\right)$が, $V$上で定義されたある線形オペレータの行列表示(サイズは$n\times n$)になっていることを用いた.
\par
(3)\
\begin{align*}
    \mathrm{Sch}(\phi) \geq \mathrm{Sch}(\gamma)
\end{align*}
としても一般性を失わない.
$\ket{\phi}, \ket{\gamma}$のSchmidt分解は,
\begin{align*}
    \ket{\phi} = \sum_{i=1}^{\mathrm{Sch}(\phi)} \lambda_i \ket{i^A} \ket{i^B} \\
    \ket{\gamma} = \sum_{i=1}^{\mathrm{Sch}(\gamma)} \xi_i \ket{\tilde{i}^A} \ket{\tilde{i}^B} .
\end{align*}
$\alpha, \beta$の一方が0の時は, 明らか
に示すべき式が成立するので,
$\alpha \neq 0, \beta \neq = 0$として,
\begin{align*}
    \ket{\psi} = \alpha \ket{\phi} + \beta \ket{\gamma}
\end{align*}
のとき,
(1)より,
\begin{align*}
    \mathrm{Sch}(\psi)
     & =
    \mathrm{rank} \left( \tr_B(\ket{\psi} \bra{\psi}) \right)
    \\
     & =
    \mathrm{rank}
    \left(
    \alpha \sum_{i=1}^{\mathrm{Sch}(\phi)} \lambda_i^2 \ket{i^A} \bra{i^A}
    +
    \beta \sum_{i=1}^{\mathrm{Sch}(\gamma)} \xi_i^2 \ket{\tilde{i}^A} \bra{\tilde{i}^A}
    \right)
\end{align*}
この右辺の最小値は,
\begin{align*}
    \alpha \sum_{i=1}^{\mathrm{Sch}(\phi)} \lambda_i^2 \ket{i^A} \bra{i^A}
    +
    \beta \sum_{i=1}^{\mathrm{Sch}(\gamma)} \xi_i^2 \ket{\tilde{i}^A} \bra{\tilde{i}^A}
     & =
    \sum_{i=1}^{\mathrm{Sch}(\phi) - \mathrm{Sch}(\gamma)} \alpha \lambda_i^2 \ket{i^A} \bra{i^A}
    +
    \sum_{i=1}^{\mathrm{Sch}(\gamma)} ( \alpha \lambda_i^2 + \beta \xi_i^2) \ket{\tilde{i}^A} \bra{\tilde{i}^A}
\end{align*}
で,
\begin{align*}
    \alpha \lambda_i^2 + \beta \xi_i^2 = 0 \ (i=1, 2, \dots \mathrm{Sch}(\gamma))
\end{align*}
のときに実現される. つまり,
\begin{align*}
    \min \mathrm{rank}
    \left(
    \alpha \sum_{i=1}^{\mathrm{Sch}(\phi)} \lambda_i^2 \ket{i^A} \bra{i^A}
    +
    \beta \sum_{i=1}^{\mathrm{Sch}(\gamma)} \xi_i^2 \ket{\tilde{i}^A} \bra{\tilde{i}^A}
    \right)
    =
    \sum_{i=1}^{\mathrm{Sch}(\phi) - \mathrm{Sch}(\gamma)} \alpha \lambda_i^2 \ket{i^A} \bra{i^A}
    =
    \mathrm{Sch}(\phi) - \mathrm{Sch}(\gamma).
\end{align*}
よって,
\begin{align*}
    \mathrm{Sch}(\psi) =
    \mathrm{rank}
    \left(
    \alpha \sum_{i=1}^{\mathrm{Sch}(\phi)} \lambda_i^2 \ket{i^A} \bra{i^A}
    +
    \beta \sum_{i=1}^{\mathrm{Sch}(\gamma)} \xi_i^2 \ket{\tilde{i}^A} \bra{\tilde{i}^A}
    \right)
    \geq  \mathrm{Sch}(\phi) - \mathrm{Sch}(\gamma).
\end{align*}
\end{problem}

\begin{problem}
\label{problem2.3}
演習\ref{ex2.35}で見たように,
\begin{align*}
    Q^2 = R^2 = S^2 = T^2 = I.
\end{align*}
このことと, $Q,R,S,T$が互いに非可換なことに注意して,
\begin{align*}
    \left(
    Q \otimes S + R \otimes S + R \otimes T - Q \otimes T
    \right)^2
    =
    4I + [Q,R] \otimes [S,T].
\end{align*}
$Q,R,S,T$がHermiteなので, 式(2,107)より,
\begin{align*}
    \left| \ \left< [Q,R]\right> \ \right|
     & \leq 2 \left| \left< Q^2 \right> \right| \left| \left< R^2\right> \right|  = 2 \\
    \left| \ \left< [S,T]\right> \ \right|
     & \leq 2 \left| \left< S^2 \right> \right| \left| \left< T^2\right> \right|  = 2
\end{align*}
が成り立つことと, 先に示した等式を合わせて,
\begin{align*}
    \left<
    Q \otimes S + R \otimes S + R \otimes T - Q \otimes T
    \right>^2
     & \leq
    \left<\
    \left(
    Q \otimes S + R \otimes S + R \otimes T - Q \otimes T
    \right)^2\
    \right>
    \\
     &
    =
    \left<\
    4I + [Q,R] \otimes [S,T]
    \
    \right>
    \leq 8.
\end{align*}
よって,
\begin{align*}
    \left<
    Q \otimes S + R \otimes S + R \otimes T - Q \otimes T
    \right>
    \leq 2 \sqrt{2}.
\end{align*}
\end{problem}