\begin{ex}
    \label{ex2.51}
    \begin{align*}
        H H^\dagger = I.
    \end{align*}
\end{ex}

\begin{ex}
    \label{ex2.52}
    \begin{align*}
        H H = I.
    \end{align*}
\end{ex}

\begin{ex}
    \label{ex2.53}
    \begin{align*}
        \ket{\lambda = \pm 1}
        =
        \begin{pmatrix}
            1 \\ \pm \sqrt{2} - 1
        \end{pmatrix}
    \end{align*}
\end{ex}

\begin{ex}
    \label{ex2.54}
    $A,B$は可換なHermiteなので, 同じ正規直交基底$\{ \ket{i}\}$で同時対角化可能で,
    \begin{align*}
        A = \sum_i a_i \ket{i} \bra{i},
        B = \sum_i b_i \ket{i} \bra{i}
    \end{align*}
    とかけるので,
    \begin{align*}
        \exp(A) \exp(B)
        =
        \sum_i \sum_j e^{a_i} \ket{i} \bra{i}
        e^{b_j} \ket{j} \bra{j}
        =
        \sum_i e^{a_i+b_i} \ket{i} \bra{i}
        =
        \exp(A+B).
    \end{align*}
\end{ex}

\begin{ex}
    \label{ex2.55}
    $H$はHermiteなので, $H,H^\dagger$が可換だから,
    演習\ref{ex2.54}より,
    \begin{align*}
        U(t_1,t_2)U^\dagger(t_1,t_2)
        =
        \exp{\left[ - iH(t_1 - t_2)\right]}
        \exp{\left[ iH^\dagger(t_1 - t_2)\right]}
        =
        \exp{\left[ i(H^\dagger - H)(t_1 - t_2)\right]}
        =
        I.
    \end{align*}
\end{ex}

\begin{ex}
    \label{ex2.56}
    ユニタリオペレータ$U$は正規なので, $\lambda_i=e^{i\theta_i} (\theta_i \in R)$として,
    \begin{align*}
        U = \sum_i \lambda_i \ket{\lambda_i} \bra{\lambda_i}
    \end{align*}
    とスペクトル分解できる. よって,
    \begin{align*}
        K = - i \log{U} = \sum_i \theta_i \ket{i} \bra{i}
    \end{align*}
    となり, これは明らかにHermite. したがって,
    \begin{align*}
        \exp(iK) = \sum_i e^{i\theta_i} \ket{i} \bra{i} = U.
    \end{align*}
\end{ex}

\begin{ex}
    \label{ex2.57}
    状態$\ket{\psi}$に対して, $L_l$を測定した後の状態$\ket{\phi}$は,
    \begin{align*}
        \ket{\phi} = \frac{L_l \ket{\psi}}{\braket{\psi|L_l^\dagger L_l| \psi}}.
    \end{align*}
    さらに, この状態に対して, $M_m$を測定した後の状態は,
    \begin{align*}
        \frac{M_m \ket{\phi}}{\braket{\phi|M_m^\dagger M_m| \phi}}
        =
        \frac{M_m L_l \ket{\psi}}{\braket{\psi| L_l^\dagger M_m^\dagger M_m L_l| \psi}}
        =
        \frac{N_{lm} \ket{\psi}}{\braket{\psi|  N_{lm}^\dagger N_{lm}| \psi}}.
    \end{align*}
\end{ex}

\begin{ex}
    \label{ex2.58}
    状態$\ket{\psi}$は, 固有値$m$をもつ$M$の固有状態なので,
    \begin{align*}
        M \ket{\psi} = m \ket{psi}.
    \end{align*}
    平均測定値$E(M)$は,
    \begin{align*}
        E(M) = \braket{\psi| M |\psi} = m \braket{\psi | \psi} = m.
    \end{align*}
    標準偏差$\Delta(M)$は,
    \begin{align*}
        \Delta(M) = \sqrt{\braket{\psi| M^2 |\psi} - \braket{\psi| M |\psi}^2} = \sqrt{m^2 - m^2} = 0.
    \end{align*}
\end{ex}

\begin{ex}
    \label{ex2.59}
    \begin{align*}
        X = \ket{0}\bra{1} + \ket{1}\bra{0}
    \end{align*}
    なので,
    平均値は,
    \begin{align*}
        \braket{0|X|0} = 0.
    \end{align*}
    標準偏差$\Delta(X)$は,
    \begin{align*}
        \Delta(X)
        = \sqrt{\braket{0| X^2 |0} - \braket{0| X |0}^2}
        = \sqrt{1 - 0}
        = 1.
    \end{align*}
\end{ex}

\begin{ex}
    \label{ex2.60}
    まず, $v_3 \neq \pm 1$のときを考える.
    \begin{align*}
        \bm{v} \cdot \bm{\sigma}
        =
        \begin{pmatrix}
            v_3         & v_1 - i v_2 \\
            v_1 + i v_2 & - v_3
        \end{pmatrix}
    \end{align*}
    より, 固有値は, $\bm{v}$が単位ベクトルなことに注意して,
    \begin{align*}
        0 = \lambda^2 - |\bm{v}|^2 = \lambda^2 - 1
        \to \lambda = \pm1.
    \end{align*}
    対応する固有ベクトルは,
    \begin{align*}
        \ket{\lambda=\pm1} =
        \frac{1}{\sqrt{2(1\mp v_3)}}
        \begin{pmatrix}
            -v_1 + i v_2 \\ v_3 \mp 1.
        \end{pmatrix}
    \end{align*}
    よって, 射影オペレータは,
    \begin{align*}
        P_{\pm}
         & = \ket{\lambda = \pm1} \bra{\lambda = \pm1} \\
         & =
        \frac{1}{2(1\mp v_3)}
        \begin{pmatrix}
            -v_1 + i v_2 \\ v_3 \mp 1
        \end{pmatrix}
        \begin{pmatrix}
            v_1 - i v_2 & v_3 \mp 1
        \end{pmatrix}                    \\
         & =
        \frac{1}{2(1\mp v_3)}
        \begin{pmatrix}
            v_1^2 + v_2^2            & (-v_1 + i v_2)(v_3 \mp1) \\
            (-v_1 + i v_2)(v_3 \mp1) & (v_3 \mp 1)^2
        \end{pmatrix}                    \\
         & =
        \frac{1}{2(1\mp v_3)}
        \begin{pmatrix}
            1 - v_3^2                & (-v_1 + i v_2)(v_3 \mp1) \\
            (-v_1 + i v_2)(v_3 \mp1) & (v_3 \mp 1)^2
        \end{pmatrix}                    \\
         & =
        \frac{1}{2}
        \begin{pmatrix}
            1\pm v_3         & \pm(v_1 - i v_2) \\
            \pm(v_1 - i v_2) & 1\mp v_3
        \end{pmatrix}                    \\
         & =
        \frac{I \pm \bm{v} \cdot \bm{\sigma}}{2}.
    \end{align*}
    \par
    一方, $v_3 = \pm 1$のとき, $v_1 = v_2 = 0$となり,
    \begin{align*}
        \bm{v} \cdot \bm{\sigma}
        =
        \pm
        \begin{pmatrix}
            1 & 0   \\
            0 & - 1
        \end{pmatrix}
        =
        \pm Z.
    \end{align*}
    よって, 射影オペレータは, $v_3 = 1$のとき, 
    \begin{align*}
        P_{+} = \ket{0}\bra{0},
        P_{-} = \ket{1}\bra{1}
        \to
        P_{\pm} = \frac{I \pm \bm{v} \cdot \bm{\sigma}}{2},
    \end{align*}
    $v_3 = -1$のとき, 
    \begin{align*}
        P_{+} = \ket{1}\bra{1},
        P_{-} = \ket{0}\bra{0}
        \to
        P_{\pm} = \frac{I \pm \bm{v} \cdot \bm{\sigma}}{2}.
    \end{align*}
\end{ex}

\begin{ex}
    \label{ex2.61}
    $v_3\neq 1$のとき,
    $\ket{0}$の状態を測定して$+1$を得る確率は,
    \begin{align*}
        \left|
        \braket{\lambda = +1 | \bm{v} \cdot \bm{\sigma} | 0}
        \right|^2
        =
        \left|
        \braket{\lambda = +1 | 0}
        \right|^2
        =
        \left|
        \frac{-v_1 + iv_2}{\sqrt{2(1-v_3)}}
        \right|^2
        =
        \frac{1+v_3}{2}.
    \end{align*}
    これは, $v_3 = 1$でも成立する.
    $+1$を測定した直後の状態は,
    \begin{align*}
        \frac{P_+ \ket{0}}{\sqrt{\braket{0 | P_+^\dagger P_+ | 0}}}
        =
        \frac{\ket{\lambda = +1}\braket{\lambda = +1|0}}{\sqrt{\braket{\lambda = +1 | 0}\braket{0 | \lambda = +1}}}
        =
        e^{i\theta} \ket{\lambda=+1}.
    \end{align*}
    ここで, $\theta$は$\braket{\lambda = +1 | 0}$の偏角.
\end{ex}

\begin{ex}
    \label{ex2.62}
    測定オペレータがPOVMと一致するとすると,
    \begin{align*}
        M_m = M_m^\dagger M_m .
    \end{align*}
    $M_m^\dagger M_m$は正のオペレータなので, $M_m$も正のオペレータ.
    よって, 演習\ref{ex2.24}より$M_m$もHermite. したがって,
    \begin{align*}
        M_m = M_m^\dagger M_m = M_m^2.
    \end{align*}
    \\
    また, 完全性条件
    \begin{align*}
        \sum_m M^\dagger_m M_m =  \sum_m M_m = I
    \end{align*}
    から,
    \begin{align*}
        M_m
         & = \sum_{m'} M_{m'} M_m
        = M_m^2 + \sum_{m'\neq m} M_{m'} M_m
        = M_m + \sum_{m'\neq m} M_{m'} M_m
        \\
         & \to  \sum_{m'\neq m} M_{m'} M_m = O
    \end{align*}
    となり, $M_{m'} M_m (m'\neq m)$は正のオペレータゆえ,
    \begin{align*}
        M_{m'} M_m  = O \ (m'\neq m).
    \end{align*}
    先に示した$M_m^2 = M_m$と合わせて,
    \begin{align*}
        M_{m'} M_m  = \delta_{m m'} M_m.
    \end{align*}
    こうして, 測定オペレータがPOVMと一致するとすると, $M_m$が直交射影オペレータになることが言えた.
\end{ex}

\begin{ex}
    \label{ex2.63}
    定理2.3より, 明らか.
\end{ex}

\begin{ex}
    \label{ex2.64}
    $\{ \ket{\psi_i}\}_{i=1}^m$で張られるヒルベルト空間$V$とする. 各$j = 1, 2, \dots , m$に対して, $\{ \ket{\psi_i} \}_{i\neq j}$で張られる$V$の部分空間$W_j$, $W_j$の直交補空間$W_j^\perp$とする. $W_j^\perp$への射影オペレータ$P_j^\perp$する. すると, 
    \begin{align*}
        P_j^\perp \ket{\psi_j} \neq 0
    \end{align*}
    である. なぜなら, 
    \begin{align*}
        P_j^\perp \ket{\psi_j} = 0
    \end{align*}
    とすると, $\ket{\psi_j} \in W_j$, つまり$\ket{\psi_j}$が$\{ \ket{\psi_i} \}_{i\neq j}$の線型結合でかけることとなり, $\{ \ket{\psi_i}\}_{i=1}^m$が線型独立であることに矛盾するからである.
    そこで,\ $\ket{\phi_j}$を
    \begin{align*}
        \ket{\phi_j} = \frac{P_j^\perp \ket{\psi_j}}{ \sqrt{\left| P_j^\perp \ket{\psi_j} \right|}}
    \end{align*}
    で定義すると, 
    \begin{align*}
            \braket{\psi_i|\phi_j} = \delta_{ij}
    \end{align*}
    を満たす. また, 
    \begin{align*}
        E_j
        =
        \begin{cases}
            \frac{1}{2m} \ket{\phi_i} \bra{\phi_i} & (j=1,2, ...,m) \\
            I - \sum_{k=1}^{k=m} E_k    & (j=m+1)
        \end{cases}
    \end{align*}
    なる$\{ E_i\}_{i=1}^{m+1}$は,
    \begin{align*}
        &\sum_{i=1}^{m+1} E_m = I \\
        &\forall i = 1, 2, ..., m 
        \ \forall \ket{\psi} \in V
        \ \braket{\psi | E_i | \psi} = \frac{|\braket{\psi|\phi_i}|^2}{2m} \ge 0 \\
        &\forall \ket{\psi} \in V 
        \ \braket{\psi | E_{m+1} | \psi} 
        = \braket{\psi|\psi} - \frac{1}{2m}\sum_{i=1}^{m} |\braket{\psi|\phi_i}|^2
        \ge
        \frac{\braket{\psi|\psi}}{2} \ge 0
    \end{align*}
    を満たすのでPOVM.
    \par
    ここで構成したPOVMによる測定を考えると,
    \begin{align*}
        \braket{\psi_j | E_i | \psi_j} = \frac{1}{2m} \delta_{ij} 
        \ \left( i, j = 1, 2, ... ,m \right)
    \end{align*}
    なので, $\{ \ket{\psi_i}\}_{i=1}^m$を与えられたBobは, 測定結果$E_i \ (i = 1,2, ..., m)$を得られれば, 測定した状態が$\ket{\psi_i}$と確定できる.
\end{ex}

\begin{ex}
    基底を
    \begin{align*}
        \left\{ \frac{\ket{0}+\ket{1}}{\sqrt{2}},  \frac{\ket{0}-\ket{1}}{\sqrt{2}}\right\}
    \end{align*}
    ととれば良い.
\end{ex}

\begin{ex}
    \begin{align*}
        \frac{\bra{0}\otimes\bra{0} + \bra{1}\otimes \bra{1}}{\sqrt{2}}
        X_1 \otimes Z_2
        \frac{\ket{0} \otimes \ket{0} + \ket{1} \otimes \ket{1}}{\sqrt{2}}
        =
        \frac{\bra{0}\otimes\bra{0} + \bra{1}\otimes \bra{1}}{\sqrt{2}}
        \frac{\ket{1} \otimes \ket{0} - \ket{0} \otimes \ket{1}}{\sqrt{2}}
        =0.
    \end{align*}
\end{ex}

\begin{ex}
    $W$の直交補空間$W^{\perp}$とかき, それぞれの正規直交基底を$\{ \ket{w_i}\}, \{ \ket{w^{\perp}_j}\}$とし,
    \begin{align*}
        U'
        =
        \sum_{i} U\ket{w_i} \bra{w_i}
        +
        \sum_{j} \ket{w^\perp_j} \bra{w^\perp_j}
    \end{align*}
    とすればよい.
\end{ex}

\begin{ex}
    \label{ex2.68}
    あらゆる単一qビット$\ket{a},\ket{b}$は,
    \begin{align*}
        \ket{a} = a_0 \ket{0} + a_1 \ket{1}
        \\
        \ket{b} = b_0 \ket{0} + b_1 \ket{1}
    \end{align*}
    でかける.
    \begin{align*}
        \ket{\psi} = \frac{\ket{00} + \ket{11}}{\sqrt{2}}
    \end{align*}
    に対して,
    \begin{align*}
        \ket{\psi}
        =
        \ket{a} \ket{b}
    \end{align*}
    とかけると仮定すると,
    \begin{align*}
        a_0 b_1 = a_1 b_0 = 0, a_0 b_0 \neq 0, a_1 b_1 \neq 0
    \end{align*}
    となるが, これを満たす$a_0, a_1, b_0, b_1$は存在せず矛盾.
\end{ex}

\begin{ex}
    \label{ex2.69}
    Bell基底は,
    \begin{align*}
        \ket{\beta_{xy}} = \frac{\ket{0,y}+ (-1)^x\ket{1,\bar{y}}}{\sqrt{2}}
    \end{align*}
    でかけ,
    \begin{align*}
        \braket{\beta_{x'y'}|\beta_{xy}} = \frac{\delta_{yy'}+ (-1)^{x+x'}\delta_{\bar{y}\bar{y'}}}{2} = \delta_{xx'}\delta_{yy'}.
    \end{align*}
    となるので, 正規直交基底.
\end{ex}

\begin{ex}
    \label{ex2.70}
    Bell状態
    \begin{align*}
        \ket{\beta_{xy}} = \frac{\ket{0,y}+ (-1)^x\ket{1,\bar{y}}}{\sqrt{2}}
    \end{align*}
    に対して, $\braket{\beta_{xy}|E \otimes I| \beta{xy}}$は,
    \begin{align*}
        \braket{\beta_{xy}|E \otimes I| \beta{xy}}
        =
        \frac{\braket{0|E|0} + \braket{1|E|1}}{2}
    \end{align*}
    と$x,y$に依らない. 上に示したことから, AliceがBobに送ったqビットを, Eveが観測しても得られる状態の期待値はBell状態に依らないので, EveはAliceの送りたかった情報を推論できない.
\end{ex}

\begin{ex}
    \label{ex2.71}
    密度オペレータ$\rho$は, 正のオペレータなので, スペクトル定理から, $\rho$の固有空間の正規直交基底$\{ \ket{\psi_i}\}$を用いて, 
    \begin{align*}
        \rho = \sum_i p_i \ket{\psi_i}\bra{\psi_i}.
    \end{align*}
    とかけるので,
    \begin{align*}
        \tr{\left(\rho^2 \right)}
         & =
        \tr
        \left(
        \sum_{i,j} p_i p_j \ket{\psi_i}\braket{\psi_i|\psi_j}\bra{\psi_j}
        \right)                                                  \\
         & =
        \sum_{i,j} p_i p_j
        \tr
        \left(
        \ket{\psi_i}\braket{\psi_i|\psi_j}\bra{\psi_j}
        \right)                                                  \\
         & =
        \sum_{i,j} p_i p_j
        \delta_{ij}
        \tr
        \left(
        \ket{\psi_i}\bra{\psi_j}
        \right)                                                  \\
         & =
        \sum_{i} p_i ^2
        \tr
        \left(
        \braket{\psi_i|{\psi_i}}
        \right)                                                  \\
         & =\sum_{i} p_i ^2 \le \sum_{i} p_i = 1.
    \end{align*}
    等号は,
    \begin{align*}
        p_i =
        \begin{cases}
            1 & (i=i_0)              \\
            0 & (\mathrm{otherwise})
        \end{cases}
    \end{align*}
    つまり, $\rho$が純粋状態の時にのみ成り立つ.
\end{ex}

\begin{ex}
    \label{ex2.71}
    (1)\
    \begin{align*}
        \rho^2
         & =
        \frac{I + \bm{r}\cdot\bm{\sigma}}{2}
        \frac{I + \bm{r}\cdot\bm{\sigma}}{2} \\
         & =
        \frac{1}{4}
        \left(
        I + 2 r_i \sigma_i + r_i r_j \sigma^i \sigma^j
        \right)                              \\
         & =
        \frac{1}{4}
        \left(
        I + 2 r_i \sigma_i + r_i r_j \frac{\{\sigma^i, \sigma^j\}}{2}
        \right)                              \\
         & =
        \frac{1}{4}
        \left(
        (1+\bm{r}^2)I + 2 \bm{r} \cdot \bm{\sigma}
        \right)                              \\
         & =
        \frac{\bm{r}^2 - 1}{4}I + \rho
    \end{align*}
    で$\rho$は,
    \begin{align*}
        \rho = \rho^\dagger
    \end{align*}
    を満たすので, 任意の状態$\ket{\psi}$に対して, $\bm{r}^2 \le 1$なら,
    \begin{align*}
        \braket{\psi | \rho | \psi}
        =
        \braket{\psi | \rho^2 + \frac{1-\bm{r}^2}{4} | \psi}
        =
        \braket{\psi | \rho^\dagger \rho | \psi}
        +
        \braket{\psi | \frac{1-\bm{r}^2}{4} | \psi}
        \ge 0.
    \end{align*}
    また, $\tr{(\sigma_i)} = 0$なので,
    \begin{align*}
        \tr(\rho) = 1.
    \end{align*}
    以上より, $\rho$は, 正かつトレースが1なので, 定理2.5より$\rho$は密度オペレータ.
    \par
    (2)\ $\bm{r} = \bm{0}$.
    \par
    (3)\
    \begin{align*}
        \rho \mathrm{が純粋状態}
        \Leftrightarrow
        \tr(\rho^2) = 1
        \Leftrightarrow
        \tr\left(
        \frac{1-\bm{r}^2}{4}+\rho
        \right)
        \Leftrightarrow
        |\bm{r}| =1.
    \end{align*}
    \par
    (4)
    $\bm{r} = (\sin\theta \cos\phi, \sin\theta \sin\phi, \cos\theta)$として,
    \begin{align*}
        \rho
        =
        \frac{I + \bm{r}\cdot\bm{\sigma}}{2}
        =
        \frac{1}{2}
        \begin{pmatrix}
            1 + \cos\theta        & \sin\theta e^{-i\phi} \\
            \sin\theta e^{i \phi} & 1 - \cos\theta
        \end{pmatrix}
        =
        \ket{\psi} \bra{\psi}.
    \end{align*}
    ただし, ここで
    \begin{align*}
        \ket{\psi} = {\cos\frac{\theta}{2}}\ket{0}+e^{i\phi}{\sin\frac{\theta}{2}}\ket{1}.
    \end{align*}
\end{ex}

\begin{ex}
    \label{ex2.73}
    密度オペレータ$\rho$の台$\mathrm{support}\left( \rho \right)$として, 
    \begin{align*}
        \rank \rho 
        = (\rho^2 \mathrm{の非負固有値の数})
        = (\rho \mathrm{の非負固有値の数})
        = \dim \mathrm{support}\left( \rho \right).
    \end{align*}
    $\rho$の非負固有値${p_i} \ \left(i = 1, 2, ..., \dim \mathrm{support}\left( \rho \right) \right)$として$\rho$のスペクトル分解を
    \begin{align*}
        \rho 
        = \sum_{i=1}^{\dim \mathrm{support}\left( \rho \right)} q_i \ket{\phi_i}\bra{\phi_i}
        = \sum_{i=1}^{\rank \rho} q_i \ket{\phi_i}\bra{\phi_i} 
    \end{align*}
    とする.
    $\mathrm{support}\left( \rho \right)$の任意の状態$\ket{\psi_1}$に対して,
    $\braket{\psi_i|\rho^{-1}|\psi_i} > 0$なので,  
    \begin{align*}
        p_1 &\equiv \frac{1}{\braket{\psi_i|\rho^{-1}|\psi_i}} \\
        u_{1j} &\equiv \sqrt{ \frac{p_1}{q_j}} \braket{\phi_j|\psi_1} \ (j = 1, 2, ..., \rank \rho) \\
    \end{align*}
    を定義すると, 
    \begin{align*}
        \ket{\tilde{\psi_1}} \equiv
        \sqrt{p_1} \ket{\psi_1} 
        =
        \sum_{j=1}^{\rank \rho} u_{1j}\sqrt{q_j}\ket{\phi_j}
    \end{align*}
    を満たす. また, 
    \begin{align*}
        \sum_{j=1}^{\rank \rho} \left| u_{1j} \right|^2 = 1
    \end{align*}
    であるから
    \begin{align*}
        \ket{u_1} \equiv
        \begin{pmatrix}
            u_{11} \\ 
            u_{12} \\
            \vdots \\ 
            u_{1 \rank \rho}
        \end{pmatrix}
    \end{align*}
    は正規化された$\mathbf{C}^{\rank \rho}$のベクトルなので, $\mathbf{C}^{\rank \rho}$の正規直交基底として, 
    \begin{align*}
        \left\{\ket{u_i}  \equiv \left( u_{i1}, u_{i2}, ..., u_{i \rank \rho}\right)^\top \right\}_{i = 1}^{\rank \rho}
    \end{align*}
    なるものが取れて, $u = (u_{ij})$はユニタリ行列となる. 
    したがって, 各$i = 2, ..., \rank \rho$に対して, 
    \begin{align*}
        \ket{\tilde{\psi_i}} &\equiv \sum_{j=1}^{\rank \rho} u_{ij} \sqrt{q_j} \ket{\phi_j} \\
        p_i &\equiv \braket{\tilde{\psi_i}|\tilde{\psi_i}} > 0 \\
        \ket{\psi_i} &\equiv \frac{\ket{\tilde{\psi_i}}}{\sqrt{p_i}}
    \end{align*}
    とすると, 
    各$i = 1, 2, ..., \rank \rho$に対して, 
    \begin{align*}
        \ket{\tilde{\psi_i}} = \sum_{j=1}^{\rank \rho} u_{ij} \sqrt{q_j} \ket{\phi_j}
    \end{align*}
    が成立するので, 定理2.6より,
    \begin{align*}
        \rho 
        = 
        \sum_{i=1}^{\rank \rho} \ket{\tilde{\psi_i}}\bra{\tilde{\psi_i}}
        =
        \sum_{i=1}^{\rank \rho} p_i \ket{\psi_i}\bra{\psi_i}
    \end{align*}
    が成立. 
    つまり, 
    $\mathrm{support}\left( \rho \right)$の任意の状態$\ket{\psi_1}$に対して,
    $\ket{\psi_1}$を含む$\rho$の極小アンサンブル$\left\{ p_i ; \ket{\psi_i}\right\}_{i=1}^{\rank \rho}$が存在することがいえた. 特に, 
    \begin{align*}
        p_1 = \frac{1}{\braket{\psi_1 | \rho^{-1} | \psi_1}}
    \end{align*}
    である.
\end{ex}

\begin{ex}
    \label{ex2.74}
    複合システムの密度オペレータ$\rho_{AB}$は,
    \begin{align*}
        \rho_{AB} = \ket{a} \bra{a} + \ket{b} \bra{b}.
    \end{align*}
    よって, システム$A$で縮約した密度オペレータは,
    \begin{align*}
        \rho^A = \ket{a} \bra{a} \tr\left(\ket{b} \bra{b}\right) = \ket{a} \bra{a}
    \end{align*}
    なので,
    \begin{align*}
        tr\left({\rho^{A}}^2\right) = 1
    \end{align*}
    と$\rho^A$が純粋状態であることがわかる.
\end{ex}

\begin{ex}
    \label{ex2.75}
    各Bell状態
    \begin{align*}
        \ket{\beta_{xy}} = \frac{\ket{0,y}+ (-1)^x\ket{1,\bar{y}}}{\sqrt{2}}
    \end{align*}
    に対して, 密度オペレータは,
    \begin{align*}
        \rho
         & =
        \frac{\ket{0,y}+ (-1)^x\ket{1,\bar{y}}}{\sqrt{2}}
        \frac{\bra{0,y}+ (-1)^x\bra{1,\bar{y}}}{\sqrt{2}} \\
         & =
        \frac{
            \ket{0,y}\bra{0,y} + (-1)^x\ket{1,\bar{y}}\bra{0,y}
            +
            (-1)^x\ket{0,y}\bra{1,\bar{y}} + \ket{1,\bar{y}}\bra{1,\bar{y}}
        }{2}.
    \end{align*}
    よって, 各qビットを縮約した密度オペレータは,
    \begin{align*}
        \rho^1 & = \frac{\ket{y}\bra{y}  + \ket{\bar{y}}\bra{\bar{y}} }{2} = \frac{I}{2} \\
        \rho^2 & = \frac{ \ket{0}\bra{0} +  \ket{1}\bra{1}}{2} = \frac{I}{2}.
    \end{align*}
\end{ex}

\begin{ex}
    \label{ex2.76}
    $m\times n$の行列$C$に対して, $m \times m$のユニタリ行列$U$, $n\times n$のユニタリ行列$V$, $m \times n$の行列$D$;
    D =
    \begin{align*}
        \begin{pmatrix}
            \lambda_1 &           &        &                          &                                        \\
                      & \lambda_2 &        &                          &                                        \\
                      &           & \ddots &                          &                                        \\
                      &           &        & \lambda_{\mathrm{rank}A} &                                        \\
                      &           &        &                          & O_{m-\mathrm{rank}A, n-\mathrm{rank}A} \\
        \end{pmatrix}
    \end{align*}
    を用いて,
    \begin{align*}
        C = UDV
    \end{align*}
    とかける(特異値分解). ここで, $\lambda_i$は$C$の特異値で非負である.
    \par
    $A \otimes B$の任意の純粋状態$\ket{\psi}$に対して,
    $A,B$の正規直交基底$\ket{j}, \ket{k}$を用いて,
    \begin{align*}
        \ket{ \psi} = \sum_{jk} C_{jk} \ket{j} \ket{k}.
    \end{align*}
    $C$の特異値分解を考えると, ユニタリ行列$U,V$が存在し,
    \begin{align*}
        \ket{ \psi} = \sum_{ijk} U_{ji} D_{ii} V_{ik}\ket{j} \ket{k}.
    \end{align*}
    ここで,
    \begin{align*}
        \ket{i_A} = \sum_j U_{ji} \ket{j}, \ket{i_B} = \sum_k V_{ik} \ket{k}
    \end{align*}
    とすれば, $\ket{i_A}, \ket{i_B}$は$A,B$の正規直交基底なので, $\ket{i_A}\ket{i_B}$
    は$A \otimes B$の正規直交基底で,
    \begin{align*}
        \ket{ \psi} = \sum_{i} D_{ii} \ket{i_A}\ket{i_B}.
    \end{align*}
    ただし, $\tr\left( \ket{\psi} \bra{\psi}\right) = 1$より,
    \begin{align*}
        \sum_{i} D_i^2 = 1
    \end{align*}
    を満たす.
\end{ex}

\begin{ex}
    \label{ex2.77}
    例えば, $\mathbb{C}^2 \otimes \mathbb{C}^2 \otimes \mathbb{C}^2$の状態$\ket{\psi}$;
    \begin{align*}
        \ket{\psi}= \ket{0} \left( {\ket{00} + \ket{11}}\right)
    \end{align*}
    は,
    \begin{align*}
        \ket{\psi}
        =
        \lambda_0 \ket{0_{A}{0_{B}}{0_{C}}}
        +
        \lambda_1 \ket{1_{A}{1_{B}}{1_{C}} } \\
        \braket{i_{A} | j_{A}} = \braket{i_{B} | j_{B}} = \braket{i_{C} | j_{C}} = \delta_{ij} \
    \end{align*}
    の形でかけない. なぜなら, $\ket{\psi}$でかっこでくくり出された$\ket{0}$のせいである.
\end{ex}

\begin{ex}
    \label{ex2.78}
    $\ket{\psi}$のSchmidt数が1なら, $\ket{\psi} = \ket{a} \otimes \ket{b}$とかけるので$\ket{\psi}$は積状態.
    逆に, $\ket{\psi}$が積状態$\ket{ \psi } = \ket{a} \ket{b}$なら, 明らかに$\mathrm{Sch}(\psi) = 1$.
    \par
    $\rho^A$が純粋状態とする. $\ket{\psi}$のSchmidt分解
    \begin{align*}
        \ket{\psi} = \sum_{i = 1}^{\mathrm{Sch}(\psi)} \sqrt{p_i} \ket{\eta_i} \otimes \ket{\zeta_i}
    \end{align*} 
    とすると, 
    \begin{align*}
        \rho^A 
        = \tr_{B} \ket{\psi}\bra{\psi} 
        =  \sum_{i = 1}^{\mathrm{Sch}(\psi)} \ket{\eta_i}\bra{\eta_i}
    \end{align*}
    で$\rho^A$が純粋状態なので, ある$i_0$があって, $p_i = \delta_{i i_0}$なので, 
    \begin{align*}
        \ket{\psi} = \ket{\eta_{i_0}} \otimes \ket{\zeta_{i_0}}
    \end{align*}
    と$\ket{\psi}$は積状態. 逆に, $\ket{\psi}$が積状態であるとすると, $\ket{\psi}= \ket{a} \otimes \ket{b}$とかけ, $\rho^A = \tr_{B} \ket{\psi}\bra{\psi}= \ket{a} \bra{a}$と$\rho^A$は純粋状態.
\end{ex}

\begin{ex}
    \label{ex2.79}
    (1)\
    \begin{align*}
        \frac{\ket{00} + \ket{11}}{\sqrt{2}}.
    \end{align*}
    \par
    (2)\
    \begin{align*}
        \frac{\ket{00}+\ket{01}+\ket{10}+\ket{11}}{2} = \frac{\ket{0} + \ket{1}}{\sqrt{2}} \frac{\ket{0} + \ket{1}}{\sqrt{2}} .
    \end{align*}
    (3)\
    \begin{align*}
        p_{\pm} &= \frac{3 \pm \sqrt{5}}{6} \\
        \ket{\eta_\pm} &= \frac{-1}{\sqrt{10}} 
        \left( 
            \sqrt{5 \pm \sqrt{5} } \ket{0}
            \pm
            \sqrt{5 \mp \sqrt{5} } \ket{1}
        \right) \\
        \ket{\zeta_\pm} &= \frac{-1}{\sqrt{10}} 
        \left( 
            \pm \sqrt{5 \pm \sqrt{5} } \ket{0}
            +
            \sqrt{5 \mp \sqrt{5} } \ket{1}
        \right)
    \end{align*}
    として, 
    \begin{align*}
        \frac{\ket{00}+\ket{01} + \ket{10}}{\sqrt{3}} =
        \sum_{i = \pm} \sqrt{p_i} \ket{\eta_i} \otimes \ket{\zeta_i}
    \end{align*}
    とSchmidt分解される.
\end{ex}

\begin{ex}
    \label{ex2.80}
    \begin{align*}
        S \equiv \mathrm{Sch}(\psi) = \mathrm{Sch}(\phi)
    \end{align*}
    として, $\ket{\psi}, \ket{\phi}$のSchmidt分解
    \begin{align*}
        \ket{\psi} = \sum_{i=1}^{S} \lambda_i \ket{i_A}\ket{i_B}                  \\
        \ket{\phi} = \sum_{i=1}^{S} {\lambda}_i \ket{\tilde{i}_A} \ket{\tilde{i}_B}.
    \end{align*}
    ここで, $\mathcal{H}_A, \mathcal{H}_B, \tilde{\mathcal{H}}_A, \tilde{\mathcal{H}}_B$をそれぞれ
    $\left\{ \ket{i_A}\right\}_{i = 1}^S, \left\{ \ket{i_B}\right\}_{i = 1}^S,
    \left\{ \ket{\tilde{i}_A}\right\}_{i = 1}^S, \left\{ \ket{\tilde{i}_B}\right\}_{i = 1}^S$で張られるヒルベルト空間として, 
    \begin{align*}
        U = \sum_{j=1}^S \ket{j_A} \bra{\tilde{j}_A} \\
        V= \sum_{j=1}^S \ket{j_B} \bra{\tilde{j}_B}
    \end{align*}
    とすれば, $U : \tilde{\mathcal{H}}_A \to \mathcal{H}_A , \ V : \tilde{\mathcal{H}}_B \to \mathcal{H}_B $は内積を保存する全単射な線型写像, つまりユニタリで,
    \begin{align*}
        \ket{\psi} = (U \otimes V)\ket{\phi}
    \end{align*}
    を満たす.
\end{ex}

\begin{ex}
    \label{ex2.81}
    \begin{align*}
        \rho^A = \sum_i p_i \ket{i^A} \bra{i^A}
    \end{align*}
    の純粋化
    \begin{align*}
        \ket{AR_1} = \sum_i \sqrt{p_i} \ket{i^A} \ket{i^{R1}} \\
        \ket{AR_2} = \sum_i \sqrt{p_i} \ket{i^A} \ket{i^{R2}}
    \end{align*}
    に対して,
    \begin{align*}
        U_R = \sum_i \ket{i^{R1}} \bra{i^{R2}}
    \end{align*}
    とすれば, $U_R$はユニタリで,
    \begin{align*}
        \ket{A R_1} = (I_A \otimes U_R) \ket{A R_2}
    \end{align*}
    を満たす.
\end{ex}

\begin{ex}
    \label{ex2.82}
    \begin{align*}
        \rho^A = \sum_i p_i \ket{\psi_i} \bra{\psi_i}
    \end{align*}
    とする. \par
    (1)\
    (2.211)式と同様.
    \par
    (2)\
    $R$を測定して, 結果$i$を得る確率は,
    \begin{align*}
        R_i = I \otimes \ket{i} \bra{i}
    \end{align*}
    として,
    \begin{align*}
        \tr(R_i^\dagger R_i \rho^{AR}) = \tr(R_i \rho^{AR}).
    \end{align*}
    結果$i$を得た後の状態は,
    \begin{align*}
        \frac{R_i \rho^{AR} R_i}{\tr (R_i \rho^{AR})}.
    \end{align*}
    (3)\
    システム$A$の正規直交基底$\left\{\ket{i}\right\}_i$を固定する.
    演習2.81より, 
    システム$AR$に対する$\rho^A$の任意の純粋化$\ket{AR}$は, あるシステム$R$のユニタリ$U_R$を用いて
    \begin{align*}
        \ket{AR} = \sum_i \sqrt{p_i} \ket{\psi_i} \otimes U_R \ket{i}.
    \end{align*}
    で書ける. 純粋状態$\ket{AR} \bra{AR}$に対して, システム$R$を測定して$i$を得る確率は,
    \begin{align*}
        R_i = I \otimes U_R \ket{i} \bra{i} U_R^\dagger
    \end{align*}
    として,
    \begin{align*}
        \tr\left[R_i^\dagger R_i ( \ket{AR} \bra{AR})\right]
         & =
        \sum_{j,k} \sqrt{p_j} \sqrt{p_k}
        \tr
        \left[
            \big(
            I \otimes U_R \ket{i}\bra{i} U_R^\dagger
            \big)
            \big(
            \ket{\psi_j} \otimes U_R\ket{j}
            \big)
            \big(
            \bra{\psi_k} \otimes \bra{k} U_R^\dagger
            \big)
            \right]
        \\
         & =
        \sum_{k} \sqrt{p_i} \sqrt{p_k}
        \tr
        \left[
            \big(
            \ket{\psi_i} \otimes U_R\ket{i}
            \big)
            \big(
            \bra{\psi_k} \otimes \bra{k} U_R^\dagger
            \big)
            \right]
        \\
         & =
        \sum_{k} \sqrt{p_i} \sqrt{p_k} \braket{\psi_k | \psi_i} \braket{i|k} \\
         & =
        p_i.
    \end{align*}
    $i$を得た後の状態は,
    \begin{align*}
        \frac{R_i \rho^{AR} R_i}{p_i}=
        \left(
        \ket{\psi_i} \otimes \ket{i}
        \right)
        \left(
        \bra{\psi_i} \otimes \bra{i}
        \right).
    \end{align*}
    以上より, 任意の$\rho^A$の任意の純粋化に対して,
    $R$を測定すると確率$p_i$で, システム$A$の測定後の状態が$\ket{\psi_i}$となる
    $R$の正規直交基底$\{ U_R \ket{i} \}_i$が存在することが言えた.
\end{ex}

\begin{problem}
\label{problem2.1}
演習\ref{ex2.35}で見たように, 単位ベクトル$\bm{n}$に対して,
\begin{align*}
    \left(\bm{n} \cdot \bm{\sigma} \right)^n =
    \begin{cases}
        \bm{n} \cdot \bm{\sigma} & (n: \mathrm{even} ) \\
        I                        & (n: \mathrm{odd} )
    \end{cases}.
\end{align*}
よって, Taylorの定理より,
\begin{align*}
    f(\theta \bm{n} \cdot \bm{\sigma} )
     & =
    I \sum_{n: \mathrm{even}} \frac{f^{(n)}(0)}{n!} \theta^n
    +
    \bm{n} \cdot \bm{\sigma} \sum_{n: \mathrm{odd}} \frac{f^{(n)}(0)}{n!} \theta^n
    \\
     & =
    I \sum_{n: \mathrm{even}} \frac{f^{(n)}(0)}{n!} \frac{\theta^n + (-\theta)^n}{2}
    +
    \bm{n} \cdot \bm{\sigma} \sum_{n: \mathrm{odd}} \frac{f^{(n)}(0)}{n!} \frac{\theta^n - (-\theta)^n}{2}
    \\
     & =
    I \sum_{n} \frac{f^{(n)}(0)}{n!} \frac{\theta^n + (-\theta)^n}{2}
    +
    \bm{n} \cdot \bm{\sigma} \sum_{n} \frac{f^{(n)}(0)}{n!} \frac{\theta^n - (-\theta)^n}{2}
    \\
     & =
    \frac{f(\theta) + f(-\theta)}{2} I + \frac{f(\theta) - f(-\theta)}{2}\bm{n} \cdot \bm{\sigma}.
\end{align*}
\end{problem}

\begin{problem}
\label{problem2.2}
(1)\
$A,B$のある正規直交基底$\{\ket{i^A}\},\{ \ket{i^B} \}$を用いて, $A \otimes B$の純粋状態$\ket{\psi}$は, $\lambda_i > 0$として,
\begin{align*}
    \ket{\psi} = \sum_{i=1}^{\mathrm{Sch}(\psi)} \lambda_i \ket{i^A} \ket{i^B}
\end{align*}
とSchmidt分解可能. すると,
\begin{align*}
    \rho^A = \tr_B(\ket{\psi} \bra{\psi}) = \sum_{i=1}^{\mathrm{Sch}(\psi)} \lambda_i^2 \ket{i^A} \bra{i^A}
\end{align*}
となり, これは$\rho^A$のスペクトル分解にもなっていて, $\rho^A$の台の次元の数($\mathrm{rank}\rho^A$)が$\mathrm{Sch}(\psi)$に等しいことを意味している.
\par
(2)\
\begin{align*}
    \ket{\psi} = \sum_{i = 1}^{n} \ket{\alpha_i}\ket{\beta_i}
\end{align*}
とかけているとき$\{ \ket{\alpha_i} \}$は互いに一次独立になっているとする. すると, $\{ \ket{\alpha_i} \}$で張られる次元$n$のベクトル空間$V$が定義できる. ここで, (1)の結果を用いると,
\begin{align*}
    \mathrm{Sch}(\psi)
    =  \mathrm{rank} \left( \tr_B(\ket{\psi} \bra{\psi}) \right)
    =  \mathrm{rank} \left( \sum_{j,k = 1}^{n} \ket{\alpha_j}\braket{\beta_k|\beta_j} \bra{\alpha_k}\right)
    \leq n.
\end{align*}
最後の不等式では, $\left( \sum_{j,k = 1}^{n} \ket{\alpha_j}\braket{\beta_k|\beta_j} \bra{\alpha_k}\right)$が, $V$上で定義されたある線形オペレータの行列表示(サイズは$n\times n$)になっていることを用いた.
\par
(3)\
\begin{align*}
    \mathrm{Sch}(\phi) \geq \mathrm{Sch}(\gamma)
\end{align*}
としても一般性を失わない.
$\ket{\phi}, \ket{\gamma}$のSchmidt分解は,
\begin{align*}
    \ket{\phi} = \sum_{i=1}^{\mathrm{Sch}(\phi)} \lambda_i \ket{i^A} \ket{i^B} \\
    \ket{\gamma} = \sum_{i=1}^{\mathrm{Sch}(\gamma)} \xi_i \ket{\tilde{i}^A} \ket{\tilde{i}^B} .
\end{align*}
$\alpha, \beta$の一方が0の時は, 明らか
に示すべき式が成立するので,
$\alpha \neq 0, \beta \neq = 0$として,
\begin{align*}
    \ket{\psi} = \alpha \ket{\phi} + \beta \ket{\gamma}
\end{align*}
のとき,
(1)より,
\begin{align*}
    \mathrm{Sch}(\psi)
     & =
    \mathrm{rank} \left( \tr_B(\ket{\psi} \bra{\psi}) \right)
    \\
     & =
    \mathrm{rank}
    \left(
    \alpha \sum_{i=1}^{\mathrm{Sch}(\phi)} \lambda_i^2 \ket{i^A} \bra{i^A}
    +
    \beta \sum_{i=1}^{\mathrm{Sch}(\gamma)} \xi_i^2 \ket{\tilde{i}^A} \bra{\tilde{i}^A}
    \right)
\end{align*}
この右辺の最小値は,
\begin{align*}
    \alpha \sum_{i=1}^{\mathrm{Sch}(\phi)} \lambda_i^2 \ket{i^A} \bra{i^A}
    +
    \beta \sum_{i=1}^{\mathrm{Sch}(\gamma)} \xi_i^2 \ket{\tilde{i}^A} \bra{\tilde{i}^A}
     & =
    \sum_{i=1}^{\mathrm{Sch}(\phi) - \mathrm{Sch}(\gamma)} \alpha \lambda_i^2 \ket{i^A} \bra{i^A}
    +
    \sum_{i=1}^{\mathrm{Sch}(\gamma)} ( \alpha \lambda_i^2 + \beta \xi_i^2) \ket{\tilde{i}^A} \bra{\tilde{i}^A}
\end{align*}
で,
\begin{align*}
    \alpha \lambda_i^2 + \beta \xi_i^2 = 0 \ (i=1, 2, \dots \mathrm{Sch}(\gamma))
\end{align*}
のときに実現される. つまり,
\begin{align*}
    \min \mathrm{rank}
    \left(
    \alpha \sum_{i=1}^{\mathrm{Sch}(\phi)} \lambda_i^2 \ket{i^A} \bra{i^A}
    +
    \beta \sum_{i=1}^{\mathrm{Sch}(\gamma)} \xi_i^2 \ket{\tilde{i}^A} \bra{\tilde{i}^A}
    \right)
    =
    \sum_{i=1}^{\mathrm{Sch}(\phi) - \mathrm{Sch}(\gamma)} \alpha \lambda_i^2 \ket{i^A} \bra{i^A}
    =
    \mathrm{Sch}(\phi) - \mathrm{Sch}(\gamma).
\end{align*}
よって,
\begin{align*}
    \mathrm{Sch}(\psi) =
    \mathrm{rank}
    \left(
    \alpha \sum_{i=1}^{\mathrm{Sch}(\phi)} \lambda_i^2 \ket{i^A} \bra{i^A}
    +
    \beta \sum_{i=1}^{\mathrm{Sch}(\gamma)} \xi_i^2 \ket{\tilde{i}^A} \bra{\tilde{i}^A}
    \right)
    \geq  \mathrm{Sch}(\phi) - \mathrm{Sch}(\gamma).
\end{align*}
\end{problem}

\begin{problem}
\label{problem2.3}
演習\ref{ex2.35}で見たように,
\begin{align*}
    Q^2 = R^2 = S^2 = T^2 = I.
\end{align*}
このことと, $Q,R,S,T$が互いに非可換なことに注意して,
\begin{align*}
    \left(
    Q \otimes S + R \otimes S + R \otimes T - Q \otimes T
    \right)^2
    =
    4I + [Q,R] \otimes [S,T].
\end{align*}
$Q,R,S,T$がHermiteなので, 式(2,107)より,
\begin{align*}
    \left| \ \left< [Q,R]\right> \ \right|
     & \leq 2 \left| \left< Q^2 \right> \right| \left| \left< R^2\right> \right|  = 2 \\
    \left| \ \left< [S,T]\right> \ \right|
     & \leq 2 \left| \left< S^2 \right> \right| \left| \left< T^2\right> \right|  = 2
\end{align*}
が成り立つことと, 先に示した等式を合わせて,
\begin{align*}
    \left<
    Q \otimes S + R \otimes S + R \otimes T - Q \otimes T
    \right>^2
     & \leq
    \left<\
    \left(
    Q \otimes S + R \otimes S + R \otimes T - Q \otimes T
    \right)^2\
    \right>
    \\
     &
    =
    \left<\
    4I + [Q,R] \otimes [S,T]
    \
    \right>
    \leq 8.
\end{align*}
よって,
\begin{align*}
    \left<
    Q \otimes S + R \otimes S + R \otimes T - Q \otimes T
    \right>
    \leq 2 \sqrt{2}.
\end{align*}
\end{problem}