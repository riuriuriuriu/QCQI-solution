\chapter{量子誤り訂正}
$F_2$は集合${0,1}$に通常の2進法の加法と乗法を入れた体.
\begin{ex}
    \label{ex10.1}
    明らか.
\end{ex}


\begin{ex}
    \label{ex10.2}
    \begin{align*}
        \qo(\rho)
         & =
        \left( 1 - 2p\right) \rho + 2 p P_{+} \rho P_{+} + 2p P_{-} \rho P_{-} \\
         & =
        \left( 1 - 2p\right) \rho + p\left( \ket{0}\bra{0} + \ket{1}\bra{1} \right)\rho
        \left( \ket{0}\bra{0} + \ket{1}\bra{1} \right)
        + p \left( \ket{0}\bra{1} +  \ket{1}\bra{0} \right)\rho
        \left( \ket{0}\bra{1} + \ket{1}\bra{0}\right)                          \\
         & =
        \left( 1 - 2p\right) \rho  + P \rho + p X \rho X
        \\
         & =
        \left( 1 - p\right) \rho  + P \rho + p X \rho X.
    \end{align*}
\end{ex}

\begin{ex}
    \label{ex10.3}
    第3qubitにビット反転が生じた後の状態, $\ket{\psi}=a\ket{001}+b\ket{110}$として,
    $Z_1 Z_2$, $Z_2 Z_3$を測定すると,
    \begin{align*}
        Z_1 Z_2 \ket{\psi} = \ket{\psi} \\
        Z_2 Z_3 \ket{\psi} = - \ket{\psi}
    \end{align*}
    より, 測定値はぞれぞれ$(1, -1)$, 測定後の状態$\ket{\psi}$となる. 同様に, 第1qubit, 第2qubitにビット反転が生じた場合, $Z_1 Z_2$,$Z_2 Z_3$の測定値は, それぞれ$(-1, 1)$, $(-1, -1)$. また, どのqubitにもビット反転が起らないのならば, $Z_1 Z_2$,$Z_2 Z_3$の測定値は, それぞれ$(1, 1)$.
\end{ex}

\begin{ex}
    \label{ex10.4}
    (1)\
    $P_0, P_1$ に対応する測定結果が得られればビット反転していない, $P_{2j}, P_{2j+1}$に対応する測定結果が得られれば第$j$ビットが反転した, と診断できる.
    (2)\
    計算基底でないと, error detection により量子状態を壊してしまう.
    (3)\
    ビット反転が単一qubitに起こる確率を$p$とする. 始状態が純粋状態$\ket{\psi}$として, 誤り訂正後の状態$\rho$は, (10.11)式と同様に,
    \begin{align*}
        \rho = \left[ (1-p)^3 + 3p(1-p)^2\right]\ket{\psi}\bra{\psi} + \left(\mathrm{positive \ operetor}\right)
    \end{align*}
    であり, (9.121)より最小忠実度$F$は純粋状態について考えれば十分なことから,
    \begin{align*}
        F = \sqrt{\braket{\psi| \rho | \psi}} \geq \sqrt{(1-p)^3 + 3p(1-p)^2}.
    \end{align*}
\end{ex}

\begin{ex}
    \label{ex10.5}
    $X_{1:6} \equiv X_1 X_2 ... X_6, X_{4:9} \equiv X_4 X_5 ... X_9$として,
    \begin{itemize}
        \item 位相反転しない場合
              \begin{align*}
                  X_{1:6}\frac{(\ket{000}+\ket{111})(\ket{000}+\ket{111})(\ket{000}+\ket{111})}{2\sqrt{2}} & = \frac{(\ket{000}+\ket{111})(\ket{000}+\ket{111})(\ket{000}+\ket{111})}{2\sqrt{2}} \\
                  X_{4:9}\frac{(\ket{000}+\ket{111})(\ket{000}+\ket{111})(\ket{000}+\ket{111})}{2\sqrt{2}} & = \frac{(\ket{000}+\ket{111})(\ket{000}+\ket{111})(\ket{000}+\ket{111})}{2\sqrt{2}} \\
              \end{align*}
        \item 第1qubitが位相反転した場合
              \begin{align*}
                  X_{1:6}\frac{(\ket{000}-\ket{111})(\ket{000}+\ket{111})(\ket{000}+\ket{111})}{2\sqrt{2}} & = -\frac{(\ket{000}-\ket{111})(\ket{000}+\ket{111})(\ket{000}+\ket{111})}{2\sqrt{2}} \\
                  X_{4:9}\frac{(\ket{000}-\ket{111})(\ket{000}+\ket{111})(\ket{000}+\ket{111})}{2\sqrt{2}} & = \frac{(\ket{000}-\ket{111})(\ket{000}+\ket{111})(\ket{000}+\ket{111})}{2\sqrt{2}}  \\
              \end{align*}
        \item 第2qubitが位相反転した場合
              \begin{align*}
                  X_{1:6}\frac{(\ket{000}+\ket{111})(\ket{000}-\ket{111})(\ket{000}+\ket{111})}{2\sqrt{2}} & = -\frac{(\ket{000}+\ket{111})(\ket{000}-\ket{111})(\ket{000}+\ket{111})}{2\sqrt{2}} \\
                  X_{4:9}\frac{(\ket{000}+\ket{111})(\ket{000}-\ket{111})(\ket{000}+\ket{111})}{2\sqrt{2}} & = -\frac{(\ket{000}+\ket{111})(\ket{000}-\ket{111})(\ket{000}+\ket{111})}{2\sqrt{2}} \\
              \end{align*}
        \item 第3qubitが位相反転した場合
              \begin{align*}
                  X_{1:6}\frac{(\ket{000}+\ket{111})(\ket{000}+\ket{111})(\ket{000}-\ket{111})}{2\sqrt{2}} & = \frac{(\ket{000}+\ket{111})(\ket{000}+\ket{111})(\ket{000}-\ket{111})}{2\sqrt{2}}  \\
                  X_{4:9}\frac{(\ket{000}+\ket{111})(\ket{000}+\ket{111})(\ket{000}-\ket{111})}{2\sqrt{2}} & = -\frac{(\ket{000}+\ket{111})(\ket{000}+\ket{111})(\ket{000}-\ket{111})}{2\sqrt{2}} \\
              \end{align*}
    \end{itemize}
    となり, シンドローム測定としての役割を果たす.
\end{ex}

\begin{ex}
    \label{ex10.6}
    \begin{align*}
        Z_1 Z_2 Z_3 \frac{\ket{000}-\ket{111}}{\sqrt{2}} = \frac{\ket{000} + \ket{111}}{\sqrt{2}}.
    \end{align*}
\end{ex}

\begin{ex}
    \label{ex10.7}
    \begin{align*}
        \alpha_{00} = (1-p)^3, \alpha_{11} = \alpha_{22} = \alpha_{33} = p(1-p)^2, \alpha_{ij} = 0 \ (i \neq j)
    \end{align*}
    として,
    \begin{align*}
        P E_i^\dagger E_j P = \alpha_{ij} P
    \end{align*}
    を満たすので, 3qubitビット反転符号は, 誤り訂正可能.
\end{ex}

\begin{ex}
    \label{ex10.8}
    計算すると,
    \begin{align*}
        \alpha_{ij} = \delta_{ij}
    \end{align*}
    となり, $\alpha_{ij}$はHermiteなので, 量子誤り訂正条件を満たす.
\end{ex}

\begin{ex}
    \label{ex10.9}
    $I = E_1, P_i = E_{2i}, Q_i = E_{2i+1}$として, 計算すると,
    \begin{align*}
        \alpha_{ij} =
        \begin{pmatrix}
            1   & 1/2 & 1/2 & 1/2 & 1/2 & 1/2 & 1/2 \\
            1/2 & 1/2 & 0   & 1/4 & 1/4 & 1/4 & 1/4 \\
            1/2 & 0   & 1/2 & 1/4 & 1/4 & 1/4 & 1/4 \\
            1/2 & 1/4 & 1/4 & 1/2 & 0   & 1/4 & 1/4 \\
            1/2 & 1/4 & 1/4 & 0   & 1/2 & 1/4 & 1/4 \\
            1/2 & 1/4 & 1/4 & 1/4 & 1/4 & 1/2 & 0   \\
            1/2 & 1/4 & 1/4 & 1/4 & 1/4 & 0   & 1/2 \\
        \end{pmatrix}
    \end{align*}
    となり, $\alpha_{ij}$はHermiteなので, 量子誤り訂正条件を満たす.
\end{ex}

\begin{ex}
    \label{ex10.10}
    定理10.1の証明と全く同様にやれば良い.
\end{ex}

\begin{ex}
    \label{ex10.11}
    分極解消チャンネルで$p=1$として, その演算要素は,
    \begin{align*}
        \left\{ \frac{1}{2} I, \frac{1}{2}X, \frac{1}{2}Y, \frac{1}{2}Z\right\}.
    \end{align*}
\end{ex}

\begin{ex}
    \label{ex10.12}
    分極解消チャンネル
    \begin{align*}
        \qo\left( \rho\right) =
        \left(1-p\right) \rho
        +
        \frac{p}{3}
        \left(
        X \rho X + Y \rho Y + Z \rho Z
        \right)
    \end{align*}
    に対して, $\braket{0|X|0} = \braket{0|Y|0} = 0$, $\braket{0|Z|0} = 1$を用いて,
    \begin{align*}
        F\left( \ket{0}, \qo\left( \ket{0}\bra{0}\right)\right)
        =
        \sqrt{\braket{0 | \qo(\rho)|0}}
        =
        \sqrt{1 - \frac{2p}{3}}.
    \end{align*}
\end{ex}

\begin{ex}
    \label{ex10.13}
    規格化された純粋状態$\ket{\psi} = a \ket{0} + b \ket{1}$に対して, 演習\ref{ex8.22}より,
    \begin{align*}
        \qo_{AD}\left( \ket{\psi}\bra{\psi} \right)
        =
        \begin{pmatrix}
            1- |b|^2(1 - \gamma) & ab^* \sqrt{1-\gamma}       \\
            a^*b\sqrt{1- \gamma} & |b|^2\left(1-\gamma\right)
        \end{pmatrix}
    \end{align*}
    なので,
    \begin{align*}
        F \left( \ket{\psi}\bra{\psi},  \qo_{AD}\left( \ket{\psi}\bra{\psi} \right)\right)
         & =
        \sqrt{\braket{\psi| \qo_{AD}\left( \ket{\psi}\bra{\psi} \right) | \psi}}                                             \\
         & =
        \sqrt{
            |a|^4 - |a|^2 |b|^2 \left( 1 - \gamma\right) + 2 |a|^2 |b|^2 \sqrt{1 - \gamma} + |b|^4 \left( 1 - \gamma\right)} \\
         & =
        \sqrt{
            2\left( 1 - \gamma - \sqrt{1 - \gamma} \right) |a|^4 + \left( 1 - 3 (1 - \gamma) + 2 \sqrt{1 - \gamma}\right) |a|^2 + 1 - \gamma}.
    \end{align*}
    $\left( 1 - \gamma - \sqrt{1 - \gamma} \right) \leq 0$より, $F \left( \ket{\psi}\bra{\psi},  \qo_{AD}\left( \ket{\psi}\bra{\psi} \right)\right)$の$0 \leq |a|^2 \leq 1$における最小値の候補は, $|a|^2 = 0, 1$の時に絞られる. 実際最小値は, $|a|^2 = 0$のときの$\sqrt{1 - \gamma}$. (9.121)式にあるように, 最小忠実度を考えるのには, 純粋状態を考えれば十分なので,振幅ダンピングの最小忠実度は$\sqrt{1-\gamma}$.
\end{ex}

\begin{ex}
    \label{ex10.14}
    \begin{align*}
        G =
        \begin{pmatrix}
            I_{k \times k} \\
            I_{k \times k} \\
            \vdots         \\
            I_{k \times k}
        \end{pmatrix}.
    \end{align*}
\end{ex}

\begin{ex}
    \label{ex10.15}
    生成行列$G = \left( \bm{g}_1, \bm{g}_2, ... ,  \bm{g}_k\right), G' = \left( \bm{g}_1 +
        \bm{g}_2, \bm{g}_2, ... ,  \bm{g}_k\right)$を用いて, $\bm{x} = \left( x_1 , ..., x_k\right)^\top$を符号化することを考えると,
    \begin{align*}
        G \bm{x}  & = \sum_{i=1}^k x_i \bm{g}_i                                    \\
        G' \bm{x} & = \left(x_1 + x_2\right) \bm{g}_1 + \sum_{i=2}^k x_i \bm{g}_i.
    \end{align*}
    よって,
    \begin{align*}
        \left\{ G \bm{x} | \bm{x} \in F_2^k \right\} = \left\{ G' \bm{x} | \bm{x} \in F_2^k \right\}.
    \end{align*}
\end{ex}

\begin{ex}
    \label{ex10.16}
    パリティ検査行列
    \begin{align*}
        H =
        \begin{pmatrix}
            \bm{y}_1^\top     \\
            \bm{y}_1^\top     \\
            \vdots            \\
            \bm{y}_{n-k}^\top \\
        \end{pmatrix},\
        H' =
        \begin{pmatrix}
            \bm{y}_1^\top                 \\
            \bm{y}_1^\top + \bm{y}_2^\top \\
            \vdots                        \\
            \bm{y}_{n-k}^\top             \\
        \end{pmatrix}
    \end{align*}
    として,
    \begin{align*}
        \Ker{H} = \left\{ \bm{x} \in F_2^k | \bm{y}_i^\top \bm{x} = 0 \ \left( i = 1, 2, ... , n-k\right)\right\} = \Ker{H'}.
    \end{align*}
\end{ex}

\begin{ex}
    \label{ex10.17}
    $G$の列ベクトルに直交する線型独立なベクトル$\bm{y}_i$を4本取ってきて, $\bm{y}_i$を転置した横ベクトルたちを行ベクトルとして持つのが, パリティ検査行列$H$. 例えば次のようなものが考えられる.
    \begin{align*}
        H =
        \begin{pmatrix}
            1 & 1 & 0 & 0 & 0 & 0 \\
            0 & 1 & 1 & 0 & 0 & 0 \\
            0 & 0 & 0 & 1 & 1 & 0 \\
            0 & 0 & 0 & 0 & 1 & 1 \\
        \end{pmatrix}.
    \end{align*}
\end{ex}

\begin{ex}
    \label{ex10.18}
    $G$の各列$\bm{g}_i$, $H$の各行$\bm{h}_i$として,
    \begin{align*}
        \left( HG\right)_{ij} = \bm{h}_i^\top \bm{g}_j = 0.
    \end{align*}
\end{ex}

\begin{ex}
    \label{ex10.19}
    パリティ検査行列の標準形$H = \left( A | I_{n-k}\right)$の$A$の各列
    \begin{align*}
        \bm{a}_i =
        \begin{pmatrix}
            a_{1,i} \\
            a_{2,i} \\
            \vdots  \\
            a_{n-k, i}
        \end{pmatrix}
    \end{align*}
    は互いに線型独立である. したがって, $\Ker{H}$の基底$\left\{ \bm{g}_j \right\}_{j=1}^k$として, 第$j$成分だけが1でそれ以外の成分が0のベクトル$\bm{e}_j$を用いて,
    \begin{align*}
        \bm{g}_j
        =
        \begin{pmatrix}
            \bm{e}_j   \\
            - a_{1, j} \\
            - a_{2, j} \\
            \vdots     \\
            - a_{n-k, j}
        \end{pmatrix}
    \end{align*}
    をとることができる. よって, 生成行列は,
    \begin{align*}
        G =
        \begin{pmatrix}
            \bm{g}_1, ..., \bm{g}_k
        \end{pmatrix}
        =
        \begin{pmatrix}
            I_{k} \\
            -A
        \end{pmatrix}.
    \end{align*}
\end{ex}

\begin{ex}
    \label{ex10.20}
    パリティ検査行列$H = \left( \bm{h}_1, \bm{h}_2, ... ,  \bm{h}_k\right)$, 符号化された状態$\bm{x} = \left( x_1 , ..., x_k\right)^\top$とする.
    \par
    ある$\bm{x}$に対して, $\wt \bm{x} \le d - 1$と仮定すると, ある$l \le d - 1$なる$l$を用いて,
    \begin{align*}
        H \bm{x} = \sum_{ i = i_1, i_2 , ... , i_l} \bm{h}_i
    \end{align*}
    となるが, $\bm{h}_{i_1} , ... , \bm{h}_{i_l}$は互いに線型独立なので, $ H \bm{x} = \bm{0}$になりえず, 矛盾. よって, 全ての$\bm{x}$に対して, $\wt \bm{x} \ge d$.
    \par
    ある$(i_1, ..., i_{d-1}, i_{d})$の組に対して, $\bm{h}_{i_1} , ... , \bm{h}_{i_{d-1}}, \bm{h}_{i_d}$が線型従属になり, どの$d$本も線型独立なので,
    \begin{align*}
        \bm{h}_{i_d} = \bm{h}_{i_1} +  ... + \bm{h}_{i_{d-1}}.
    \end{align*}
    よって, 成分$i = i_1, i_2 , ..., i_d$が1になる$\bm{x}$, つまり$\wt \bm{x} = d$となる$\bm{x}$が,
    \begin{align*}
        H \bm{x} = \sum_{ i = i_1, i_2 , ... , i_d} \bm{h}_i = \bm{h}_{i_d} + \bm{h}_{i_d} = \bm{0}
    \end{align*}
    を満たす. つまり, $\wt \bm{x} = d$を満たす, $H$で定義された符号$\bm{x}$が存在.
    \par
    以上より,
    \begin{align*}
        d \left(H \right) = d.
    \end{align*}
\end{ex}

\begin{ex}
    \label{ex10.21}
    演習\ref{ex10.20}より,
    \begin{align*}
        d - 1 = ( H の任意のベクトルの組が線型独立になるときのそのベクトルの本数の最大値) \le \rank(H) = n - k.
    \end{align*}
\end{ex}

\begin{ex}
    \label{ex10.22}
    Hamming codeのどの2列を見ても線型独立だが, ある3列を見ると線形従属なので, 演習\ref{ex10.20}より,  Hamming codeの距離は3.
\end{ex}

\begin{ex}
    \label{ex10.23}
    $ n > 4t > 0 $のとき,
    \begin{align*}
        x \log \frac{2t}{n} + \left( n - x \right) \log \frac{ n - 2t}{n}
    \end{align*}
    は$x$について単調減少なので,
    \begin{align*}
        \left( \frac{2t}{n}\right)^{x} \left( \frac{ n - 2t}{n}\right)^{n - x}
    \end{align*}
    も$x$について単調減少. すると,  $ n > 4t > 0 $のとき,
    \begin{align*}
        \frac{2^{n H\left(\frac{2t}{n}\right)}}{\sum_{j=0}^t {でもない}オペレータ._n C_j}
         & =
        \frac{1}{\sum_{j=0}^t {}_n C_j \left( \frac{2t}{n}\right)^{2t} \left(\frac{n-2t}{n} \right)^{n-2t}}
        \geq
        \frac{1}{\sum_{j=0}^{2t} {}_n C_j \left( \frac{2t}{n}\right)^{2t} \left(\frac{n-2t}{n} \right)^{n-2t}}
        \geq
        \frac{1}{\sum_{j=0}^{2t} {}_n C_j \left( \frac{2t}{n}\right)^{j} \left(\frac{n-2t}{n} \right)^{n-j}}= 1
    \end{align*}
    が成立するので, 任意の$n (>4t > 0)$に対して,
    \begin{align*}
        2^{n H\left(\frac{2t}{n}\right)} \ge 2^{n-k} \ge \sum_{j=0}^t {}_n C_j
    \end{align*}
    なる$k$が存在する.
    つまり,
    \begin{align*}
        \begin{cases}
            \frac{k}{n} \ge 1 - H \left( \frac{2t}{n}\right) \\
            \sum_{j=0}^t {}_n C_j 2^k \le 2^n
        \end{cases}
        \longleftrightarrow
        \begin{cases}
            \frac{k}{n} \ge 1 - H \left( \frac{2t}{n}\right) \\
            t ビット以下の誤りから護る[n,k]符号が存在.
        \end{cases}
    \end{align*}
    を満たす$k$が存在する.
\end{ex}

\begin{ex}
    \label{ex10.24}
    \begin{align*}
        C \subseteq C^\perp
        \longleftrightarrow
        \forall_{x \in F_2^k} \ G[C]x \in C^\perp
        \longleftrightarrow
        \forall_{x \in F_2^k} \ H[C^\perp]G[C] x =  G^\top[C]G[C] x = 0
        \longleftrightarrow
        G^\top[C]G[C] = O.
    \end{align*}
\end{ex}

\begin{ex}
    \label{ex10.25}
    $x \in C^\perp$のとき,
    \begin{align*}
        \forall_{y \in C} \ x \cdot y = 0
    \end{align*}
    なので,
    \begin{align*}
        \sum_{y \in C} (-1)^{x \cdot y } = |C|.
    \end{align*}
    一方, $x \notin C^\perp$のとき, $C$の生成行列$G[C] = \left( g_1, g_2, ..., g_k\right)$として,
    $x \cdot g_i = 0$となる$g_i$が存在するので, ある$l \ge 1$が存在して,
    \begin{align*}
        x \cdot g_1     & = ... = x \cdot g_l = 1 \\
        x \cdot g_{l+1} & = ... = x \cdot g_k = 0
    \end{align*}
    と書いても, 一般性を失わない.
    任意の$C$の符号語$y$が$ y = G[C] a = \sum_i a_i g_i$とかけるので,
    \begin{align*}
        \sum_{y \in C} (-1)^{x \cdot y} = \sum_{a_1, a_2, ..., a_k = 0,1}  (-1)^{\sum_i a_i x \cdot g_i}
        = \sum_{a_1, a_2, ... , a_l = 0, 1}  (-1)^{a_1 + a_2, ... + a_l} = \prod_{i=1}^k \sum_{a_i = 0, 1} (-1)^{a_i} = 0.
    \end{align*}
\end{ex}

\begin{ex}
    \label{ex10.26}
    始状態$\ket{x}\ket{0}$の$\ket{x}$を第1レジスタ, $\ket{0}$を第2レジスタと呼ぶ.
    パリティ検査行列$H$の$(i,j)$成分が$H_{ij}$が$1$となる$(i,j)$に対に対して,
    第1レジスタの第$i$ qubitを制御ビットとして, 第2レジスタの第$j$ qubitを反転させるゲートを考えれば良い.
\end{ex}

\begin{ex}
    \label{ex10.27}
\end{ex}

\begin{ex}
    \label{ex10.28}
    \begin{align*}
        H[C_1]
        \begin{pmatrix}
            1 & 0 & 0 & 0 \\
            0 & 1 & 0 & 0 \\
            0 & 0 & 1 & 0 \\
            0 & 0 & 0 & 1 \\
            0 & 1 & 1 & 1 \\
            1 & 0 & 1 & 1 \\
            1 & 1 & 0 & 1
        \end{pmatrix}
        =
        \begin{pmatrix}
            0 & 0 & 0 & 1 & 1 & 1 & 1 \\
            0 & 1 & 1 & 0 & 0 & 1 & 1 \\
            1 & 0 & 1 & 0 & 1 & 0 & 1 \\
        \end{pmatrix}
        \begin{pmatrix}
            1 & 0 & 0 & 0 \\
            0 & 1 & 0 & 0 \\
            0 & 0 & 1 & 0 \\
            0 & 0 & 0 & 1 \\
            0 & 1 & 1 & 1 \\
            1 & 0 & 1 & 1 \\
            1 & 1 & 0 & 1
        \end{pmatrix}
        =
        O
    \end{align*}
    なので,
    \begin{align*}
        G [ C_1]
        =
        \begin{pmatrix}
            1 & 0 & 0 & 0 \\
            0 & 1 & 0 & 0 \\
            0 & 0 & 1 & 0 \\
            0 & 0 & 0 & 1 \\
            0 & 1 & 1 & 1 \\
            1 & 0 & 1 & 1 \\
            1 & 1 & 0 & 1
        \end{pmatrix}
    \end{align*}
\end{ex}

\begin{ex}
    \label{ex10.29}
    任意の$g \in S$に対して, $\ket{\psi_1}, \ket{\psi_2} \in V_S$は,
    \begin{align*}
        g \ket{\psi_1} = \ket{\psi_1}, g \ket{\psi_2} = \ket{\psi_2}
    \end{align*}
    であることと, $g$の線形性より,
    \begin{align*}
        g \left( c_1 \ket{\psi_1} + c_2 \ket{\psi_2}\right) = c_1 \ket{\psi_1} + c_2 \ket{\psi_2}.
    \end{align*}
    よって, $V_S$の任意の2つの要素の線型結合もまた$V_S$の要素. 後半の主張については明らか.
\end{ex}

\begin{ex}
    \label{ex10.30}
    \begin{align*}
        - I \notin S
        \longleftrightarrow
        (\pm i I)^2 \notin S
        \longleftrightarrow
        (\pm i I) \notin S.
    \end{align*}
\end{ex}

\begin{ex}
    \label{ex10.31}
    \ \par
    $\longleftarrow)$ 明らか.
    \par
    $\longrightarrow)$
    任意の$S$の元が, 互いに可換な$S$の生成元での積でかけるので, $S$の要素も互いに可換.
\end{ex}

\begin{ex}
    \label{ex10.32}
    計算すれば良い.
\end{ex}

\begin{ex}
    \label{ex10.33}
    $r(g)$の各成分と$g$は, ある係数$c$を用いて
    \begin{align*}
        g = c \prod_{i=1}^n X_i^{r_i} Z_i^{r_{n+i}}
    \end{align*}
    で対応づけられる. よって,
    \begin{align*}
        [g, g'] = 0
         & \longleftrightarrow
        \prod_{i = 1}^n X_i^{r_i}
        Z_i^{r_{n+i}} X_i^{r'_i} Z_i^{r'_{n+i}}
        -
        \prod_{i = 1}^n
        X_i^{r'_i} Z_i^{r'_{n+i}}X_i^{r_i} Z_i^{r_{n+i}}
        =0
        \\
         & \longleftrightarrow
        \prod_{i = 1}^n
        (-1)^{r'_i r_{n+i}} X_i^{r_i} X_i^{r'_i} Z_i^{r_{n+i}}  Z_i^{r'_{n+i}}
        -
        \prod_{i = 1}^n
        (-1)^{r_i r'_{n+i}} X_i^{r'_i} X_i^{r_i}Z_i^{r'_{n+i}} Z_i^{r_{n+i}}
        =0
        \\
         & \longleftrightarrow
        \prod_{i = 1}^n (-1)^{r'_i r_{n+i}} =
        \prod_{i = 1}^n (-1)^{r_i r'_{n+i}}
        \\
         & \longleftrightarrow
        \sum_{i=1}^n r'_i r_{n+i} + r_i r'_{n+i} = 0
        \\
         & \longleftrightarrow
        r(g) \Lambda r(g')^\top = 0.
    \end{align*}
\end{ex}

\begin{ex}
    \label{ex10.34}
    示すべきは
    \begin{align*}
        -I \in S
        \longleftrightarrow
        \exists_i \ g_i^2 = -I または g_i = -I
    \end{align*}
    \par
    $\longleftarrow)$ 明らか.
    \par
    $\longrightarrow)$
\end{ex}

\begin{ex}
    \label{ex10.35}
    $G_n$は反HermiteまたはHermiteを要素にもち,
    任意の$g \in S \subset G_n$
    に対して
    \begin{align*}
        g^2 = \pm I^{\otimes n}
    \end{align*}
    なので,
    \begin{align*}
        - I^{\otimes n} \in S \subset G_n
        \longrightarrow
        \forall g  = \bigotimes_{ i = 1, 2, ... , n} g_i \in S \subset G_n \ \
        g^2 = \bigotimes_{ i = 1, 2, ... , n} g_i^2 = I^{\otimes n} \ かつ \ g \neq  - I^{\otimes n}
    \end{align*}
    であるので, $\{g_i\}_{i=1}^n$のうちの偶数個が, 2乗すると$-I$になる反Hermiteであり, 残りが2乗すると$I$になるHermiteになることが言える. よって,
    \begin{align*}
        g  = \bigotimes_{ i = 1, 2, ... , n} g_i
    \end{align*}
    はHermite.
\end{ex}

\begin{ex}
    \label{ex10.36}
    量子回路を考えれば明らか.
\end{ex}

\begin{ex}
    \label{ex10.37}
    \begin{align*}
        UY_1 U^\dagger =
        i U X_1 U^\dagger U  Z_1 U^\dagger
        =
        i X_1 X_2 Z_1
        =
        Y_1 Z_2
    \end{align*}
\end{ex}

\begin{ex}
    \label{ex10.38}
    $\Sigma = X_1, X_2, Z_1, Z_2$
    として,
    \begin{align*}
        U \Sigma U^\dagger = V \Sigma V^\dagger
        \longleftrightarrow
        [ U^\dagger V,  \Sigma ] = 0
    \end{align*}
    とすると, $i, j = 0, 1,2,3$に対して,
    \begin{align*}
        [ U^\dagger V , \sigma_i \otimes \sigma_j ] = 0
    \end{align*}
    なので, $U^\dagger V$は任意のオペレータと可換. つまり, $U^\dagger V = I $. ゆえに$U = V$.
\end{ex}

\begin{ex}
    \label{ex10.39}
    計算すれば良い.
\end{ex}

\begin{ex}
    \label{ex10.40}
    (1)\
    グローバルな位相を考えないので, $G_1$の要素$g$として$g = X, Y, Z$だけを考えて, $UgU \in G_1$なる$U$がHadamardゲート$H$と位相ゲート$S$でかけることを言えば良い.
    たとえば$U=H$とすると,
    \begin{align*}
        HXH^\dagger = Z, HYH^\dagger = Y, HZH = X
    \end{align*}
    なので, グローバル位相を無視すれば$HG_1H \in G_1$. $U = S$としても同様.
    \par
    (2)

    \begin{align*}
        \Qcircuit @C=1em @R=1em {
         & \qw    & \qw       & \ctrl{1}  & \gate{H} & \ctrl{1} & \qw \\
         & {/}\qw & \gate{U'} & \gate{g'} & \qw      & \gate{g} & \qw
        }
    \end{align*}
    の作用を, 始状態$\ket{0}\otimes \ket{\psi} + \ket{1} \otimes \ket{\phi}$
    として考えると,
    \begin{align*}
        \ket{0}\otimes \ket{\psi} + \ket{1} \otimes \ket{\phi}
         & \to
        \ket{0}\otimes U'\ket{\psi} + \ket{1} \otimes U'\ket{\phi}                          \\
         & \to
        \ket{0}\otimes U'\ket{\psi} + \ket{1} \otimes g'U'\ket{\phi}                        \\
         & \to
        \frac{\ket{0}+\ket{1}}{\sqrt{2}}\otimes U'\ket{\psi} +
        \frac{\ket{0}-\ket{1}}{\sqrt{2}} \otimes g'U'\ket{\phi}                             \\
         & \to
        \frac{\ket{0}\otimes U'\ket{\psi}+\ket{1}\otimes gU'\ket{\psi}}{\sqrt{2}} +
        \frac{\ket{0}\otimes g'U'\ket{\phi}-\ket{1}\otimes gg'U'\ket{\phi}}{\sqrt{2}}       \\
         & \to
        \frac{ ( I + X_1 \otimes g)(\ket{0}\otimes U'\ket{\psi})}{\sqrt{2}} +
        \frac{ ( I - X_1 \otimes g)(Z_1 \otimes g')(\ket{0}\otimes U'\ket{\psi})}{\sqrt{2}} \\
         & =
        \frac{ ( I + UZ_1 U^\dagger )(\ket{0}\otimes U'\ket{\psi})}{\sqrt{2}} +
        \frac{ ( I - UZ_1 U^\dagger)UX_1U^\dagger(\ket{0}\otimes U'\ket{\psi})}{\sqrt{2}}   \\
    \end{align*}
\end{ex}

\begin{ex}
    \label{ex10.41}
    計算すれば良い.
\end{ex}

\begin{ex}
    \label{ex10.42}
    \begin{align*}
        \Qcircuit @C=1em @R=1em {
         & \lstick{\ket{\psi}} & \ctrl{1} & \gate{H} & \meter & \cw           & \control \cw  &     &                     \\
         & \lstick{}           & \targ    & \qw      & \meter & \control \cw  & \cwx          &     &                     \\
         & \lstick{}           & \qw      & \qw      & \qw    & \gate{X} \cwx & \gate{Z} \cwx & \qw & \rstick{\ket{\psi}}
        \gategroup{2}{2}{3}{3}{.7em}{\{}
        }
    \end{align*}
    AliceとBobに分け与えるEPR対を$\ket{\beta_{00}} = \frac{\ket{00} + \ket{11}}{\sqrt{2}}$とする. また, Bobに届けたい状態$\ket{\psi} = \ket{+}$とする.上記の量子回路の各ゲートの作用を固定部分群形式で追っていく.
    \par
    \begin{enumerate}
        \item 始状態$\ket{+}\ket{\beta_{00}}$ \par
              $< X \otimes X \otimes I,\  I \otimes X \otimes X,\  I \otimes Z \otimes Z>$
        \item CNOT作用後 \par
              $< X \otimes X \otimes I,\ I \otimes X \otimes X,\ Z \otimes Z \otimes Z>$
        \item $H$作用後 \par
              $< Z \otimes X \otimes I,\ I \otimes X \otimes X,\ X \otimes Z \otimes Z>$
        \item 生成元のうち1つだけ$I \otimes Z \otimes I$と反可換になるように生成元を取り替える \par
              $< Z \otimes X \otimes I,\ Z \otimes I \otimes X,\ X \otimes Z \otimes Z>$
        \item 2qubit目の測定($M_2$) \par
              if $M_2 = 0$,  $< I \otimes Z \otimes I,\ Z \otimes I \otimes X,\ X \otimes Z \otimes Z>$ \\
              if $M_2 = 1$,  $< - I \otimes Z \otimes I,\ Z \otimes I \otimes X,\ X \otimes Z \otimes Z>$
        \item $I \otimes I \otimes X^{M_2}$ 作用後 \par
              if $M_2 = 0$,  $< I \otimes Z \otimes I,\ Z \otimes I \otimes X,\ X \otimes Z \otimes Z>$ \\
              if $M_2 = 1$,  $< - I \otimes Z \otimes I,\ Z \otimes I \otimes X,\ - X \otimes Z \otimes Z>$
        \item 1qubit目の測定($M_1$)をして, $I \otimes I \otimes Z^{M_1}$ 作用後\par
              if $(M_1, M_2) = (0,0)$,  $< I \otimes Z \otimes I,\ Z \otimes I \otimes X,\ Z \otimes I \otimes I>$ \\
              if $(M_1, M_2) = (0,1)$,  $< -I \otimes Z \otimes I,\ Z \otimes I \otimes X,\ -Z \otimes I \otimes I>$ \\
              if $(M_1, M_2) = (1,0)$,  $< I \otimes Z \otimes I,\ - Z \otimes I \otimes X,\ - Z \otimes I \otimes I>$ \\
              if $(M_1, M_2) = (1,1)$,  $< -I \otimes Z \otimes I,\ -Z \otimes I \otimes X,\ - Z \otimes I \otimes I>$ \\
    \end{enumerate}
    終状態の固定部分群の生成元は$\ket{+}$を安定化させる. 
\end{ex}

\begin{ex}
    \label{ex10.43}
    $g \in S, \ket{\psi} \in V_S$とする. すると, 任意の$h \in S$に対して, 
    \begin{align*}
        h g \ket{\psi} = h \ket{\psi} = \ket{\psi} = g \ket{\psi}
        \longrightarrow
        g^{-1} h g \ket{\psi} = \ket{\psi}
        \longrightarrow
        g^{-1} h g \in S
    \end{align*}
    なので, $g \in N(S)$. つまり, $S \subset N(S)$.
\end{ex}

\begin{ex}
    \label{ex10.44}
    $\longrightarrow)$
    \begin{align*}
        g \in N(S) \longrightarrow \forall s \in S \ gs = sg \longrightarrow gS = Sg \longrightarrow g \in N(S)
    \end{align*}
    \par
    $\longleftarrow)$
    $g \in N(S)$とする. $-I \notin S$なので, $S$の要素はだから, $\forall s \in S \ gs = sg$. よって, $g \in Z(S)$.

\end{ex}

\begin{ex}
    \label{ex10.45}
    
\end{ex}


